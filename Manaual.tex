
\documentclass[twoside]{article}
\usepackage[colorlinks,linkcolor=black]{hyperref}
\usepackage{xeCJK}
\usepackage{fancyhdr}
\usepackage{amsmath, amsthm}
\usepackage{listings,xcolor}
\usepackage{geometry}
\usepackage{fontspec}
\setsansfont{Monaco}
\setmonofont[Mapping={}]{Monaco}
\newcommand{\HRule}{\rule{\linewidth}{0.5mm}}
\geometry{left=2.5cm,right=2.5cm,top=2.5cm,bottom=2.5cm}
\lstset{
    language    = c++,
    breaklines  = true,
    captionpos  = b,
    tabsize     = 4,
    numbers     = left,
    columns     = fullflexible,
    keepspaces  = true,
    commentstyle = \color[RGB]{0,128,0},
    keywordstyle = \color[RGB]{0,0,255},
    basicstyle   = \small\ttfamily,
    rulesepcolor = \color{red!20!green!20!blue!20},
    showstringspaces = false,
}
\title{ACM/ICPC Template Manaual}
\author{Happy Otaku}
\pagestyle{fancy}
\fancyhf{}
\fancyhead[C]{ACM/ICPC Template Manaual, Happy Otaku}
\begin{document}\small
\begin{titlepage}
\begin{center}
\vspace*{0.5cm}\includegraphics[width=0.75\textwidth]{logo.jpg} \\ [2cm]
\HRule \\ [1cm]
\textbf{\Huge{ACM/ICPC Template Manaual}} \\ [0.5cm]
\HRule \\ [4cm]
\textbf{\Huge{ZheJiang GongShang University}} \\ [1cm]
\LARGE{Happy Otaku}
\vfill
\Large{\today}
\end{center}
\clearpage
\end{titlepage}
\tableofcontents\clearpage
\pagestyle{fancy}
\lfoot{}
\cfoot{\thepage}\rfoot{}
\setcounter{section}{-1}
\setcounter{page}{1}
\clearpage\section{头文件}
\begin{lstlisting}
// 巨菜的ACMer-Happy233

#include <bits/stdc++.h>

using namespace std;

//-----
typedef double db;
typedef long long ll;
typedef vector<int> vi;
typedef pair<int, int> pii;
typedef pair<ll, ll> pll;
#define fi first
#define se second
#define mp make_pair
#define pb push_back
#define pw(x) (1ll << (x))
#define sz(x) ((int)(x).size())
#define all(x) (x).begin(),(x).end()
#define rep(i, l, r) for(int i=(l);i<(r);++i)
#define per(i, l, r) for(int i=(r)-1;i>=(l);--i)
#define sf(x) scanf("%d", &(x))

const double pi = acos(-1);\end{lstlisting}
\clearpage\section{字串符}
\subsection{KMP}
\begin{lstlisting}
template<class elemType>
inline void NEXT(elemType &T, vector<int> &next) {
    next[0] = -1;
    for (int i = 1; i < T.size(); i++) {
        int j = next[i - 1];
        while (j >= 0 && T[i - 1] != T[j]) j = next[j];
        if (j >= 0 && T[i - 1] == T[j]) next[i] = j + 1;
        else next[i] = 0;
    }
}

template<class elemType>
inline int COUNT_KMP(elemType &S, elemType &T) {
    vector<int> next(T.size());
    NEXT(T, next);
    int index, count = 0;
    for (index = 0; index < S.size(); ++index) {
        int pos = 0;
        int iter = index;
        while (pos < T.size() && iter < S.size()) {
            if (S[iter] == T[pos]) {
                ++iter;
                ++pos;
            } else {
                if (pos == 0) ++iter;
                else pos = next[pos - 1] + 1;
            }
        }
        if (pos == T.size() && (iter - index) == T.size()) ++count;
    }
    return count;
}

template<class elemType>
inline void NEXT(elemType T[], int count, vector<int> &next) {
    next[0] = -1;
    for (int i = 1; i < count; i++) {
        int j = next[i - 1];
        while (j >= 0 && T[i - 1] != T[j]) j = next[j];
        if (j >= 0 && T[i - 1] == T[j]) next[i] = j + 1;
        else next[i] = 0;
    }
}

template<class elemType>
inline int COUNT_KMP(elemType S[], int c1, elemType T[], int c2) {
    vector<int> next(c2);
    NEXT(T, c2, next);
    int index, count = 0;
    for (index = 0; index < c1; ++index) {
        int pos = 0;
        int iter = index;
        while (pos < c2 && iter < c1) {
            if (S[iter] == T[pos]) {
                ++iter;
                ++pos;
            }
            else {
                if (pos == 0) ++iter;
                else pos = next[pos - 1] + 1;
            }
        }
        if (pos == c2 && (iter - index) == c2) ++count;
    }
    return count;
}
\end{lstlisting}
\clearpage\section{动态规划}
\subsection{01Bag}
\begin{lstlisting}
void dp(int n, int m) {
    // n=物品个数
    for (int i = 0; i < n; i++) {
        // m=背包最大容量
        for (int j = m; j >= wei[i]; j--)
            // wei=大小 val=价值
            f[j] = max(f[j], f[j - wei[i]] + val[i]);
    }
}
\end{lstlisting}
\subsection{BagProblem}
\begin{lstlisting}
#define N 1000
// val=价值 wei=重量 num=数量
int val[N], wei[N], num[N], f[N];
// n=种类个数 m=背包最大值

// 01背包
void dp1(int n, int m) {
    for (int i = 0; i < n; i++) {
        for (int j = m; j >= wei[i]; j--)
            f[j] = max(f[j], f[j - wei[i]] + val[i]);
    }
}

// 完全背包
void dp2(int n, int m) {
    //初始化看要求
    for (int i = 0; i <= m; i++) {
        f[i] = INF;
    }
    f[0] = 0;
    //若要求恰好装满背包,那在初始化时除了f[0]=0其它f[1..V]均=-∞
    //若没要求背包装满,只希望价格大,初始化时应将f[0..V]=0)
    for (int i = 0; i < n; i++)
        for (int j = wei[i]; j <= m; j++)
            f[j] = max(f[j], f[j - wei[i]] + val[i]);
}

// 多重背包
void dp3(int n, int m) {
    for (int i = 0; i < n; i++)
        for (int k = 0; k < num[i]; k++)
            for (int j = m; j >= wei[i]; j--)
                f[j] = max(f[j], f[j - wei[i]] + val[i]);
}
\end{lstlisting}
\subsection{FullBag}
\begin{lstlisting}
/* 
完全背包问题的特点是,每种物品可以无限制的重复使用,可以选择放或不放。 
完全背包问题描述: 
有N物品和一个容量为V的背包。第i件物品的重量是wei[i],价值是val[i]。 
*/  
  
#include <cstdio>  
#define INF 0x3fffffff  
#define N 10047  
int f[N],val[N],wei[N];  
int min(int a,int b)  
{  
    return x<y?x:y;  
}  
int main()  
{  
    int t,i,j,k,E,F,m,n;  
    scanf("%d",&t);  
    while(t--)  
    {  
        scanf("%d%d",&E,&F);  
        int c = F-E;  
        for(i = 0 ; i <= c ; i++)  
            f[i]=INF;  
        scanf("%d",&n);  
        for(i = 0 ; i < n ; i++)  
        {  
            scanf("%d%d",&val[i],&wei[i]);//val[i]为面额,wei[i]为重量  
        }  
        f[0]=0;//因为此处假设的是小猪储钱罐 恰好装满 的情况  
        //注意初始化(要求恰好装满背包,那么在初始化时除了f[0]为0其它f[1..V]均设为-∞,  
        //这样就可以保证最终得到的f[N]是一种恰好装满背包的最优解。  
        //如果并没有要求必须把背包装满,而是只希望价格尽量大,初始化时应该将f[0..V]全部设为0)  
        for(i =0 ; i < n ; i++)  
        {  
            for(j = wei[i] ; j <= c ; j++)  
            {  
                f[j] = min(f[j],f[j-wei[i]]+val[i]);//此处求的是最坏的情况所以用min,确定最少的钱,当然最后就用max了,HEHE  
            }  
        }  
        if(f[c] == INF)  
            printf("This is impossible.\n");  
        else  
            printf("The minimum amount of money in the piggy-bank is %d.\n",f[c]);  
    }  
    return 0;  
}  
//此代码为HDU1114;  \end{lstlisting}
\subsection{MultiBag}
\begin{lstlisting}
//多重背包(MultiplePack): 有N种物品和一个容量为V的背包。  
//第i种物品最多有n[i]件可用,每件费用是c[i],价值是w[i]。  
//求解将哪些物品装入背包可使这些物品的费用总和不超过背包容量,  
//且价值总和最大。  
//HDU 2191  
  
#include <cstdio>  
#include <cstring>  
#define N 247  
int max(int x,int y){
    return x>y?x:y;  
}  
int main()  {  
    int t,n,m,i,j,k;  
    int w[N],pri[N],num[N],f[N];  
    while(~scanf("%d",&t)){  
        while(t--){  
            memset(f,0,sizeof(f));  
            scanf("%d%d",&n,&m);//n为总金额,m为大米种类  
            for(i = 0 ; i < m ; i++){  
                scanf("%d%d%d",&pri[i],&w[i],&num[i]);//num[i]为每种大米的袋数  
            }  
            for(i = 0 ; i < m ; i++){  
                for(k = 0 ; k < num[i] ; k++){  
                    for(j = n ; j >= pri[i]; j--){  
                        f[j] = max(f[j],f[j-pri[i]]+w[i]);  
                    }  
                }  
            }  
            printf("%d\n",f[n]);  
        }  
    }  
    return 0;  
}  \end{lstlisting}
\clearpage\section{数据结构}
\subsection{BTree}
\begin{lstlisting}
template<class T>

struct TreeNode {
    T value;
    TreeNode *left;
    TreeNode *right;
};

template<class T>
TreeNode<T> *createTree(const T *pre, const T *in, const int len) {
    TreeNode<T> *t = NULL;
    if (len > 0) {
        t = new TreeNode<T>;
        t->value = pre[0];
        int index;
        for (index = 0; index < len; index++) {
            if (in[index] == pre[0]) {
                break;
            }
        }
        if (index == len) {
            index = -1;
        }
        t->left = createTree(pre + 1, in, index);
        t->right = createTree(pre + index + 1, in + index + 1, len - index - 1);
    }
    return t;
}

template<class T>
int preOrder(TreeNode<T> *root, queue<T> &out) {
    if (root) {
        int count = 1;
        out.push(root->value);
        count += preOrder(root->left, out);
        count += preOrder(root->right, out);
        return count;
    } else {
        return 0;
    }
}

template<class T>
int inOrder(TreeNode<T> *root, queue<T> &out) {
    if (root) {
        int count = 1;
        count += inOrder(root->left, out);
        out.push(root->value);
        count += inOrder(root->right, out);
        return count;
    } else {
        return 0;
    }
}

template<class T>
void postOrder(TreeNode<T> *root, queue<T> &out) {
    if (root) {
        postOrder(root->left, out);
        postOrder(root->right, out);
        out.push(root->value);
    } else {
        return;
    }
}

template<class T>
T *convertQueueToArray(queue<T> &out, int len) {
    T *list = new T[len];
    int now = 0;
    while (!out.empty() && now < len) {
        list[now] = out.front();
        out.pop();
        now++;
    }
    return list;
}

template<class T>
void destroyTree(TreeNode<T> *root) {
    if (root) {
        destroyTree(root->left);
        destroyTree(root->right);
        delete root;
    } else return;
}

template<class T>
void insertIntoBSTree(TreeNode<T> *root, const T &value) {
    if (!root) {
        return;
    }
    if (value < root->value) {
        if (root->left) {
            insertIntoTree(root->left, value);
        } else {
            root->left = new TreeNode<T>;
            root->left->value = value;
            root->left->left = NULL;
            root->left->right = NULL;
        }
    } else if (value > root->value) {
        if (root->right) {
            insertIntoTree(root->right, value);
        } else {
            root->right = new TreeNode<T>;
            root->right->value = value;
            root->right->left = NULL;
            root->right->right = NULL;
        }
    }
}

template<class T>
TreeNode<T> *createBSTree(T *list, int len) {
    if (len < 1) {
        return NULL;
    }
    TreeNode<T> *root = new TreeNode<char>;
    root->value = list[0];
    root->left = NULL;
    root->right = NULL;
    for (int i = 1; i < len; i++) {
        insertIntoBSTree(root, list[i]);
    }
    return root;
}\end{lstlisting}
\subsection{pbds-bbtree}
\begin{lstlisting}
// 红黑树
tree<int, null_type, less<int>, rb_tree_tag, tree_order_statistics_node_update> t;
// null_type无映射(低版本g++为null_mapped_type)
// 类似multiset
tree<int, null_type, less_equal<int>, rb_tree_tag, tree_order_statistics_node_update> t;

find_by_order(size_t order);
// 结点更新
tree_order_statistics_node_update
insert(p);
erase(it);
// 求k在树中是第几大:
order_of_key(p);
// 找到第order小的迭代器
find_by_order(order);
// 前驱
lower_bound(p);
// 后驱
upper_bound(p);
// 合并
a.join(b);
// 分割 key小于等于v的元素属于a,其余的属于b
a.split(v, b);
\end{lstlisting}
\clearpage\section{图论}
\clearpage\section{博弈}
\clearpage\section{分治}
\clearpage\section{其他}

\end{document}
