
\documentclass[twoside]{article}
\usepackage[colorlinks,linkcolor=black]{hyperref}
\usepackage{xeCJK}
\usepackage{fancyhdr}
\usepackage{amsmath, amsthm}
\usepackage{listings,xcolor}
\usepackage{geometry}
\usepackage{fontspec}
\setsansfont{Monaco}
\setmonofont[Mapping={}]{Monaco}
\newcommand{\HRule}{\rule{\linewidth}{0.5mm}}
\geometry{left=2.5cm,right=2.5cm,top=2.5cm,bottom=2.5cm}
\lstset{
    language    = c++,
    breaklines  = true,
    captionpos  = b,
    tabsize     = 4,
    numbers     = left,
    columns     = fullflexible,
    keepspaces  = true,
    commentstyle = \color[RGB]{0,128,0},
    keywordstyle = \color[RGB]{0,0,255},
    basicstyle   = \small\ttfamily,
    rulesepcolor = \color{red!20!green!20!blue!20},
    showstringspaces = false,
}
\title{ACM/ICPC Template Manaual}
\author{Happy Otaku}
\pagestyle{fancy}
\fancyhf{}
\fancyhead[C]{ACM/ICPC Template Manaual, Happy Otaku}
\begin{document}\small
\begin{titlepage}
\begin{center}
\vspace*{0.5cm}\includegraphics[width=0.75\textwidth]{logo.jpg} \\ [2cm]
\HRule \\ [1cm]
\textbf{\Huge{ACM/ICPC Template Manaual}} \\ [0.5cm]
\HRule \\ [4cm]
\textbf{\Huge{ZheJiang GongShang University}} \\ [1cm]
\LARGE{Happy Otaku}
\vfill
\Large{\today}
\end{center}
\clearpage
\end{titlepage}
\tableofcontents\clearpage
\pagestyle{fancy}
\lfoot{}
\cfoot{\thepage}\rfoot{}
\setcounter{section}{-1}
\setcounter{page}{1}
\clearpage\section{头文件}
\begin{lstlisting}
// 巨菜的ACMer-Happy233

#include <bits/stdc++.h>

using namespace std;

//-----
typedef long long ll;
typedef vector<int> vi;
typedef pair<int, int> pii;
#define pw(x) (1ll << (x))
#define sz(x) ((int)(x).size())
#define all(x) (x).begin(),(x).end()
#define rep(i, l, r) for(int i=(l);i<(r);++i)
#define per(i, l, r) for(int i=(r)-1;i>=(l);--i)
#define mst(t, v, n) memset(t, v, sizeof(decltype(*(t))) * (n))
#define sf(x) scanf("%d", &(x))
//---debug array *(ll(*)[10])ptr

using namespace std;

const double pi = acos(-1);\end{lstlisting}
\subsection{stdc++}
\begin{lstlisting}
// C
#ifndef _GLIBCXX_NO_ASSERT
#include <cassert>
#endif
#include <cctype>
#include <cerrno>
#include <cfloat>
#include <ciso646>
#include <climits>
#include <clocale>
#include <cmath>
#include <csetjmp>
#include <csignal>
#include <cstdarg>
#include <cstddef>
#include <cstdio>
#include <cstdlib>
#include <cstring>
#include <ctime>

#if __cplusplus >= 201103L
#include <ccomplex>
#include <cfenv>
#include <cinttypes>
#include <cstdalign>
#include <cstdbool>
#include <cstdint>
#include <ctgmath>
#include <cuchar>
#include <cwchar>
#include <cwctype>
#endif

// C++
#include <algorithm>
#include <bitset>
#include <complex>
#include <deque>
#include <exception>
#include <fstream>
#include <functional>
#include <iomanip>
#include <ios>
#include <iosfwd>
#include <iostream>
#include <istream>
#include <iterator>
#include <limits>
#include <list>
#include <locale>
#include <map>
#include <memory>
#include <new>
#include <numeric>
#include <ostream>
#include <queue>
#include <set>
#include <sstream>
#include <stack>
#include <stdexcept>
#include <streambuf>
#include <string>
#include <typeinfo>
#include <utility>
#include <valarray>
#include <vector>

#if __cplusplus >= 201103L
#include <array>
#include <atomic>
#include <chrono>
#include <codecvt>
#include <condition_variable>
#include <forward_list>
#include <future>
#include <initializer_list>
#include <mutex>
#include <random>
#include <ratio>
#include <regex>
#include <scoped_allocator>
#include <system_error>
#include <thread>
#include <tuple>
#include <typeindex>
#include <type_traits>
#include <unordered_map>
#include <unordered_set>
#endif
#if __cplusplus >= 201402L
#include <shared_mutex>
#endif\end{lstlisting}
\clearpage\section{字串符}
\subsection{KMP}
\begin{lstlisting}
template<class elemType>
inline void kmp_nxt(elemType &T, vector<int> &nxt) {
    nxt[0] = -1;
    for (int i = 1; i < T.size(); i++) {
        int j = nxt[i - 1];
        while (j >= 0 && T[i - 1] != T[j]) j = nxt[j];
        if (j >= 0 && T[i - 1] == T[j]) nxt[i] = j + 1;
        else nxt[i] = 0;
    }
}

template<class elemType>
inline int kmp_count(elemType &S, elemType &T) {
    vector<int> nxt(T.size());
    kmp_nxt(T, nxt);
    int index, count = 0;
    for (index = 0; index < S.size(); ++index) {
        int pos = 0;
        int iter = index;
        while (pos < T.size() && iter < S.size()) {
            if (S[iter] == T[pos]) {
                ++iter;
                ++pos;
            } else {
                if (pos == 0) ++iter;
                else pos = nxt[pos - 1] + 1;
            }
        }
        if (pos == T.size() && (iter - index) == T.size()) ++count;
    }
    return count;
}

template<class elemType>
inline void kmp_next(elemType T[], int count, vector<int> &nxt) {
    nxt[0] = -1;
    for (int i = 1; i < count; i++) {
        int j = nxt[i - 1];
        while (j >= 0 && T[i - 1] != T[j]) j = nxt[j];
        if (j >= 0 && T[i - 1] == T[j]) nxt[i] = j + 1;
        else nxt[i] = 0;
    }
}

template<class elemType>
inline int kmp_count(elemType S[], int c1, elemType T[], int c2) {
    vector<int> nxt(c2);
    kmp_nxt(T, c2, nxt);
    int index, count = 0;
    for (index = 0; index < c1; ++index) {
        int pos = 0;
        int iter = index;
        while (pos < c2 && iter < c1) {
            if (S[iter] == T[pos]) {
                ++iter;
                ++pos;
            }
            else {
                if (pos == 0) ++iter;
                else pos = nxt[pos - 1] + 1;
            }
        }
        if (pos == c2 && (iter - index) == c2) ++count;
    }
    return count;
}\end{lstlisting}
\clearpage\section{动态规划}
\subsection{01Bag}
\begin{lstlisting}
void dp(int n, int m) {
    // n=物品个数
    for (int i = 0; i < n; i++) {
        // m=背包最大容量
        for (int j = m; j >= wei[i]; j--)
            // wei=大小 val=价值
            f[j] = max(f[j], f[j - wei[i]] + val[i]);
    }
}
\end{lstlisting}
\subsection{BagProblem}
\begin{lstlisting}
#define N 1000
// val=价值 wei=重量 num=数量
int val[N], wei[N], num[N], f[N];
// n=种类个数 m=背包最大值

// 01背包
void dp1(int n, int m) {
    for (int i = 0; i < n; i++) {
        for (int j = m; j >= wei[i]; j--)
            f[j] = max(f[j], f[j - wei[i]] + val[i]);
    }
}

// 完全背包
void dp2(int n, int m) {
    //初始化看要求
    for (int i = 0; i <= m; i++) {
        f[i] = INF;
    }
    f[0] = 0;
    //若要求恰好装满背包,那在初始化时除了f[0]=0其它f[1..V]均=-∞
    //若没要求背包装满,只希望价格大,初始化时应将f[0..V]=0)
    for (int i = 0; i < n; i++)
        for (int j = wei[i]; j <= m; j++)
            f[j] = max(f[j], f[j - wei[i]] + val[i]);
}

// 多重背包
void dp3(int n, int m) {
    for (int i = 0; i < n; i++)
        for (int k = 0; k < num[i]; k++)
            for (int j = m; j >= wei[i]; j--)
                f[j] = max(f[j], f[j - wei[i]] + val[i]);
}
\end{lstlisting}
\subsection{FullBag}
\begin{lstlisting}
/* 
完全背包问题的特点是,每种物品可以无限制的重复使用,可以选择放或不放。 
完全背包问题描述: 
有N物品和一个容量为V的背包。第i件物品的重量是wei[i],价值是val[i]。 
*/  
  
#include <cstdio>  
#define INF 0x3fffffff  
#define N 10047  
int f[N],val[N],wei[N];  
int min(int a,int b)  
{  
    return x<y?x:y;  
}  
int main()  
{  
    int t,i,j,k,E,F,m,n;  
    scanf("%d",&t);  
    while(t--)  
    {  
        scanf("%d%d",&E,&F);  
        int c = F-E;  
        for(i = 0 ; i <= c ; i++)  
            f[i]=INF;  
        scanf("%d",&n);  
        for(i = 0 ; i < n ; i++)  
        {  
            scanf("%d%d",&val[i],&wei[i]);//val[i]为面额,wei[i]为重量  
        }  
        f[0]=0;//因为此处假设的是小猪储钱罐 恰好装满 的情况  
        //注意初始化(要求恰好装满背包,那么在初始化时除了f[0]为0其它f[1..V]均设为-∞,  
        //这样就可以保证最终得到的f[N]是一种恰好装满背包的最优解。  
        //如果并没有要求必须把背包装满,而是只希望价格尽量大,初始化时应该将f[0..V]全部设为0)  
        for(i =0 ; i < n ; i++)  
        {  
            for(j = wei[i] ; j <= c ; j++)  
            {  
                f[j] = min(f[j],f[j-wei[i]]+val[i]);//此处求的是最坏的情况所以用min,确定最少的钱,当然最后就用max了,HEHE  
            }  
        }  
        if(f[c] == INF)  
            printf("This is impossible.\n");  
        else  
            printf("The minimum amount of money in the piggy-bank is %d.\n",f[c]);  
    }  
    return 0;  
}  
//此代码为HDU1114;  \end{lstlisting}
\subsection{MultiBag}
\begin{lstlisting}
//多重背包(MultiplePack): 有N种物品和一个容量为V的背包。  
//第i种物品最多有n[i]件可用,每件费用是c[i],价值是w[i]。  
//求解将哪些物品装入背包可使这些物品的费用总和不超过背包容量,  
//且价值总和最大。  
//HDU 2191  
  
#include <cstdio>  
#include <cstring>  
#define N 247  
int max(int x,int y){
    return x>y?x:y;  
}  
int main()  {  
    int t,n,m,i,j,k;  
    int w[N],pri[N],num[N],f[N];  
    while(~scanf("%d",&t)){  
        while(t--){  
            memset(f,0,sizeof(f));  
            scanf("%d%d",&n,&m);//n为总金额,m为大米种类  
            for(i = 0 ; i < m ; i++){  
                scanf("%d%d%d",&pri[i],&w[i],&num[i]);//num[i]为每种大米的袋数  
            }  
            for(i = 0 ; i < m ; i++){  
                for(k = 0 ; k < num[i] ; k++){  
                    for(j = n ; j >= pri[i]; j--){  
                        f[j] = max(f[j],f[j-pri[i]]+w[i]);  
                    }  
                }  
            }  
            printf("%d\n",f[n]);  
        }  
    }  
    return 0;  
}  \end{lstlisting}
\subsection{Maze01}
\begin{lstlisting}
struct Maze01 {
    // 原始01矩阵 1-n 1-m
    bool a[N][N];
    // 以(i,j)向上最大高度的最大子矩阵的左右高
    int l[N][N], r[N][N], h[N][N];
    // 最大连续前缀1的左端位置, 如果(i,j)为0则ml=j+1
    int ml[N][N];
    // 矩阵大小
    int n, m;

    void prework() {
        for (int i = 1; i <= m; i++) {
            l[0][i] = 0;
            r[0][i] = m;
        }
        for (int i = 1; i <= n; i++) {
            int maxl = 1;
            int maxr = m;
            for (int j = 1; j <= m; j++) {
                if (a[i][j] == 0) {
                    maxl = j + 1;
                    h[i][j] = l[i][j] = 0;
                } else {
                    h[i][j] = h[i - 1][j] + 1;
                    l[i][j] = max(maxl, l[i - 1][j]);
                }
                ml[i - 1][j] = maxl;
                ml[i][j] = j + 1;
            }
            for (int j = m; j >= 1; --j) {
                if (a[i][j] == 0) {
                    maxr = j - 1;
                    r[i][j] = m;
                } else {
                    r[i][j] = min(maxr, r[i - 1][j]);
                }
            }
        }
    }

    // 单调栈
    pii st[N];

    int calc() {
        prework();
        int ans = 0;
        for (int i = 1; i <= n; i++) {
            int top = 0;
            for (int j = 1; j <= m; j++) {
                if (h[i][j] == 0) continue;
                if (top == 0 || st[top] != make_pair(l[i][j], r[i][j])) {
                    st[++top] = make_pair(l[i][j], r[i][j]);
                }
                while (top && st[top].second == j) {
                    int pos = st[top--].first;
                    if (pos < ml[i][j]) {
                        // l[i][j]-r[i][j]为底, h[i][j]为高
                        // 可以得到所有的唯一子矩阵, 不存在maze1完全属于maze2
                        ans++;
                    }
                }
            }
        }
        return ans;
    }
} dp;\end{lstlisting}
\clearpage\section{数据结构}
\subsection{BTree}
\begin{lstlisting}
template<class T>

struct TreeNode {
    T value;
    TreeNode *left;
    TreeNode *right;
};

template<class T>
TreeNode<T> *createTree(const T *pre, const T *in, const int len) {
    TreeNode<T> *t = NULL;
    if (len > 0) {
        t = new TreeNode<T>;
        t->value = pre[0];
        int index;
        for (index = 0; index < len; index++) {
            if (in[index] == pre[0]) {
                break;
            }
        }
        if (index == len) {
            index = -1;
        }
        t->left = createTree(pre + 1, in, index);
        t->right = createTree(pre + index + 1, in + index + 1, len - index - 1);
    }
    return t;
}

template<class T>
int preOrder(TreeNode<T> *root, queue<T> &out) {
    if (root) {
        int count = 1;
        out.push(root->value);
        count += preOrder(root->left, out);
        count += preOrder(root->right, out);
        return count;
    } else {
        return 0;
    }
}

template<class T>
int inOrder(TreeNode<T> *root, queue<T> &out) {
    if (root) {
        int count = 1;
        count += inOrder(root->left, out);
        out.push(root->value);
        count += inOrder(root->right, out);
        return count;
    } else {
        return 0;
    }
}

template<class T>
void postOrder(TreeNode<T> *root, queue<T> &out) {
    if (root) {
        postOrder(root->left, out);
        postOrder(root->right, out);
        out.push(root->value);
    } else {
        return;
    }
}

template<class T>
T *convertQueueToArray(queue<T> &out, int len) {
    T *list = new T[len];
    int now = 0;
    while (!out.empty() && now < len) {
        list[now] = out.front();
        out.pop();
        now++;
    }
    return list;
}

template<class T>
void destroyTree(TreeNode<T> *root) {
    if (root) {
        destroyTree(root->left);
        destroyTree(root->right);
        delete root;
    } else return;
}

template<class T>
void insertIntoBSTree(TreeNode<T> *root, const T &value) {
    if (!root) {
        return;
    }
    if (value < root->value) {
        if (root->left) {
            insertIntoTree(root->left, value);
        } else {
            root->left = new TreeNode<T>;
            root->left->value = value;
            root->left->left = NULL;
            root->left->right = NULL;
        }
    } else if (value > root->value) {
        if (root->right) {
            insertIntoTree(root->right, value);
        } else {
            root->right = new TreeNode<T>;
            root->right->value = value;
            root->right->left = NULL;
            root->right->right = NULL;
        }
    }
}

template<class T>
TreeNode<T> *createBSTree(T *list, int len) {
    if (len < 1) {
        return NULL;
    }
    TreeNode<T> *root = new TreeNode<char>;
    root->value = list[0];
    root->left = NULL;
    root->right = NULL;
    for (int i = 1; i < len; i++) {
        insertIntoBSTree(root, list[i]);
    }
    return root;
}\end{lstlisting}
\subsection{pbds-bbtree}
\begin{lstlisting}
// RBTree 红黑树
#include <ext/pb_ds/tree_policy.hpp>
#include <ext/pb_ds/assoc_container.hpp>
// 红黑树
__gnu_pbds::tree<int, null_type, less<int>, rb_tree_tag, tree_order_statistics_node_update> t;
// null_type无映射(低版本g++为null_mapped_type)
// 类似multiset
__gnu_pbds::tree<int, null_type, less_equal<int>, rb_tree_tag, tree_order_statistics_node_update> t;
find_by_order(size_t order);
// 结点更新
tree_order_statistics_node_update
insert(p);
erase(it);
// 求k在树中是第几大:
order_of_key(p);
// 找到第order小的迭代器
find_by_order(order);
// 前驱
lower_bound(p);
// 后驱
upper_bound(p);
// 合并
a.join(b);
// 分割 key小于等于v的元素属于a,其余的属于b
a.split(v, b);

// 优先队列
#include <ext/pb_ds/priority_queue.hpp>
#include <ext/pb_ds/assoc_container.hpp>
// 操作类似于stl的优先队列
typedef __gnu_pbds::priority_queue<node, greater<node>, __gnu_pbds::thin_heap_tag> heap;
heap::point_iterator; // 指向元素的指针\end{lstlisting}
\subsection{树状数组}
\begin{lstlisting}
const int N = 1000005;
struct BITree {
    int n;
    ll c[N];

    void init(int _n) {
        n = _n;
        memset(c, 0, sizeof(ll) * ++n);
    }

    void change(int pos, ll v) {
        for (int i = pos; i < n; i += i & (-i))
            c[i] += v;
    }

    ll query(int x) {
        ll ans = 0;
        for (int i = x; i > 0; i -= i & (-i))
            ans += c[i];
        return ans;
    }

    void update(int l, int r, ll v) {
        change(l, v);
        change(r + 1, -v);
    }
};\end{lstlisting}
\subsection{二维树状数组}
\begin{lstlisting}
const int N = 2005;

inline int lowbit(const int &x) {
        return x & -x;
    }

struct TdBITree {
    int n, m;
    ll c[N][N];

    void init(int n, int m) {
        this->n = n;
        this->m = m;
        memset(c, 0, sizeof(c))
    }

    void init(int n, int m, ll v) {
        this->n = n;
        this->m = m;
        rep(x, 1, N) {
            rep(y, 1, N) {
                c[x][y] = (x * y + (x - lowbit(x)) * (y - lowbit(y)) - x * (y - lowbit(y)) - (x - lowbit(x)) * y) * v;
            }
        }
    }

    void change(int x, int y, ll v) {
        for (int i = x; i <= n; i += lowbit(i))
            for (int j = y; j <= m; j += lowbit(j))
                c[i][j] += v;
    }

    ll query(int x, int y) {
        ll ans = 0;
        for (int i = x; i >= 1; i -= lowbit(i))
            for (int j = y; j >= 1; j -= lowbit(j))
                ans += c[i][j];
        return ans;
    }

    ll solve(int x1, int y1, int x2, int y2) {
        return query(x2, y2) - query(x1 - 1, y2) - query(x2, y1 - 1) + query(x1 - 1, y1 - 1);
    }
};\end{lstlisting}
\subsection{线段树}
\begin{lstlisting}
// hdu 6562
struct TreeNode {
    int l, r;
    int lson, rson;
    ll sum;
    ll len;
    ll laz1, laz2, laz3;

    inline void init(int a, int b, int ls, int rs) {
        lson = ls;
        rson = rs;
        l = a, r = b;
        sum = 0;
        len = 1;
        laz1 = laz2 = 0;
        laz3 = 1;
    }

    inline int mid() {
        return (l + r) >> 1;
    }

    inline int width() {
        return r - l + 1;
    }

    inline void add(ll val) {
        sum = (sum * 10 + val * len * 10 + val * width()) % MOD;
        len = (len * 100) % MOD;
        // 右懒惰
        laz1 = (laz1 * 10 + val) % MOD;
        // 左懒惰
        laz2 = (laz2 + val * laz3) % MOD;
        // 长度懒惰
        laz3 = (laz3 * 10) % MOD;
    }
};

struct SegTree {
    int tot;
    TreeNode node[N << 1];

    inline void init() {
        tot = 0;
    }

    inline void up(int k) {
        TreeNode &nd = node[k];
        nd.sum = (node[nd.lson].sum + node[nd.rson].sum) % MOD;
        nd.len = (node[nd.lson].len + node[nd.rson].len) % MOD;
    }

    inline void push(int k) {
        TreeNode &nd = node[k];
        if (nd.laz3 == 1) return;
        TreeNode &lson = node[nd.lson];
        TreeNode &rson = node[nd.rson];

        lson.sum = (nd.laz1 * lson.width() + lson.sum * nd.laz3 + nd.laz2 * lson.len % MOD * nd.laz3) % MOD;
        lson.len = (lson.len * nd.laz3 % MOD * nd.laz3) % MOD;
        lson.laz1 = (lson.laz1 * nd.laz3 + nd.laz1) % MOD;
        lson.laz2 = (nd.laz2 * lson.laz3 + lson.laz2) % MOD;
        lson.laz3 = (lson.laz3 * nd.laz3) % MOD;

        rson.sum = (nd.laz1 * rson.width() + rson.sum * nd.laz3 + nd.laz2 * rson.len % MOD * nd.laz3) % MOD;
        rson.len = (rson.len * nd.laz3 % MOD * nd.laz3) % MOD;
        rson.laz1 = (rson.laz1 * nd.laz3 + nd.laz1) % MOD;
        rson.laz2 = (nd.laz2 * rson.laz3 + rson.laz2) % MOD;
        rson.laz3 = (rson.laz3 * nd.laz3) % MOD;

        nd.laz1 = nd.laz2 = 0;
        nd.laz3 = 1;
    }

    void build(int k, int l, int r) {
        TreeNode &nd = node[k];
        nd.init(l, r, tot + 1, tot + 2);
        tot += 2;
        if (l == r) {
            return;
        }
        int mid = nd.mid();
        build(nd.lson, nd.l, mid);
        build(nd.rson, mid + 1, nd.r);
        up(k);
    }

    void change(int k, int l, int r, ll val) {
        TreeNode &nd = node[k];
        if (nd.l == l && nd.r == r) {
            nd.add(val);
            return;
        }
        push(k);
        int mid = nd.mid();
        if (r <= mid) {
            change(nd.lson, l, r, val);
        } else if (l > mid) {
            change(nd.rson, l, r, val);
        } else {
            change(nd.lson, l, mid, val);
            change(nd.rson, mid + 1, r, val);
        }
        up(k);
    }

    ll query(int k, int l, int r) {
        TreeNode &nd = node[k];
        if (nd.l == l && nd.r == r) {
            return nd.sum;
        }
        push(k);
        int mid = nd.mid();
        ll ans = 0;
        if (r <= mid) {
            ans += query(nd.lson, l, r);
        } else if (l > mid) {
            ans += query(nd.rson, l, r);
        } else {
            ans += query(nd.lson, l, mid);
            ans += query(nd.rson, mid + 1, r);
        }
        return ans % MOD;
    }
} tree;\end{lstlisting}
\subsection{二维线段树}
\begin{lstlisting}

const int N = 1005;

struct SegTree {

    inline int son(int k, int x) {
        return (k << 2) - 2 + x;
    }

    struct node {
        int l, r;

        node() = default;

        node(int a, int b) : l(a), r(b) {}

        inline int mid() {
            return (l + r) >> 1;
        }

        inline node left() {
            return node(l, mid());
        }

        inline node right() {
            return node(mid() + 1, r);
        }

        inline bool in(int x) {
            return x >= l && x <= r;
        }

        inline bool more() {
            return l < r;
        }

        bool operator==(const node &t) {
            return l == t.l && r == t.r;
        }
    };

    ll c[N << 2][N << 2];
    ll ans[N << 4];
    ll laz[N << 4];

    inline void up(int k, bool x, bool y) {
        int s = (k << 2) - 2;
        ll t = 0;
        if (x) t += ans[s] + ans[s + 1] + laz[s] + laz[s + 1];
        if (y) t += ans[s + 2] + ans[s + 3] + laz[s + 2] + laz[s + 3];
        ans[k] = t;
    }

    inline void push(int k) {
        int s = (k << 2) - 2;
        laz[s] += laz[k];
        laz[s + 1] += laz[k];
        laz[s + 2] += laz[k];
        laz[s + 3] += laz[k];
        ans[k] += laz[k];
        laz[k] = 0;
    }

    void build(node x, node y, int k) {
        laz[k] = 0;
        if (x.more() && y.more()) {
            ans[k] = c[x.l][y.l];
            return;
        }
        ans[k] = 0;
        bool ax = false;
        bool ay = false;
        if (x.more()) {
            build(x.left(), y, son(k, 0));
            build(x.right(), y, son(k, 1));
        }
        if (y.more()) {
            build(x, y.left(), son(k, 2));
            build(x, y.right(), son(k, 3));
        }
        up(k, x.more(), y.more());
    }

    void change(node x, node y, int k, node l, node r, ll v) {
        if (x == l && y == r) {
            laz[k] += v;
            return;
        }
        push(k);
        if (x.more()) {
            if (l.r <= x.mid()) {
                change(x.left(), y, son(k, 0), l, r, v);
            } else if (l.l > x.mid()) {
                change(x.right(), y, son(k, 1), l, r, v);
            } else {
                change(x.left(), y, son(k, 0), node(l.l, x.mid()), r, v);
                change(x.right(), y, son(k, 1), node(x.mid() + 1, l.r), r, v);
            }
        }
        if (y.more()) {
            if (r.l <= y.mid()) {
                change(x, y.left(), son(k, 2), l, r, v);
            } else if (r.r > y.mid()) {
                change(x, y.right(), son(k, 3), l, r, v);
            } else {
                change(x, y.left(), son(k, 2), l, node(r.l, y.mid()), v);
                change(x, y.right(), son(k, 3), l, node(y.mid() + 1, r.r), v);
            }
        }
        up(k, x.more(), y.more());
    }

    ll query(node x, node y, int k, node l, node r) {
        if (x == l && y == r) {
            return ans[k] + laz[k];
        }
        push(k);
        ll t = 0;
        if (x.more()) {
            if (l.r <= x.mid()) {
                t += query(x.left(), y, son(k, 0), l, r);
            } else if (l.l > x.mid()) {
                t += query(x.right(), y, son(k, 1), l, r);
            } else {
                t += query(x.left(), y, son(k, 0), node(l.l, x.mid()), r);
                t += query(x.right(), y, son(k, 1), node(x.mid() + 1, l.r), r);
            }
        }
        if (y.more()) {
            if (r.l <= y.mid()) {
                t += query(x, y.left(), son(k, 2), l, r);
            } else if (r.r > y.mid()) {
                t += query(x, y.right(), son(k, 3), l, r);
            } else {
                t += query(x, y.left(), son(k, 2), l, node(r.l, y.mid()));
                t += query(x, y.right(), son(k, 3), l, node(y.mid() + 1, r.r));
            }
        }
        return t;
    }
};\end{lstlisting}
\subsection{树状数组求逆序对}
\begin{lstlisting}
BITree t;
int n;
pii a[N];

void solve() {
    t.init(n);
    for (int i = 1; i <= n; i++) {
        int x;
        cin >> x;
        a[i] = make_pair(x, i);
    }
    sort(a + 1, a + n + 1);
    ll ans = 0;
    for (int i = 1; i <= n; i++) {
        t.change(a[i].second, 1);
        ans += (i - t.query(a[i].second));
    }
    cout << ans << endl;
}

\end{lstlisting}
\subsection{ST}
\begin{lstlisting}
// 只需要取值
struct ST {
    int ck[N];
    int dp[20][N];

    void init(int n, int squ[]) {
        ++n;
        ck[0] = -1;
        for (int i = 1; i <= n; i++) {
            ck[i] = ck[i - 1] + ((i & (i - 1)) == 0 ? 1 : 0);
        }
        for (int i = 0; i < n; i++) {
            dp[0][i] = squ[i];
        }
        for (int k = 1; k <= ck[n]; k++) {
            int dk = k - 1;
            for (int i = 0; i + (1 << k) - 1 < n; i++) {
                dp[k][i] = max(dp[dk][i], dp[dk][i + (1 << dk)]);
            }
        }
    }

    int query(int l, int r) {
        if (l > r) swap(l, r);
        int k = ck[r - l + 1];
        return max(dp[k][l], dp[k][r - (1 << k) + 1]);
    }
};

// 可得到下标
struct ST {
    int ck[N];
    int rmq[N];
    int dp[20][N];

    void init(int n, int squ[]) {
        ++n;
        ck[0] = -1;
        for (int i = 1; i <= n; i++) {
            ck[i] = ck[i - 1] + ((i & (i - 1)) == 0 ? 1 : 0);
        }
        memcpy(rmq, squ, sizeof(int) * n);
        for (int i = 0; i < n; i++) {
            dp[0][i] = i;
        }
        for (int k = 1; k <= ck[n]; k++) {
            int dk = k - 1;
            for (int i = 0; i + (1 << k) - 1 < n; i++) {
                int a = dp[dk][i];
                int b = dp[dk][i + (1 << dk)];
                dp[k][i] = rmq[a] < rmq[b] ? a : b;
            }
        }
    }

    int query(int l, int r) {
        if (l > r) swap(l, r);
        int k = ck[r - l + 1];
        int a = dp[k][l];
        int b = dp[k][r - (1 << k) + 1];
        return rmq[a] < rmq[b] ? a : b;
    }
};\end{lstlisting}
\subsection{笛卡尔树}
\begin{lstlisting}
// 笛卡尔树,静态建树,区间最值跳转
struct CartesianTree {
    int rt; // 根节点
    pii ch[N]; // 左右儿子
    int st[N]; // 单调栈

    void build(int n, int p[]) {
        rt = 0;
        int t = 0;
        for (int i = 1; i <= n; i++) {
            // 决定了大于还是小于
            while (t && p[st[t]] > p[i]) --t;
            if (t) { 
				// 成为上一个点的右儿子
				// 前一个点成为自己的左子节点
                ch[st[t]].second = i;
                ch[i].first = i - 1;
            } else { // 自己作为根节点
                ch[i].first = rt;
                rt = i;
            }
            st[++t] = i;
        }
    }
} dika;
\end{lstlisting}
\subsection{DancingLinks}
\begin{lstlisting}
// Dancing Links
struct DLX {
    int n, m, size;
    int U[MaxNode], D[MaxNode], L[MaxNode], R[MaxNode], Row[MaxNode], Col[MaxNode];
    int H[MaxN], S[MaxM];
    int ansd, ans[MaxN];

    void init(int _n, int _m) {
        n = _n;
        m = _m;
        for (int i = 0; i <= m; i++) {
            S[i] = 0;
            U[i] = D[i] = i;
            L[i] = i - 1;
            R[i] = i + 1;
        }
        R[m] = 0;
        L[0] = m;
        size = m;
        for (int i = 0; i <= n; i++) {
            H[i] = -1;
        }
    }

    void Link(int r, int c) {
        ++S[Col[++size] = c];
        Row[size] = r;
        D[size] = D[c];
        U[D[c]] = size;
        U[size] = c;
        D[c] = size;
        if (H[r] < 0) {
            H[r] = L[size] = R[size] = size;
        } else {
            R[size] = R[H[r]];
            L[R[H[r]]] = size;
            L[size] = H[r];
            R[H[r]] = size;
        }
    }

    void remove(int c) {
        L[R[c]] = L[c];
        R[L[c]] = R[c];
        for (int i = D[c]; i != c; i = D[i]) {
            for (int j = R[i]; j != i; j = R[j]) {
                U[D[j]] = U[j];
                D[U[j]] = D[j];
                --S[Col[j]];
            }
        }
    };

    void resume(int c) {
        for (int i = U[c]; i != c; i = U[i])
            for (int j = L[i]; j != i; j = L[j])
                ++S[Col[U[D[j]] = D[U[j]] = j]];
        L[R[c]] = R[L[c]] = c;
    }

    bool Dance(int d) {
        if (R[0] == 0) {
            for (int i = 0; i < d; i++) {
                printf("%d%c", ans[i], " \n"[i == d - 1]);
            }
            return true;
        }
        int c = R[0];
        for (int i = R[0]; i != 0; i = R[i]) if (S[i] < S[c]) c = i;
        remove(c);
        for (int i = D[c]; i != c; i = D[i]) {
            ans[d] = Row[i];
            for (int j = R[i]; j != i; j = R[j])remove(Col[j]);
            if (Dance(d + 1))return true;
            for (int j = L[i]; j != i; j = L[j])resume(Col[j]);
        }
        resume(c);
        return false;
    }
};\end{lstlisting}
\subsection{主席树}
\begin{lstlisting}
struct SegTree {
    int num[N];
    int c[M];
    ll s[M];
    pii ran;
    int tot = 0;
    int lson[M], rson[M];

    int build(int l, int r) {
        int k = ++tot;
        s[k] = c[k] = 0;
        if (l == r) {
            return k;
        }
        int mid = (l + r) >> 1;
        lson[k] = build(l, mid);
        rson[k] = build(mid + 1, r);
        return k;
    }

    int init(int l, int r) {
        ran = pii(l, r);
        tot = 0;
        return build(l, r);
    }

    int update(int rt, int p) {
        int val = num[p];
        int l, r;
        tie(l, r) = ran;
        int k = ++tot;
        int tmp = k;
        c[k] = c[rt] + 1, s[k] = s[rt] + val;
        while (l < r) {
            int mid = (l + r) >> 1;
            if (p <= mid) {
                lson[k] = ++tot;
                rson[k] = rson[rt];
                k = lson[k], rt = lson[rt];
                r = mid;
            } else {
                lson[k] = lson[rt];
                rson[k] = ++tot;
                k = rson[k], rt = rson[rt];
                l = mid + 1;
            }
            c[k] = c[rt] + 1, s[k] = s[rt] + val;
        }
        return tmp;
    }

    pll query(int a, int b, int h) {
        int l, r;
        tie(l, r) = ran;
        int cs = c[a] - c[b];
        int cnt = 0;
        ll sum = 0;
        while (l < r) {
            int mid = (l + r) >> 1;
            if (h <= mid) {
                a = lson[a], b = lson[b];
                r = mid;
            } else {
                cnt += c[lson[a]] - c[lson[b]];
                sum += s[lson[a]] - s[lson[b]];
                a = rson[a], b = rson[b];
                l = mid + 1;
            }
        }
        cs -= cnt;
        sum += cs * num[l];
        return pll(sum, cs);
    }
} tree;\end{lstlisting}
\clearpage\section{图论}
\subsection{Graph}
\begin{lstlisting}
#define forg(i, h, eg) for(int i = (h); ~i; i = (eg[i]).nxt)

struct Edge {
    int e, nxt;
    ll v;
    Edge() = default;
    Edge(int a, ll b, int c = 0) : e(a), v(b), nxt(c) {}

    bool operator<(const Edge &a) const {
        return (a.v == v ? e < a.e : v < a.v);
    }
};

const ll INF = ll(1e11);
const int N = int(1e5 + 10);
const int M = int(3e5 + 10);

struct Graph {
    Edge eg[M];
    int head[N];
    int cnt;

    void init(int n) {
        memset(head, -1, sizeof(int) * ++n);
        cnt = 0;
    }

    inline void addEdge(int x, int y, ll v = 0) {
        eg[cnt] = Edge(y, v, head[x]);
        head[x] = cnt++;
    }
} gh;\end{lstlisting}
\subsection{Dijkstra}
\begin{lstlisting}
int dist[N];
int path[N];

void bfs(int s, int n) {
    n++;
    rep(i, 0, n) dist[i] = INF;
    memset(path, -1, sizeof(int) * n);
    dist[s] = 0;
    path[s] = s;
	// 注意优先队列默认less运算,但选择最大的作为top,注意cmp!!!
    priority_queue<Edge, vector<Edge>, greater<Edge>> q;
    q.push(Edge(s, dist[s]));
    while (!q.empty()) {
        Edge f = q.top();
        q.pop();
        for (int i = gh.head[f.e]; ~i; i = gh.eg[i].nxt) {
            Edge &t = gh.eg[i];
            if (dist[t.e] > f.v + t.v) {
                dist[t.e] = f.v + t.v;
                path[t.e] = f.e;
                q.push(Edge(t.e, dist[t.e]));
            }
        }
    }
}

#include <ext/pb_ds/priority_queue.hpp>
#include <ext/pb_ds/assoc_container.hpp>
typedef __gnu_pbds::priority_queue<Edge, greater<Edge>> heap;
// 使用该模板,需要注意因为使用了greater,所以需要重载大于运算
// 默认pairing_heap_tag
// push O(1), pop O(logn) modify O(logn) erase O(logn) join O(1)
// 可选thin_heap_tag
// push O(1), pop O(logn) modify O(1) erase O(logn) join O(n)

heap::point_iterator its[N];
int cnt[N];

void bfs(int s, int n) {
    n++;
    rep(i, 0, n) dist[i] = INF;
    memset(cnt, 0, sizeof(int) * n);
    dist[s] = 0;
    cnt[s] = 1;
    heap q;
    its[s] = q.push(Edge(s, dist[s]));
    while (!q.empty()) {
        Edge f = q.top();
        q.pop();
        for (int i = gh.head[f.e]; ~i; i = gh.eg[i].nxt) {
            Edge &t = gh.eg[i];
            its[t.e] = 0;
            int v = f.v + t.v;
            if (dist[t.e] > v) {
                dist[t.e] = v;
                if (its[t.e] != 0) {
                    q.modify(its[t.e], Edge(t.e, dist[t.e]));
                } else {
                    its[t.e] = q.push(Edge(t.e, dist[t.e]));
                }
                cnt[t.e] = cnt[f.e];
            } else if (dist[t.e] == v) {
                (cnt[t.e] += cnt[f.e]) %= 100003;
            }
        }
    }
}\end{lstlisting}
\subsection{spfa}
\begin{lstlisting}
vector<int> dist;
vector<vector<node>> eg;
vector<int> path;

bool spfa(int n, int start) {
    dist.assign(n, INF);
    dist[start] = 0;
    deque<int> q;
    q.push_back(start);
    path.assign(n, -1);
    vector<int> cnt(n, 0);
    vector<bool> flag(n, false);
    cnt[start] = flag[start] = true;
    while (!q.empty()) {
        const int now = q.front();
        q.pop_front();
        flag[now] = false;
        for (auto i: eg[now]) {
            if (dist[i.x] > dist[now] + i.d) {
                dist[i.x] = dist[now] + i.d;
                path[i.x] = now;
                if (!flag[i.x]) {
                    if (n == ++cnt[i.x]) return false;
                    //队列非空且优于队首(SLF)
                    if (!q.empty() && dist[i.x] < dist[q.front()]) {
                        q.push_front(i.x);
                    } else {
                        q.push_back(i.x);
                    }
                    flag[i.x] = true;
                }
            }
        }
    }
    return true;
}
\end{lstlisting}
\subsection{Dinic}
\begin{lstlisting}
struct Dinic {
    Graph gh;
    // 点的范围[0, n)
    int n;
    // 弧优化
    int cur[N], dis[N];

    Dinic(){};

    // 设置N
    void init(int _n) {
        n = _n;
        gh.init(n);
    }

    // 加流量
    void addFlow(int x, int y, ll f) {
        gh.addEdge(x, y, f);
        gh.addEdge(y, x, 0);
    }

    bool bfs(int s, int e) {
        memset(dis, -1, sizeof(int) * n);
        int q[N];
        int l, r;
        l = r = 0;
        dis[s] = 0;
        q[r++] = s;
        while (l < r) {
            int f = q[l++];
            for (int i = gh.head[f]; ~i; i = gh.eg[i].nxt) {
                if (gh.eg[i].v > 0 && dis[gh.eg[i].e] == -1) {
                    dis[gh.eg[i].e] = dis[f] + 1;
                    q[r++] = gh.eg[i].e;
                }
            }
        }
        return dis[e] > 0;
    }

    ll dfs(int s, int e, ll mx) {
        if (s == e || mx == 0) {
            return mx;
        }
        ll flow = 0;
        for (int &k = cur[s]; ~k; k = gh.eg[k].nxt) {
            auto &eg = gh.eg[k];
            ll a;
            if (eg.v > 0 && dis[eg.e] == dis[s] + 1 && (a = dfs(eg.e, e, min(eg.v, mx)))) {
                eg.v -= a;
                gh.eg[k ^ 1].v += a;
                flow += a;
                mx -= a;
                if (mx <= 0) break;
            }
        }
        return flow;
    }

    ll max_flow(int s, int e) {
        ll ans = 0;
        while (bfs(s, e)) {
            memcpy(cur, gh.head, sizeof(int) * n);
            ans += dfs(s, e, INF);
        }
        return ans;
    }
} dinic;\end{lstlisting}
\subsection{hungry}
\begin{lstlisting}
#define N 105
#define M 10005
int n, m, k;
pii eg[M * 2];
int result[N * 2];
int head[N * 2];
int cnt = 0;

void addEdge(int x, int y) {
    eg[cnt].first = y;
    eg[cnt].second = head[x];
    head[x] = cnt++;
}

bool vis[M * 2] = {false};

int dfs(int x) {
    for (int i = head[x]; ~i; i = eg[i].second) {
        int y = eg[i].first;
        if (!vis[y]) {
            vis[y] = true;
            if (result[y] == -1 || dfs(result[y])) {
                result[y] = x;
                return 1;
            }
        }
    }
    return 0;
}

int MaxMatch() {
    int ans = 0;
    memset(result, -1, sizeof(result));
    rep(i, 1, n + 1) {
        memset(vis, 0, sizeof(vis));
        ans += dfs(i);
    }
    return ans;
}

void solve() {
    scanf("%d%d", &m, &k);
    memset(head, -1, sizeof(head));
    cnt = 0;
    rep(i, 0, k) {
        int x, y;
        scanf("%d%d", &x, &y);
        addEdge(x, y);
    }
    int ans = MaxMatch();
    printf("%d\n", ans);
}\end{lstlisting}
\subsection{MinSpanTree}
\begin{lstlisting}
/*
* Prim 求 MST
* 耗费矩阵 cost[][],标号从 0 开始,0∼n-1
* 返回最小生成树的权值,返回 -1 表示原图不连通
 */
const int INF = 0x3f3f3f3f;
const int N = 110;
bool vis[N];
int lowc[N]; //点是 0 n-1 
int prim(int cost[][N], int n) {
    int ans = 0;
    memset(vis, false, sizeof(vis));
    vis[0] = true;
    for (int i = 1; i < n; i++)lowc[i] = cost[0][i];
    for (int i = 1; i < n; i++) {
        int minc = INF;
        int p =−1;
        19
        for (int j = 0; j < n; j++)
            if (!vis[j] && minc > lowc[j]) {
                minc = lowc[j];
                p = j;
            }
        if (minc == INF)return −1;//原图不连通 
        ans += minc;
        vis[p] = true;
        for (int j = 0; j < n; j++)
            if (!vis[j] && lowc[j] > cost[p][j])
                lowc[j] = cost[p][j];
    }
    return ans;
}\end{lstlisting}
\subsection{MinCostMaxFlow}
\begin{lstlisting}
struct Edge {
    int e, nxt;
    ll flow, cost;

    Edge() {};

    Edge(int a, ll b, ll c, int d = 0) : e(a), flow(b), cost(c), nxt(d) {}
};

const ll INF = 1000000;
const int N = int(1e5 + 10);
const int M = int(1e5 + 10);

//前向星
struct Graph {
    Edge eg[M];
    int head[N];
    int cnt;

    void init(int n) {
        memset(head, -1, sizeof(int) * ++n);
        cnt = 0;
    }

    inline void addEdge(int x, int y, ll v, ll c) {
        eg[cnt] = Edge(y, v, c, head[x]);
        head[x] = cnt++;
    }
};

struct MinCostMaxFlow {
    Graph gh;
    // 点的范围[0, n)
    int n;

    // 设置N
    void init(int _n) {
        n = _n + 1;
        gh.init(n);
    }

    // 加流量,反向是负的花费
    void addFlow(int x, int y, ll f, ll c) {
        // printf("%d->%d: %lld\t%lld\n", x, y, f, c); fflush(stdout);
        gh.addEdge(x, y, f, c);
        gh.addEdge(y, x, 0, -c);
    }

    // 该pre存的是边
    int pre[N];
    int dis[N];
    bool vis[N];

    bool spfa(int s, int e) {
        queue<int> q;
        for (int i = 0; i < n; i++) {
            dis[i] = INF;
            vis[i] = false;
            pre[i] = -1;
        }
        dis[s] = 0;
        vis[s] = true;
        q.push(s);
        while (!q.empty()) {
            int u = q.front();
            q.pop();
            vis[u] = false;
            for (int i = gh.head[u]; ~i; i = gh.eg[i].nxt) {
                Edge &eg = gh.eg[i];
                if (eg.flow > 0 && dis[eg.e] > dis[u] + eg.cost) {
                    dis[eg.e] = dis[u] + eg.cost;
                    pre[eg.e] = i;
                    if (!vis[eg.e]) {
                        vis[eg.e] = true;
                        q.push(eg.e);
                    }
                }
            }
        }
        return pre[e] != -1;
    }

    pll cal(int s, int e) {
        ll flow = 0, cost = 0;
        while (spfa(s, e)) {
            ll f = INF;
            for (int i = pre[e]; ~i; i = pre[gh.eg[i ^ 1].e]) {
                f = min(f, gh.eg[i].flow);
            }
            for (int i = pre[e]; ~i; i = pre[gh.eg[i ^ 1].e]) {
                gh.eg[i].flow -= f;
                gh.eg[i ^ 1].flow += f;
                cost += gh.eg[i].cost;
            }
            flow += f;
        }
        return make_pair(flow, cost);
    }

} network;

// vector图存
struct MinCostMaxFlow {
    vector<Edge> g[N];
    // 点的范围[0, n)
    int n = 0;

    // 设置N
    void init(int _n) {
        rep(i, 0, n) {
            g[i].clear();
        }
        n = _n + 1;
    }

    // 加流量,反向是负的花费
    void addFlow(int x, int y, int f, int c) {
        g[x].push_back(Edge(y, f, c, g[y].size()));
        g[y].push_back(Edge(x, 0, -c, g[x].size() - 1));
    }

    // 该pre存的是(点,边)
    pii pre[N];
    int dis[N];
    bool vis[N];
    int h[N];

    int cnt = 0;

    bool bfs(int s, int e) {
        priority_queue<pii, vector<pii>, greater<pii>> q;
        for (int i = 0; i < n; i++) {
            dis[i] = INF;
            vis[i] = false;
            pre[i] = pii(-1, -1);
        }
        dis[s] = 0;
        q.push(pii(0, s));
        while (!q.empty()) {
            pii f = q.top();
            int u = f.second;
            q.pop();
            if (f.first != dis[u]) continue;
            for (int i = 0; i < sz(g[u]); i++) {
                auto &eg = g[u][i];
                if (eg.flow == 0) continue;
                int v = eg.e;
                int cost = eg.cost + dis[u] + h[u] - h[v];
                if (dis[v] > cost) {
                    cnt++;
                    dis[v] = cost;
                    pre[v] = pii(u, i);
                    q.push(pii(dis[v], v));
                }
            }
        }
        for (int i = 0; i < n; i++) {
            h[i] += dis[i];
        }
        return pre[e].second != -1;
    }

    pii cal(int s, int e, int limit) {
        int flow = 0, cost = 0;
        memset(h, 0, sizeof(int) * n);
        cnt = 0;
        while (limit) {
            if (!bfs(s, e)) break;
            int f = INF;
            for (int i = e; ~pre[i].second; i = pre[i].first) {
                f = min(f, g[pre[i].first][pre[i].second].flow);
            }
            for (int i = e; ~pre[i].second; i = pre[i].first) {
                g[pre[i].first][pre[i].second].flow -= f;
                g[i][g[pre[i].first][pre[i].second].nxt].flow += f;
            }
            cost += f * h[e];
            flow += f;
            limit -= f;
        }
        return make_pair(flow, cost);
    }

} network;  \end{lstlisting}
\subsection{ISAP}
\begin{lstlisting}
struct ISAP {
    Graph gh;
    // 点的范围[0, n)
    int n;
    // 弧优化
    int cur[N], dis[N];
    ISAP() {};
    // 设置N
    void init(int _n) {
        n = _n;
        gh.init(n);
    }

    // 加流量
    inline void addFlow(int x, int y, ll f) {
        gh.addEdge(x, y, f);
        gh.addEdge(y, x, 0);
    }

    int dep[N]; // 记录距离标号
    int gap[N]; // gap常数优化
    int q[N]; // 数组模拟队列

    void bfs(int s, int e) {
        memset(dep, -1, sizeof(int) * n);
        memset(gap, 0, sizeof(int) * n);
        gap[0] = 1;
        dep[e] = 0;
        int l = 0, r = 0;
        q[r++] = e;
        while (l < r) {
            int u = q[l++];
            for (int i = gh.head[u]; ~i; i = gh.eg[i].nxt) {
                int v = gh.eg[i].e;
                if (~dep[v]) continue;
                q[r++] = v;
                dep[v] = dep[u] + 1;
                gap[dep[v]]++;
            }
        }
    }

    ll st[N]; // 栈优化

    ll max_flow(int s, int e) {
        bfs(s, e);
        memcpy(cur, gh.head, sizeof(int) * n);
        int top = 0;
        int u = s;
        ll ans = 0;
        while (dep[s] < N) {
            if (u == e) {
                ll mf = INF;
                int sel = 0;
                for (int i = 0; i < top; i++) {
                    if (mf > gh.eg[st[i]].v) {
                        mf = gh.eg[st[i]].v;
                        sel = i;
                    }
                }

                for (int i = 0; i < top; i++) {
                    gh.eg[st[i]].v -= mf;
                    gh.eg[st[i] ^ 1].v += mf;
                }
                ans += mf;
                top = sel;
                u = gh.eg[st[top] ^ 1].e;
                continue;
            }
            bool flag = false;
            int v = 0;
            for (int i = cur[u]; ~i; i = gh.eg[i].nxt) {
                v = gh.eg[i].e;
                if (gh.eg[i].v > 0 && dep[v] + 1 == dep[u]) {
                    flag = true;
                    cur[u] = i;
                    break;
                }
            }
            if (flag) {
                st[top++] = cur[u];
                u = v;
                continue;
            }
            int mind = N;
            for (int i = gh.head[u]; ~i; i = gh.eg[i].nxt) {
                if (gh.eg[i].v > 0 && dep[gh.eg[i].e] < mind) {
                    mind = dep[gh.eg[i].e];
                    cur[u] = i;
                }
            }
            gap[dep[u]]--; // 当前层无法连通,降层
            if (!gap[dep[u]]) return ans; // 断层结束运算
            dep[u] = mind + 1; // 进入更高层
            gap[dep[u]]++;
            if (u != s)  u = gh.eg[st[--top] ^ 1].e;
        }
        return ans;
    }
} isap;\end{lstlisting}
\subsection{树链剖分}
\begin{lstlisting}
struct TreeChain {
    int top[N]; // 链条顶端点ID
    int fa[N]; // 父亲节点
    int son[N]; // 重儿子
    int deep[N]; // 深度
    int num[N]; // 儿子节点数(包括自己)


    int p[N]; // 在线段树中的ID,
    int fp[N]; // 线段树中ID对应的点
    int tot;

    void dfs(int u, int pre, int d) {
        num[u] = 1;
        deep[u] = d;
        fa[u] = pre;
        for (int i = gh.head[u]; ~i; i = gh.eg[i].nxt) {
            int v = gh.eg[i].e;
            if (v == pre) continue;
            dfs(v, u, d + 1);
            num[u] += num[v];
            if (son[u] == -1 || num[v] > num[son[u]]) {
                son[u] = v;
            }
        }
    }

    void getpos(int u, int sp) {
        top[u] = sp;
        p[u] = tot++;
        fp[p[u]] = u;
        if (son[u] == -1) return;
        getpos(son[u], sp);
        for (int i = gh.head[u]; ~i; i = gh.eg[i].nxt) {
            int v = gh.eg[i].e;
            if (v == son[u] || v == fa[u]) continue;
            getpos(v, v);
        }
    }

    void build(int start, int root, int n) {
        memset(son, -1, sizeof(int) * ++n);
        tot = start; // start是线段树中的ID起始数值
        dfs(root, 0, 0);
        getpos(root, root);
    }
} treec;

// 树状数组,如果无需在线查询可以使用差分树
BITree tree;

// 点权修改
void change1(int u, int v, ll val) {
    int f1 = treec.top[u];
    int f2 = treec.top[v];
    while (f1 != f2) {
        if (treec.deep[f1] < treec.deep[f2]) {
            swap(f1, f2);
            swap(u, v);
        }
        tree1.update(treec.p[f1], treec.p[u], val);
        u = treec.fa[f1];
        f1 = treec.top[u];
    }
    if (treec.deep[u] > treec.deep[v]) {
        swap(u, v);
    }
    tree1.update(treec.p[u], treec.p[v], val);
}

// 边权修改
void change2(int u, int v, ll val) {
    int f1 = treec.top[u];
    int f2 = treec.top[v];
    while (f1 != f2) {
        if (treec.deep[f1] < treec.deep[f2]) {
            swap(f1, f2);
            swap(u, v);
        }
        tree2.update(treec.p[f1], treec.p[u], val);
        u = treec.fa[f1];
        f1 = treec.top[u];
    }
    if (treec.deep[u] > treec.deep[v]) {
        swap(u, v);
    }
    tree2.update(treec.p[treec.son[u]], treec.p[v], val);
}\end{lstlisting}
\subsection{倍增LCA}
\begin{lstlisting}
const int MAX_DEP = 20;

// 倍增2^k的父亲
int fa[N][MAX_DEP];

// 倍增LCA
int lca(int u, int v) {
    if (dep[u] > dep[v]) {
        swap(u, v);
    }
    int hu = dep[u], hv = dep[v];
    int tu = u, tv = v;
    for (int det = hv - hu, i = 0; det; det >>= 1, i++) {
        if (det & 1)
            tv = fa[tv][i];
    }
    if (tu == tv) {
        return tu;
    }
    for (int i = MAX_DEP - 1; i >= 0; i--) {
        if (fa[tu][i] == fa[tv][i]) {
            continue;
        }
        tu = fa[tu][i];
        tv = fa[tv][i];
    }
    return fa[tu][0];
}

// 动态更新节点的父亲属性
void lineFa(int u, int v) {
    fa[u][0] = v;
    for (int i = 1; i < MAX_DEP; i++) {
        v = fa[u][i] = fa[v][i - 1];
    }
}\end{lstlisting}
\subsection{Tarjan}
\begin{lstlisting}
int dfn[N], low[N], st[N], belong[N], num[N];
bool inst[N];
int idx, top, scc;

void tarjan(int u) {
    dfn[u] = low[u] = ++idx;
    st[top++] = u;
    inst[u] = true;
    for (int i = gh.head[u]; i != -1; i = gh.eg[i].nxt) {
        int v = gh.eg[i].e;
        if (!dfn[v]) {
            tarjan(v);
            low[u] = min(low[u], low[v]);
        } else if (inst[v]) {
            low[u] = min(low[u], dfn[v]);
        }
    }
    int v;
    if (dfn[u] == low[u]) {
        scc++;
        do {
            v = st[--top];
            inst[v] = false;
            belong[v] = scc;
            num[scc]++;
        } while (u != v);
    }
}\end{lstlisting}
\subsection{支配树}
\begin{lstlisting}
const int MAX_DEP = 20;

// 注意0,1点的边界问题
struct DominatorTree {
    int deg[N]; // 入度
    int dep[N]; //
    int dfn[N];
    int st[N];
    int tot;

    // 拓扑序, 要保证root是入度为0
    void bfs(Graph &gh, int root) {
        queue<int> q;
        q.push(root);
        tot = 0;
        while (!q.empty()) {
            int u = q.front();
            q.pop();
            dfn[u] = ++tot;
            st[tot] = u;
            forg(i, gh.head[u], gh.eg) {
                int v = gh.eg[i].e;
                if ((--deg[v]) == 0) {
                    q.push(v);
                }
            }
        }
    }

    // 倍增2^k的父亲
    int fa[N][MAX_DEP];

    // 倍增LCA
    int lca(int u, int v) {
        if (dep[u] > dep[v]) {
            swap(u, v);
        }
        int hu = dep[u], hv = dep[v];
        int tu = u, tv = v;
        for (int det = hv - hu, i = 0; det; det >>= 1, i++) {
            if (det & 1)
                tv = fa[tv][i];
        }
        if (tu == tv) {
            return tu;
        }
        for (int i = MAX_DEP - 1; i >= 0; i--) {
            if (fa[tu][i] == fa[tv][i]) {
                continue;
            }
            tu = fa[tu][i];
            tv = fa[tv][i];
        }
        return fa[tu][0];
    }

    // 动态更新节点的父亲属性
    void lineFa(int u, int v) {
        fa[u][0] = v;
        for (int i = 1; i < MAX_DEP; i++) {
            v = fa[u][i] = fa[v][i - 1];
        }
    }

    // 建树, op是gh的反向图,用来寻找其父亲
    void build(Graph &gh, Graph &op, int n, int root) {
        memcpy(deg, gh.deg, sizeof(int) * (n + 1));
        bfs(gh, root);
        for (int k = 1; k <= tot; k++) {
            int u = st[k], fath = -1;
            dep[u] = 0;
            for (int i = op.head[u]; ~i; i = op.eg[i].nxt) {
                int v = op.eg[i].e;
                if (dfn[v] > dfn[u]) continue;
                fath = (fath == -1 ? v : lca(fath, v));
            }
            if (fath == -1) fath = u;
            lineFa(u, fath);
            dep[u] = dep[fath] + 1;
        }
    }
} dtree;\end{lstlisting}
\subsection{Hopcroft-Karp}
\begin{lstlisting}
int dis;
int linkx[N], linky[N];
int dx[N], dy[N];
bool vis[N];

bool searchP(int n) {
    queue<int> q;
    dis = INF;
    mst(dx, -1, n);
    mst(dy, -1, n);
    for (int i = 0; i < n; i++) {
        if (linkx[i] == -1) {
            q.push(i);
            dx[i] = 0;
        }
    }
    while (!q.empty()) {
        int u = q.front();
        q.pop();
        if (dx[u] > dis) break;
        for (int i = gh.head[u]; ~i; i = gh.eg[i].nxt) {
            int v = gh.eg[i].e;
            if (dy[v] == -1) {
                dy[v] = dx[u] + 1;
                if (linky[v] == -1) {
                    dis = dy[v];
                } else {
                    dx[linky[v]] = dy[v] + 1;
                    q.push(linky[v]);
                }
            }
        }
    }
    return dis != INF;
}

bool dfs(int u) {
    for (int i = gh.head[u]; ~i; i = gh.eg[i].nxt) {
        int v = gh.eg[i].e;
        if (!vis[v] && dy[v] == dx[u] + 1) {
            vis[v] = true;
            if (linky[v] != -1 && dy[v] == dis) continue;
            if (linky[v] == -1 || dfs(linky[v])) {
                linky[v] = u;
                linkx[u] = v;
                return true;
            }
        }
    }
    return false;
}


int MaxMatch(int n) {
    int rst = 0;
    mst(linkx, -1, n);
    mst(linky, -1, n);
    while (searchP(n)) {
        mst(vis, false, n);
        for (int i = 0; i < n; i++) {
            if (linkx[i] == -1 && dfs(i)) {
                rst++;
            }
        }
    }
    return rst;
}\end{lstlisting}
\clearpage\section{博弈}
\subsection{GameProblem}
\begin{lstlisting}
// 巴什博奕, 是否先手必胜
inline bool bash_game(int n, int m) {
    //一堆东西, n个物品,最多选m个
    return n % (m + 1);
}

// 威佐夫博弈, 是否先手必胜
// 有两堆各若干的物品,两人轮流从其中一堆取至少一件物品,至多不限,或从两堆中同时取相同件物品,规定最后取完者胜利。
inline bool wythoff_game(int n, int m) {
    if (n > m) {
        swap(n, m);
    }
    int temp = floor((n2 - n1) * (1 + sqrt(5.0)) / 2.0);
    return temp != n1;
}
// SG函数
#define N 1001
//f[]:可以取走的石子个数
//sg[]:0~n的SG函数值
int f[N], sg[N], mex[N];

void getSG(int n) {
    int i, j;
    memset(sg, 0, sizeof(sg));
    for (i = 1; i <= n; i++) {
        memset(mex, 0, sizeof(mex));
        for (j = 1; f[j] <= i; j++)
            mex[sg[i - f[j]]] = 1;
        for (j = 0; j <= n; j++) { //求mes{}中未出现的最小的非负整数
            if (mex[j] == 0) {
                sg[i] = j;
                break;
            }
        }
    }
}

// Auti-nim 反尼姆游戏
// 当先拿完所有石子时候输
// 当如下条件时,先手必胜
// ①:所有堆的石子数均=1,且有偶数堆。 
// ②:至少有一个堆的石子数>1,且石子堆的异或和≠0。
\end{lstlisting}
\clearpage\section{分治}
\subsection{IntegerFastPower}
\begin{lstlisting}
ll fpow(ll x, ll k) {
    ll base = x, r = 1;
    for (; k; k >>= 1) {
        if (k & 1) r = r * base;
        base = base * base;
    }
    return r;
}\end{lstlisting}
\subsection{MatrixFastPower}
\begin{lstlisting}
#define MAX_N 10
#define mod_num 9973

struct Mat {
	long long mat[MAX_N][MAX_N];
	long long n;
	Mat() {
		memset(mat, 0, sizeof(mat));
		n = 0;
	}
	Mat(long long n) {
		memset(mat, 0, sizeof(mat));
		this->n = n;
	}
	void init() {
		for (int i = 0; i < n; ++i) {
			mat[i][i] = 1;
		}
	}
	Mat(const long long ** list, long long n) {
		this->n = n;
		for (int i = 0; i < n; ++i) {
			for (int j = 0; j < n; ++j) {
				mat[i][j] = list[i][j];
			}
		}
	}
};

Mat operator * (Mat a, Mat b) {
	long long n = a.n;
	Mat c(n);
	memset(c.mat, 0, sizeof(c.mat));
	for (int i = 0; i < n; ++i) {
		for (int j = 0; j < n; ++j) {
			for (int k = 0; k < n; ++k) {
				c.mat[i][j] += (a.mat[i][k] * b.mat[k][j]) % mod_num;
				c.mat[i][j] %= mod_num;
			}
		}
	}
	return c;
}

Mat operator ^ (Mat a, int k) {
	long long n = a.n;
	Mat c(n);
	c.init();
	for (; k; k >>= 1) {
		if (k & 1)   c = c * a;
		a = a * a;
	}
	return c;
}\end{lstlisting}
\clearpage\section{数论}
\subsection{线性基}
\begin{lstlisting}
#define rep(i, l, r) for(int i=(l);i<(r);++i)
#define per(i, l, r) for(int i=(r)-1;i>=(l);--i)
#define pw(x) (1ll << (x))
#define bt(x, i) ((x >> i) & 1)
const int LN = 61;
struct LB {
    ll d[LN] = {0}, p[LN] = {0};
    int g[LN] = {0};
    int cnt = 0;

    LB() = default;

    ll &operator[](int pos) {
        return d[pos];
    }

    const ll &operator[](int pos) const {
        return d[pos];
    }

    void insert(ll val, int pos) {
        per(i, 0, LN) {
            if (val & pw(i)) {
                if (!d[i]) {
                    d[i] = val;
                    g[i] = pos;
                    return;
                }
				// 贪心保留最右
                if (pos > g[i]) {
                    swap(pos, g[i]);
                    swap(val, d[i]);
                }
                val ^= d[i];
            }
        }
    }

    ll query_max(int l) {
        ll ret = 0;
        per(i, 0, LN) {
            if (g[i] >= l)
                ret = max(ret, ret ^ d[i]);
        }
        return ret;
    }

    ll query_max() {
        ll ret = 0;
        per(i, 0, LN) {
            ret = max(ret, ret ^ d[i]);
        }
        return ret;
    }

    ll query_min() {
        rep(i, 0, LN) {
            if (d[i]) return d[i];
        }
        return 0;
    }
	
	bool test(ll x) const {
        per(i, 0, LN) {
            if (bt(x, i)) {
                if (!d[i]) return false;
                x ^= d[i];
            }
        }
        return true;
    }

    void rebuild() {
        per(i, 0, LN) {
            per(j, 0, i) {
                if (d[i] & (1LL << j)) d[i] ^= d[j];
            }
        }
        rep(i, 0, LN) {
            if (d[i]) p[cnt++] = d[i];
        }
    }

    ll kth_query(ll k) {
        int ret = 0;
        if (k >= pw(cnt)) return -1;
        per(i, 0, LN) {
            if (bt(k, 1)) ret ^= p[i];
        }
        return ret;
    }
};
// 求并集
LB operator+(const LB &n1, const LB &n2) {
    LB ret = n1;
    per(i, 0, LN)
        if (n2.d[i])
            ret.insert(n1.d[i], n1.g[i]);
    return ret;
}
// 求交集
LB operator^(const LB &n1, const LB &n2) {
    LB ans = {}, c = n2, d = n2;
    rep(i, 0, LN) {
        ll x = n1[i];
        if (!x) continue;
        int p = i;
        ll T = 0;
        per(j, 0, p + 1) {
            if (bt(x, j)) {
                if (c[j]) {
                    x ^= c[j];
                    T ^= d[j];
                } else {
                    p = j;
                    break;
                }
            }
        }
        if (!x) {
            ans[i] = T;
        } else {
            c[p] = x;
            d[p] = T;
        }
    }
    return ans;
}\end{lstlisting}
\clearpage\section{其他}
\subsection{BigInteger}
\begin{lstlisting}
// base and base_digits must be consistent
constexpr int base = 1000000000;
constexpr int base_digits = 9;

struct bigint {
    // value == 0 is represented by empty z
    vector<int> z; // digits

    // sign == 1 <==> value >= 0
    // sign == -1 <==> value < 0
    int sign;

    bigint() : sign(1) {}

    bigint(ll v) { *this = v; }

    bigint &operator=(ll v) {
        sign = v < 0 ? -1 : 1;
        v *= sign;
        z.clear();
        for (; v > 0; v = v / base) z.push_back((int) (v % base));
        return *this;
    }

    bigint(const string &s) { read(s); }

    bigint &operator+=(const bigint &other) {
        if (sign == other.sign) {
            for (int i = 0, carry = 0; i < other.z.size() || carry; ++i) {
                if (i == z.size())
                    z.push_back(0);
                z[i] += carry + (i < other.z.size() ? other.z[i] : 0);
                carry = z[i] >= base;
                if (carry)
                    z[i] -= base;
            }
        } else if (other != 0 /* prevent infinite loop */) {
            *this -= -other;
        }
        return *this;
    }

    friend bigint operator+(bigint a, const bigint &b) { return a += b; }

    bigint &operator-=(const bigint &other) {
        if (sign == other.sign) {
            if (sign == 1 && *this >= other || sign == -1 && *this <= other) {
                for (int i = 0, carry = 0; i < other.z.size() || carry; ++i) {
                    z[i] -= carry + (i < other.z.size() ? other.z[i] : 0);
                    carry = z[i] < 0;
                    if (carry)
                        z[i] += base;
                }
                trim();
            } else {
                *this = other - *this;
                this->sign = -this->sign;
            }
        } else {
            *this += -other;
        }
        return *this;
    }

    friend bigint operator-(bigint a, const bigint &b) {
        return a -= b;
    }

    bigint &operator*=(int v) {
        if (v < 0) sign = -sign, v = -v;
        for (int i = 0, carry = 0; i < z.size() || carry; ++i) {
            if (i == z.size()) z.push_back(0);
            ll cur = (ll) z[i] * v + carry;
            carry = (int) (cur / base);
            z[i] = (int) (cur % base);
        }
        trim();
        return *this;
    }

    bigint operator*(int v) const { return bigint(*this) *= v; }

    friend pair<bigint, bigint> divmod(const bigint &a1, const bigint &b1) {
        int norm = base / (b1.z.back() + 1);
        bigint a = a1.abs() * norm;
        bigint b = b1.abs() * norm;
        bigint q, r;
        q.z.resize(a.z.size());

        for (int i = (int) a.z.size() - 1; i >= 0; i--) {
            r *= base;
            r += a.z[i];
            int s1 = b.z.size() < r.z.size() ? r.z[b.z.size()] : 0;
            int s2 = b.z.size() - 1 < r.z.size() ? r.z[b.z.size() - 1] : 0;
            int d = (int) (((ll) s1 * base + s2) / b.z.back());
            r -= b * d;
            while (r < 0) r += b, --d;
            q.z[i] = d;
        }

        q.sign = a1.sign * b1.sign;
        r.sign = a1.sign;
        q.trim();
        r.trim();
        return {q, r / norm};
    }

    friend bigint sqrt(const bigint &a1) {
        bigint a = a1;
        while (a.z.empty() || a.z.size() % 2 == 1) a.z.push_back(0);

        int n = a.z.size();

        int firstDigit = (int) ::sqrt((double) a.z[n - 1] * base + a.z[n - 2]);
        int norm = base / (firstDigit + 1);
        a *= norm;
        a *= norm;
        while (a.z.empty() || a.z.size() % 2 == 1) a.z.push_back(0);

        bigint r = (ll) a.z[n - 1] * base + a.z[n - 2];
        firstDigit = (int) ::sqrt((double) a.z[n - 1] * base + a.z[n - 2]);
        int q = firstDigit;
        bigint res;

        for (int j = n / 2 - 1; j >= 0; j--) {
            for (;; --q) {
                bigint r1 = (r - (res * 2 * base + q) * q) * base * base +
                            (j > 0 ? (ll) a.z[2 * j - 1] * base + a.z[2 * j - 2] : 0);
                if (r1 >= 0) {
                    r = r1;
                    break;
                }
            }
            (res *= base) += q;

            if (j > 0) {
                int d1 = res.z.size() + 2 < r.z.size() ? r.z[res.z.size() + 2] : 0;
                int d2 = res.z.size() + 1 < r.z.size() ? r.z[res.z.size() + 1] : 0;
                int d3 = res.z.size() < r.z.size() ? r.z[res.z.size()] : 0;
                q = (int) (((ll) d1 * base * base + (ll) d2 * base + d3) / (firstDigit * 2));
            }
        }

        res.trim();
        return res / norm;
    }

    bigint operator/(const bigint &v) const {
        return divmod(*this, v).first;
    }

    bigint operator%(const bigint &v) const {
        return divmod(*this, v).second;
    }

    bigint &operator/=(int v) {
        if (v < 0) sign = -sign, v = -v;
        for (int i = (int) z.size() - 1, rem = 0; i >= 0; --i) {
            ll cur = z[i] + rem * (ll) base;
            z[i] = (int) (cur / v);
            rem = (int) (cur % v);
        }
        trim();
        return *this;
    }

    bigint operator/(int v) const {
        return bigint(*this) /= v;
    }

    int operator%(int v) const {
        if (v < 0) v = -v;
        int m = 0;
        for (int i = (int) z.size() - 1; i >= 0; --i)
            m = (int) ((z[i] + m * (ll) base) % v);
        return m * sign;
    }

    bigint &operator*=(const bigint &v) {
        return *this = *this * v;;
    }

    bigint &operator/=(const bigint &v) {
        return *this = *this / v;
    }

    bool operator<(const bigint &v) const {
        if (sign != v.sign)
            return sign < v.sign;
        if (z.size() != v.z.size())
            return z.size() * sign < v.z.size() * v.sign;
        for (int i = (int) z.size() - 1; i >= 0; i--)
            if (z[i] != v.z[i])
                return z[i] * sign < v.z[i] * sign;
        return false;
    }

    bool operator>(const bigint &v) const { return v < *this; }

    bool operator<=(const bigint &v) const { return !(v < *this); }

    bool operator>=(const bigint &v) const { return !(*this < v); }

    bool operator==(const bigint &v) const { return !(*this < v) && !(v < *this); }

    bool operator!=(const bigint &v) const { return *this < v || v < *this; }

    void trim() {
        while (!z.empty() && z.back() == 0) z.pop_back();
        if (z.empty()) sign = 1;
    }

    bool isZero() const {
        return z.empty();
    }

    friend bigint operator-(bigint v) {
        if (!v.z.empty()) v.sign = -v.sign;
        return v;
    }

    bigint abs() const {
        return sign == 1 ? *this : -*this;
    }

    ll longValue() const {
        ll res = 0;
        for (int i = (int) z.size() - 1; i >= 0; i--)
            res = res * base + z[i];
        return res * sign;
    }

    friend bigint gcd(const bigint &a, const bigint &b) {
        return b.isZero() ? a : gcd(b, a % b);
    }

    friend bigint lcm(const bigint &a, const bigint &b) {
        return a / gcd(a, b) * b;
    }

    void read(const string &s) {
        sign = 1;
        z.clear();
        int pos = 0;
        while (pos < s.size() && (s[pos] == '-' || s[pos] == '+')) {
            if (s[pos] == '-') sign = -sign;
            ++pos;
        }
        for (int i = (int) s.size() - 1; i >= pos; i -= base_digits) {
            int x = 0;
            for (int j = max(pos, i - base_digits + 1); j <= i; j++)
                x = x * 10 + s[j] - '0';
            z.push_back(x);
        }
        trim();
    }

    friend istream &operator>>(istream &stream, bigint &v) {
        string s;
        stream >> s;
        v.read(s);
        return stream;
    }

    friend ostream &operator<<(ostream &stream, const bigint &v) {
        if (v.sign == -1)
            stream << '-';
        stream << (v.z.empty() ? 0 : v.z.back());
        for (int i = (int) v.z.size() - 2; i >= 0; --i)
            stream << setw(base_digits) << setfill('0') << v.z[i];
        return stream;
    }

    static vector<int> convert_base(const vector<int> &a, int old_digits, int new_digits) {
        vector<ll> p(max(old_digits, new_digits) + 1);
        p[0] = 1;
        for (int i = 1; i < p.size(); i++)
            p[i] = p[i - 1] * 10;
        vector<int> res;
        ll cur = 0;
        int cur_digits = 0;
        for (int v : a) {
            cur += v * p[cur_digits];
            cur_digits += old_digits;
            while (cur_digits >= new_digits) {
                res.push_back(int(cur % p[new_digits]));
                cur /= p[new_digits];
                cur_digits -= new_digits;
            }
        }
        res.push_back((int) cur);
        while (!res.empty() && res.back() == 0) res.pop_back();
        return res;
    }

    typedef vector<ll> vll;

    static vll karatsubaMultiply(const vll &a, const vll &b) {
        int n = a.size();
        vll res(n + n);
        if (n <= 32) {
            for (int i = 0; i < n; i++)
                for (int j = 0; j < n; j++)
                    res[i + j] += a[i] * b[j];
            return res;
        }

        int k = n >> 1;
        vll a1(a.begin(), a.begin() + k);
        vll a2(a.begin() + k, a.end());
        vll b1(b.begin(), b.begin() + k);
        vll b2(b.begin() + k, b.end());

        vll a1b1 = karatsubaMultiply(a1, b1);
        vll a2b2 = karatsubaMultiply(a2, b2);

        for (int i = 0; i < k; i++) a2[i] += a1[i];
        for (int i = 0; i < k; i++) b2[i] += b1[i];

        vll r = karatsubaMultiply(a2, b2);
        for (int i = 0; i < a1b1.size(); i++) r[i] -= a1b1[i];
        for (int i = 0; i < a2b2.size(); i++) r[i] -= a2b2[i];

        for (int i = 0; i < r.size(); i++) res[i + k] += r[i];
        for (int i = 0; i < a1b1.size(); i++) res[i] += a1b1[i];
        for (int i = 0; i < a2b2.size(); i++) res[i + n] += a2b2[i];
        return res;
    }

    bigint operator*(const bigint &v) const {
        vector<int> a6 = convert_base(this->z, base_digits, 6);
        vector<int> b6 = convert_base(v.z, base_digits, 6);
        vll a(a6.begin(), a6.end());
        vll b(b6.begin(), b6.end());
        while (a.size() < b.size()) a.push_back(0);
        while (b.size() < a.size()) b.push_back(0);
        while (a.size() & (a.size() - 1)) a.push_back(0), b.push_back(0);
        vll c = karatsubaMultiply(a, b);
        bigint res;
        res.sign = sign * v.sign;
        for (int i = 0, carry = 0; i < c.size(); i++) {
            ll cur = c[i] + carry;
            res.z.push_back((int) (cur % 1000000));
            carry = (int) (cur / 1000000);
        }
        res.z = convert_base(res.z, 6, base_digits);
        res.trim();
        return res;
    }
};
\end{lstlisting}
\subsection{FastIO}
\begin{lstlisting}
/*
 * FastIO
 * 代码模板 !
 * 如有雷同 !
 * 纯属巧合 !
 */
namespace FastIO {
#define BUF_SIZE 10000000
#define OUT_SIZE 10000000
#define ll long long
    //fread->read
    bool IOerror = 0;

    inline char nc() {
        static char buf[BUF_SIZE], *p1 = buf + BUF_SIZE, *pend = buf + BUF_SIZE;
        if (p1 == pend) {
            p1 = buf;
            pend = buf + fread(buf, 1, BUF_SIZE, stdin);
            if (pend == p1) {
                IOerror = 1;
                return -1;
            }
            //{printf("IO error!\n");system("pause");for (;;);exit(0);}
        }
        return *p1++;
    }

    inline bool blank(char ch) { return ch == ' ' || ch == '\n' || ch == '\r' || ch == '\t'; }

    inline void read(int &x) {
        bool sign = 0;
        char ch = nc();
        x = 0;
        for (; blank(ch); ch = nc());
        if (IOerror)return;
        if (ch == '-')sign = 1, ch = nc();
        for (; ch >= '0' && ch <= '9'; ch = nc())x = x * 10 + ch - '0';
        if (sign)x = -x;
    }

    inline void read(ll &x) {
        bool sign = 0;
        char ch = nc();
        x = 0;
        for (; blank(ch); ch = nc());
        if (IOerror)return;
        if (ch == '-')sign = 1, ch = nc();
        for (; ch >= '0' && ch <= '9'; ch = nc())x = x * 10 + ch - '0';
        if (sign)x = -x;
    }

    inline void read(double &x) {
        bool sign = 0;
        char ch = nc();
        x = 0;
        for (; blank(ch); ch = nc());
        if (IOerror)return;
        if (ch == '-')sign = 1, ch = nc();
        for (; ch >= '0' && ch <= '9'; ch = nc())x = x * 10 + ch - '0';
        if (ch == '.') {
            double tmp = 1;
            ch = nc();
            for (; ch >= '0' && ch <= '9'; ch = nc())tmp /= 10.0, x += tmp * (ch - '0');
        }
        if (sign)x = -x;
    }

    inline void read(char *s) {
        char ch = nc();
        for (; blank(ch); ch = nc());
        if (IOerror)return;
        for (; !blank(ch) && !IOerror; ch = nc())*s++ = ch;
        *s = 0;
    }

    inline void read(char &c) {
        for (c = nc(); blank(c); c = nc());
        if (IOerror) {
            c = -1;
            return;
        }
    }

    //fwrite->write
    struct Ostream_fwrite {
        char *buf, *p1, *pend;
        Ostream_fwrite() {
            buf = new char[OUT_SIZE];
            p1 = buf;
            pend = buf + OUT_SIZE;
        }
        void out(char ch) {
            if (p1 == pend) {
                fwrite(buf, 1, OUT_SIZE, stdout);
                p1 = buf;
            }
            *p1++ = ch;
        }
        void print(int x) {
            static char s[15], *s1;
            s1 = s;
            if (!x)*s1++ = '0';
            if (x < 0)out('-'), x = -x;
            while (x)*s1++ = x % 10 + '0', x /= 10;
            while (s1-- != s)out(*s1);
        }
        void println(int x) {
            static char s[15], *s1;
            s1 = s;
            if (!x)*s1++ = '0';
            if (x < 0)out('-'), x = -x;
            while (x)*s1++ = x % 10 + '0', x /= 10;
            while (s1-- != s)out(*s1);
            out('\n');
        }
        void print(ll x) {
            static char s[25], *s1;
            s1 = s;
            if (!x)*s1++ = '0';
            if (x < 0)out('-'), x = -x;
            while (x)*s1++ = x % 10 + '0', x /= 10;
            while (s1-- != s)out(*s1);
        }
        void println(ll x) {
            static char s[25], *s1;
            s1 = s;
            if (!x)*s1++ = '0';
            if (x < 0)out('-'), x = -x;
            while (x)*s1++ = x % 10 + '0', x /= 10;
            while (s1-- != s)out(*s1);
            out('\n');
        }
        void print(double x, int y) {
            static ll mul[] = {1, 10, 100, 1000, 10000, 100000, 1000000, 10000000, 100000000,
                               1000000000, 10000000000LL, 100000000000LL, 1000000000000LL, 10000000000000LL,
                               100000000000000LL, 1000000000000000LL, 10000000000000000LL, 100000000000000000LL};
            if (x < -1e-12)out('-'), x = -x;
            x *= mul[y];
            ll x1 = (ll) floor(x);
            if (x - floor(x) >= 0.5)++x1;
            ll x2 = x1 / mul[y], x3 = x1 - x2 * mul[y];
            print(x2);
            if (y > 0) {
                out('.');
                for (size_t i = 1; i < y && x3 * mul[i] < mul[y]; out('0'), ++i);
                print(x3);
            }
        }
        void println(double x, int y) {
            print(x, y);
            out('\n');
        }
        void print(char *s) { while (*s)out(*s++); }
        void println(char *s) {
            while (*s)out(*s++);
            out('\n');
        }
        void flush() {
            if (p1 != buf) {
                fwrite(buf, 1, p1 - buf, stdout);
                p1 = buf;
            }
        }
        ~Ostream_fwrite() { flush(); }
    } Ostream;
    inline void print(int x) { Ostream.print(x); }
    inline void println(int x) { Ostream.println(x); }
    inline void print(char x) { Ostream.out(x); }
    inline void println(char x) {
        Ostream.out(x);
        Ostream.out('\n');
    }
    inline void print(ll x) { Ostream.print(x); }
    inline void println(ll x) { Ostream.println(x); }
    inline void print(double x, int y) { Ostream.print(x, y); }
    inline void println(double x, int y) { Ostream.println(x, y); }
    inline void print(char *s) { Ostream.print(s); }
    inline void println(char *s) { Ostream.println(s); }
    inline void println() { Ostream.out('\n'); }
    inline void flush() { Ostream.flush(); }
};
using namespace FastIO;\end{lstlisting}
\subsection{InputOutputSpeedUp}
\begin{lstlisting}


template <class T>
inline bool read(T &x) {
    x = 0;
    char c = getchar();
	if(c == EOF) return false;
    bool f = false;
    for (; !isdigit(c); c = getchar()) f ^= (c == '-');
    for (; isdigit(c); c = getchar()) x = x * 10 + (c - '0');
    x = f ? -x : x;
	return true;
}
 
template <class T>
inline void write(T x) {
    if (x < 0) {
        putchar('-');
        x = -x;
    }
    T y = 1;
    int len = 1;
    for (; y <= x / 10; y *= 10) ++len;
    for (; len; --len, x %= y, y /= 10) putchar(x / y + '0');
}\end{lstlisting}
\subsection{gcd}
\begin{lstlisting}
ll gcd(ll x, ll y) { // 循环版
	ll t;
	while (y){
		t = x % y;
		x = y;
		y = t;
	}
	return  x;
}

ll gcd(ll a, ll b) { // 递归版
    return b == 0 ? a : gcd(b, a % b);
}

// 扩展欧几里得
ll exgcd(ll a, ll b, ll &x, ll &y) {
    if (b == 0) {
        x = 1, y = 0;
        return a;
    }
    ll q = exgcd(b, a % b, y, x);
    y -= a / b * x;
    return q;
}\end{lstlisting}
\subsection{myItoa}
\begin{lstlisting}
char * myItoa(int value, char* result, int base = 10);

char * myItoa(int value, char* result, int base) {
	// check that the base if valid  

	if (base < 2 || base > 16) { *result = 0; return result; }
	char* out = result;
	int quotient = abs(value);
	do {
		const int tmp = quotient / base;
		*out = "0123456789abcdef"[quotient - (tmp*base)];
		++out;
		quotient = tmp;
	} while (quotient);
	// Apply negative sign  
	if (value < 0) *out++ = '-';
	std::reverse(result, out);
	*out = 0;
	return result;
}\end{lstlisting}
\subsection{Permutation}
\begin{lstlisting}
// 错排问题
// D(n) = n! [(-1)^2/2! + … + (-1)^(n-1)/(n-1)! + (-1)^n/n!].
long long table[1000] = {0, 0, 1};
void init() {
    for (int i = 3; i <= 20; i++) {
        table[i] = (i - 1) * (table[i - 1] + table[i - 2]);
    }
}\end{lstlisting}
\subsection{prime}
\begin{lstlisting}
// 普通素数筛
const int PMAX = 1000000;

int prime_count = 0;
bool prime_list[PMAX] = { false };//元素值为0代表是素数
int prime_table[PMAX] = { 0 };

void initPrime() {
    for (int i = 2; i < PMAX; i++) {
        if (!prime_list[i])
            prime_table[prime_count++] = i;
        for (int j = 0, e = PMAX / i;
             j < prime_count && prime_table[j] <= e; j++) {
            prime_list[i * prime_table[j]] = true;
            if (i % prime_table[j] == 0) break;
        }
    }
}


// 可以得到其中一个质因子的素数筛
const int PMAX = 1000005;

int prime_count = 0;
int prime_list[PMAX] = {0}; //元素值为0代表是素数
int prime_table[PMAX] = {0};

void initPrime() {
    for (int i = 2; i < PMAX; i++) {
        if (!prime_list[i])
            prime_list[i] = prime_table[prime_count++] = i;
        for (int j = 0, e = PMAX / i, now; 
        	j < prime_count && (now = prime_table[j]) <= e; j++) {
            prime_list[i * now] = now;
            if (i % now == 0) break;
        }
    }
}\end{lstlisting}
\subsection{Hash}
\begin{lstlisting}
struct Hash {
    int num[N];
    int tot;

    void init() { tot = 0; }

    void insert(int x) { num[tot++] = x; }

    void build() {
        sort(num, num + tot);
        tot = unique(num, num + tot) - num;
    }

    inline int operator[](int x) { return lower_bound(num, num + tot, x) - num; }
} hs;\end{lstlisting}

\end{document}
