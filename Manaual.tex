
\documentclass[twoside]{article}
\usepackage[colorlinks,linkcolor=black]{hyperref}
\usepackage{xeCJK,xunicode}
\usepackage{fancyhdr}
\usepackage{amsmath,amsthm}
\usepackage{listings,xcolor}
\usepackage{geometry}
\usepackage{fontspec}
\setCJKmainfont[BoldFont=STYuanti-SC-Regular,ItalicFont=STYuanti-SC-Regular]{STYuanti-SC-Regular} 
\setCJKmonofont[BoldFont=STYuanti-SC-Regular]{STYuanti-SC-Regular} 
\setCJKsansfont{STYuanti-SC-Regular}
\setsansfont{Monaco}
\setmonofont[Mapping={}]{Monaco}
\newcommand{\HRule}{\rule{\linewidth}{0.5mm}}
\geometry{left=2.5cm,right=2.5cm,top=2.5cm,bottom=2.5cm}
\lstset{
    language    = c++,
    breaklines  = true,
    captionpos  = b,
    tabsize     = 4,
    numbers     = left,
    columns     = fullflexible,
    keepspaces  = true,
    commentstyle = \color[RGB]{0,128,0},
    keywordstyle = \color[RGB]{0,0,255},
    basicstyle   = \small\ttfamily,
    rulesepcolor = \color{red!20!green!20!blue!20},
    showstringspaces = false,
}
\title{ACM/ICPC Template Manaual}
\author{肥宅很快乐}
\pagestyle{fancy}
\fancyhf{}
\fancyhead[C]{ACM/ICPC Template Manaual, 肥宅很快乐}
\begin{document}\small
\begin{titlepage}
\begin{center}
\vspace*{0.5cm}\includegraphics[width=0.75\textwidth]{logo.jpg} \\ [2cm]
\HRule \\ [1cm]
\textbf{\Huge{ACM/ICPC Template Manaual}} \\ [0.5cm]
\HRule \\ [4cm]
\textbf{\Huge{浙江工商大学}} \\ [1cm]
\LARGE{肥宅很快乐}
\vfill
\Large{\today}
\end{center}
\clearpage
\end{titlepage}
\tableofcontents\clearpage
\pagestyle{fancy}
\lfoot{}
\cfoot{\thepage}\rfoot{}
\setcounter{section}{-1}
\setcounter{page}{1}
\clearpage\section{头文件}
\begin{lstlisting}
// 巨菜的ACMer-Happy233

#include <bits/stdc++.h>

using namespace std;

//-----
typedef double db;
typedef long long ll;
typedef unsigned int ui;
typedef vector<int> vi;
typedef pair<int, int> pii;
typedef pair<ll, ll> pll;
#define fi first
#define se second
#define pw(x) (1ll << (x))
#define bt(x, k) (((x) >> k) & 1)
#define sz(x) ((int)(x).size())
#define all(x) (x).begin(),(x).end()
#define rep(i, l, r) for(int i=(l);i<(r);++i)
#define per(i, l, r) for(int i=(r)-1;i>=(l);--i)
#define mst(t, v, n) memset(t, v, sizeof(decltype(*(t))) * (n))
#define sf(x) scanf("%d", &(x))
#ifndef ACM_LOCAL
#define endl '\n'
#endif

int main() {
#ifdef ACM_LOCAL
    freopen("./data/std.in", "r", stdin);
    // freopen("./data/std.out", "w", stdout);
#else
    ios_base::sync_with_stdio(false);
    cin.tie(0);
    cout.tie(0);
#endif

#ifdef ACM_LOCAL
    auto start = clock();
#endif
    int t = 1;
//    cin >> t;
    while (t--)
        solve();
#ifdef ACM_LOCAL
    auto end = clock();
    cerr << "Run Time: " << double(end - start) / CLOCKS_PER_SEC << "s" << endl;
#endif
    return 0;
}\end{lstlisting}
\clearpage\section{字串符}
\subsection{KMP}
\begin{lstlisting}
template<class elemType>
inline void kmp_nxt(elemType &T, vector<int> &nxt) {
    nxt[0] = -1;
    for (int i = 1; i < T.size(); i++) {
        int j = nxt[i - 1];
        while (j >= 0 && T[i - 1] != T[j]) j = nxt[j];
        if (j >= 0 && T[i - 1] == T[j]) nxt[i] = j + 1;
        else nxt[i] = 0;
    }
}

template<class elemType>
inline int kmp_count(elemType &S, elemType &T) {
    vector<int> nxt(T.size());
    kmp_nxt(T, nxt);
    int index, count = 0;
    for (index = 0; index < S.size(); ++index) {
        int pos = 0;
        int iter = index;
        while (pos < T.size() && iter < S.size()) {
            if (S[iter] == T[pos]) {
                ++iter;
                ++pos;
            } else {
                if (pos == 0) ++iter;
                else pos = nxt[pos - 1] + 1;
            }
        }
        if (pos == T.size() && (iter - index) == T.size()) ++count;
    }
    return count;
}

template<class elemType>
inline void kmp_next(elemType T[], int count, vector<int> &nxt) {
    nxt[0] = -1;
    for (int i = 1; i < count; i++) {
        int j = nxt[i - 1];
        while (j >= 0 && T[i - 1] != T[j]) j = nxt[j];
        if (j >= 0 && T[i - 1] == T[j]) nxt[i] = j + 1;
        else nxt[i] = 0;
    }
}

template<class elemType>
inline int kmp_count(elemType S[], int c1, elemType T[], int c2) {
    vector<int> nxt(c2);
    kmp_nxt(T, c2, nxt);
    int index, count = 0;
    for (index = 0; index < c1; ++index) {
        int pos = 0;
        int iter = index;
        while (pos < c2 && iter < c1) {
            if (S[iter] == T[pos]) {
                ++iter;
                ++pos;
            }
            else {
                if (pos == 0) ++iter;
                else pos = nxt[pos - 1] + 1;
            }
        }
        if (pos == c2 && (iter - index) == c2) ++count;
    }
    return count;
}\end{lstlisting}
\clearpage\section{动态规划}
\subsection{01Bag}
\begin{lstlisting}
void dp(int n, int m) {
    // n=物品个数
    for (int i = 0; i < n; i++) {
        // m=背包最大容量
        for (int j = m; j >= wei[i]; j--)
            // wei=大小 val=价值
            f[j] = max(f[j], f[j - wei[i]] + val[i]);
    }
}
\end{lstlisting}
\subsection{BagProblem}
\begin{lstlisting}
#define N 1000
// val=价值 wei=重量 num=数量
int val[N], wei[N], num[N], f[N];
// n=种类个数 m=背包最大值

// 01背包
void dp1(int n, int m) {
    for (int i = 0; i < n; i++) {
        for (int j = m; j >= wei[i]; j--)
            f[j] = max(f[j], f[j - wei[i]] + val[i]);
    }
}

// 完全背包
void dp2(int n, int m) {
    //初始化看要求
    for (int i = 0; i <= m; i++) {
        f[i] = INF;
    }
    f[0] = 0;
    //若要求恰好装满背包,那在初始化时除了f[0]=0其它f[1..V]均=-∞
    //若没要求背包装满,只希望价格大,初始化时应将f[0..V]=0)
    for (int i = 0; i < n; i++)
        for (int j = wei[i]; j <= m; j++)
            f[j] = max(f[j], f[j - wei[i]] + val[i]);
}

// 多重背包
void dp3(int n, int m) {
    for (int i = 0; i < n; i++)
        for (int k = 0; k < num[i]; k++)
            for (int j = m; j >= wei[i]; j--)
                f[j] = max(f[j], f[j - wei[i]] + val[i]);
}
\end{lstlisting}
\subsection{FullBag}
\begin{lstlisting}
/* 
完全背包问题的特点是,每种物品可以无限制的重复使用,可以选择放或不放。 
完全背包问题描述: 
有N物品和一个容量为V的背包。第i件物品的重量是wei[i],价值是val[i]。 
*/  
  
#include <cstdio>  
#define INF 0x3fffffff  
#define N 10047  
int f[N],val[N],wei[N];  
int min(int a,int b)  
{  
    return x<y?x:y;  
}  
int main()  
{  
    int t,i,j,k,E,F,m,n;  
    scanf("%d",&t);  
    while(t--)  
    {  
        scanf("%d%d",&E,&F);  
        int c = F-E;  
        for(i = 0 ; i <= c ; i++)  
            f[i]=INF;  
        scanf("%d",&n);  
        for(i = 0 ; i < n ; i++)  
        {  
            scanf("%d%d",&val[i],&wei[i]);//val[i]为面额,wei[i]为重量  
        }  
        f[0]=0;//因为此处假设的是小猪储钱罐 恰好装满 的情况  
        //注意初始化(要求恰好装满背包,那么在初始化时除了f[0]为0其它f[1..V]均设为-∞,  
        //这样就可以保证最终得到的f[N]是一种恰好装满背包的最优解。  
        //如果并没有要求必须把背包装满,而是只希望价格尽量大,初始化时应该将f[0..V]全部设为0)  
        for(i =0 ; i < n ; i++)  
        {  
            for(j = wei[i] ; j <= c ; j++)  
            {  
                f[j] = min(f[j],f[j-wei[i]]+val[i]);//此处求的是最坏的情况所以用min,确定最少的钱,当然最后就用max了,HEHE  
            }  
        }  
        if(f[c] == INF)  
            printf("This is impossible.\n");  
        else  
            printf("The minimum amount of money in the piggy-bank is %d.\n",f[c]);  
    }  
    return 0;  
}  
//此代码为HDU1114;  \end{lstlisting}
\subsection{MultiBag}
\begin{lstlisting}
//多重背包(MultiplePack): 有N种物品和一个容量为V的背包。  
//第i种物品最多有n[i]件可用,每件费用是c[i],价值是w[i]。  
//求解将哪些物品装入背包可使这些物品的费用总和不超过背包容量,  
//且价值总和最大。  
//HDU 2191  
  
#include <cstdio>  
#include <cstring>  
#define N 247  
int max(int x,int y){
    return x>y?x:y;  
}  
int main()  {  
    int t,n,m,i,j,k;  
    int w[N],pri[N],num[N],f[N];  
    while(~scanf("%d",&t)){  
        while(t--){  
            memset(f,0,sizeof(f));  
            scanf("%d%d",&n,&m);//n为总金额,m为大米种类  
            for(i = 0 ; i < m ; i++){  
                scanf("%d%d%d",&pri[i],&w[i],&num[i]);//num[i]为每种大米的袋数  
            }  
            for(i = 0 ; i < m ; i++){  
                for(k = 0 ; k < num[i] ; k++){  
                    for(j = n ; j >= pri[i]; j--){  
                        f[j] = max(f[j],f[j-pri[i]]+w[i]);  
                    }  
                }  
            }  
            printf("%d\n",f[n]);  
        }  
    }  
    return 0;  
}  \end{lstlisting}
\subsection{Maze01}
\begin{lstlisting}
struct Maze01 {
    // 原始01矩阵 1-n 1-m
    bool a[N][N];
    // 以(i,j)向上最大高度的最大子矩阵的左右高
    int l[N][N], r[N][N], h[N][N];
    // 最大连续前缀1的左端位置, 如果(i,j)为0则ml=j+1
    int ml[N][N];
    // 矩阵大小
    int n, m;

    void prework() {
        for (int i = 1; i <= m; i++) {
            l[0][i] = 0;
            r[0][i] = m;
        }
        for (int i = 1; i <= n; i++) {
            int maxl = 1;
            int maxr = m;
            for (int j = 1; j <= m; j++) {
                if (a[i][j] == 0) {
                    maxl = j + 1;
                    h[i][j] = l[i][j] = 0;
                } else {
                    h[i][j] = h[i - 1][j] + 1;
                    l[i][j] = max(maxl, l[i - 1][j]);
                }
                ml[i - 1][j] = maxl;
                ml[i][j] = j + 1;
            }
            for (int j = m; j >= 1; --j) {
                if (a[i][j] == 0) {
                    maxr = j - 1;
                    r[i][j] = m;
                } else {
                    r[i][j] = min(maxr, r[i - 1][j]);
                }
            }
        }
    }

    // 单调栈
    pii st[N];

    int calc() {
        prework();
        int ans = 0;
        for (int i = 1; i <= n; i++) {
            int top = 0;
            for (int j = 1; j <= m; j++) {
                if (h[i][j] == 0) continue;
                if (top == 0 || st[top] != make_pair(l[i][j], r[i][j])) {
                    st[++top] = make_pair(l[i][j], r[i][j]);
                }
                while (top && st[top].second == j) {
                    int pos = st[top--].first;
                    if (pos < ml[i][j]) {
                        // l[i][j]-r[i][j]为底, h[i][j]为高
                        // 可以得到所有的唯一子矩阵, 不存在maze1完全属于maze2
                        ans++;
                    }
                }
            }
        }
        return ans;
    }
} dp;\end{lstlisting}
\clearpage\section{数据结构}
\subsection{BTree}
\begin{lstlisting}
template<class T>

struct TreeNode {
    T value;
    TreeNode *left;
    TreeNode *right;
};

template<class T>
TreeNode<T> *createTree(const T *pre, const T *in, const int len) {
    TreeNode<T> *t = NULL;
    if (len > 0) {
        t = new TreeNode<T>;
        t->value = pre[0];
        int index;
        for (index = 0; index < len; index++) {
            if (in[index] == pre[0]) {
                break;
            }
        }
        if (index == len) {
            index = -1;
        }
        t->left = createTree(pre + 1, in, index);
        t->right = createTree(pre + index + 1, in + index + 1, len - index - 1);
    }
    return t;
}

template<class T>
int preOrder(TreeNode<T> *root, queue<T> &out) {
    if (root) {
        int count = 1;
        out.push(root->value);
        count += preOrder(root->left, out);
        count += preOrder(root->right, out);
        return count;
    } else {
        return 0;
    }
}

template<class T>
int inOrder(TreeNode<T> *root, queue<T> &out) {
    if (root) {
        int count = 1;
        count += inOrder(root->left, out);
        out.push(root->value);
        count += inOrder(root->right, out);
        return count;
    } else {
        return 0;
    }
}

template<class T>
void postOrder(TreeNode<T> *root, queue<T> &out) {
    if (root) {
        postOrder(root->left, out);
        postOrder(root->right, out);
        out.push(root->value);
    } else {
        return;
    }
}

template<class T>
T *convertQueueToArray(queue<T> &out, int len) {
    T *list = new T[len];
    int now = 0;
    while (!out.empty() && now < len) {
        list[now] = out.front();
        out.pop();
        now++;
    }
    return list;
}

template<class T>
void destroyTree(TreeNode<T> *root) {
    if (root) {
        destroyTree(root->left);
        destroyTree(root->right);
        delete root;
    } else return;
}

template<class T>
void insertIntoBSTree(TreeNode<T> *root, const T &value) {
    if (!root) {
        return;
    }
    if (value < root->value) {
        if (root->left) {
            insertIntoTree(root->left, value);
        } else {
            root->left = new TreeNode<T>;
            root->left->value = value;
            root->left->left = NULL;
            root->left->right = NULL;
        }
    } else if (value > root->value) {
        if (root->right) {
            insertIntoTree(root->right, value);
        } else {
            root->right = new TreeNode<T>;
            root->right->value = value;
            root->right->left = NULL;
            root->right->right = NULL;
        }
    }
}

template<class T>
TreeNode<T> *createBSTree(T *list, int len) {
    if (len < 1) {
        return NULL;
    }
    TreeNode<T> *root = new TreeNode<char>;
    root->value = list[0];
    root->left = NULL;
    root->right = NULL;
    for (int i = 1; i < len; i++) {
        insertIntoBSTree(root, list[i]);
    }
    return root;
}\end{lstlisting}
\subsection{pbds-bbtree}
\begin{lstlisting}
// RBTree 红黑树
#include <ext/pb_ds/tree_policy.hpp>
#include <ext/pb_ds/assoc_container.hpp>
// 红黑树
__gnu_pbds::tree<int, null_type, less<int>, rb_tree_tag, tree_order_statistics_node_update> t;
// null_type无映射(低版本g++为null_mapped_type)
// 类似multiset
__gnu_pbds::tree<int, null_type, less_equal<int>, rb_tree_tag, tree_order_statistics_node_update> t;
find_by_order(size_t order);
// 结点更新
tree_order_statistics_node_update
insert(p);
erase(it);
// 求k在树中是第几大:
order_of_key(p);
// 找到第order小的迭代器
find_by_order(order);
// 前驱
lower_bound(p);
// 后驱
upper_bound(p);
// 合并
a.join(b);
// 分割 key小于等于v的元素属于a,其余的属于b
a.split(v, b);

// 优先队列
#include <ext/pb_ds/priority_queue.hpp>
#include <ext/pb_ds/assoc_container.hpp>
// 操作类似于stl的优先队列
typedef __gnu_pbds::priority_queue<node, greater<node>, __gnu_pbds::thin_heap_tag> heap;
heap::point_iterator; // 指向元素的指针\end{lstlisting}
\subsection{树状数组}
\begin{lstlisting}
const int N = 1000005;
struct BITree {
    int n;
    ll c[N];

    void init(int _n) {
        n = _n + 1;
        memset(c, 0, sizeof(ll) * n);
    }

    void change(int pos, ll v) {
        for (int i = pos; i < n; i += i & (-i))
            c[i] += v;
    }

    ll query(int x) {
        ll ans = 0;
        for (int i = x; i > 0; i -= i & (-i))
            ans += c[i];
        return ans;
    }

    void update(int l, int r, ll v) {
        change(l, v);
        change(r + 1, -v);
    }
} tree;

// 区间更新区间查询
struct BITree {
    int n;
    ll c[N], d[N];

    void init(int _n) {
        n = _n + 1;
        memset(c, 0, sizeof(ll) * n);
        memset(d, 0, sizeof(ll) * n);
    }

    void change(int pos, ll v) {
        for (int i = pos; i < n; i += i & (-i))
            c[i] += v, d[i] += v * (pos - 1);
    }

    ll query(int x) {
        ll ans = 0;
        for (int i = x; i > 0; i -= i & (-i))
            ans += x * c[i] - d[i];
        return ans;
    }

    void update(int l, int r, ll v) {
        change(l, v);
        change(r + 1, -v);
    }

    ll sum(int l, int r) {
        return query(r) - query(l - 1);
    }
} tree;\end{lstlisting}
\subsection{二维树状数组}
\begin{lstlisting}
const int N = 2005;

inline int lowbit(const int &x) {
        return x & -x;
    }

struct TdBITree {
    int n, m;
    ll c[N][N];

    void init(int n, int m) {
        this->n = n;
        this->m = m;
        memset(c, 0, sizeof(c))
    }

    void init(int n, int m, ll v) {
        this->n = n;
        this->m = m;
        rep(x, 1, N) {
            rep(y, 1, N) {
                c[x][y] = (x * y + (x - lowbit(x)) * (y - lowbit(y)) - x * (y - lowbit(y)) - (x - lowbit(x)) * y) * v;
            }
        }
    }

    void change(int x, int y, ll v) {
        for (int i = x; i <= n; i += lowbit(i))
            for (int j = y; j <= m; j += lowbit(j))
                c[i][j] += v;
    }

    ll query(int x, int y) {
        ll ans = 0;
        for (int i = x; i >= 1; i -= lowbit(i))
            for (int j = y; j >= 1; j -= lowbit(j))
                ans += c[i][j];
        return ans;
    }

    ll solve(int x1, int y1, int x2, int y2) {
        return query(x2, y2) - query(x1 - 1, y2) - query(x2, y1 - 1) + query(x1 - 1, y1 - 1);
    }
};\end{lstlisting}
\subsection{线段树}
\begin{lstlisting}
// hdu 6562
struct TreeNode {
    int l, r;
    int lson, rson;
    ll sum;
    ll len;
    ll laz1, laz2, laz3;

    inline void init(int a, int b, int ls, int rs) {
        lson = ls;
        rson = rs;
        l = a, r = b;
        sum = 0;
        len = 1;
        laz1 = laz2 = 0;
        laz3 = 1;
    }

    inline int mid() {
        return (l + r) >> 1;
    }

    inline int width() {
        return r - l + 1;
    }

    inline void add(ll val) {
        sum = (sum * 10 + val * len * 10 + val * width()) % MOD;
        len = (len * 100) % MOD;
        // 右懒惰
        laz1 = (laz1 * 10 + val) % MOD;
        // 左懒惰
        laz2 = (laz2 + val * laz3) % MOD;
        // 长度懒惰
        laz3 = (laz3 * 10) % MOD;
    }
};

struct SegTree {
    int tot;
    TreeNode node[N << 1];

    inline void init() {
        tot = 0;
    }

    inline void up(int k) {
        TreeNode &nd = node[k];
        nd.sum = (node[nd.lson].sum + node[nd.rson].sum) % MOD;
        nd.len = (node[nd.lson].len + node[nd.rson].len) % MOD;
    }

    inline void push(int k) {
        TreeNode &nd = node[k];
        if (nd.laz3 == 1) return;
        TreeNode &lson = node[nd.lson];
        TreeNode &rson = node[nd.rson];

        lson.sum = (nd.laz1 * lson.width() + lson.sum * nd.laz3 + nd.laz2 * lson.len % MOD * nd.laz3) % MOD;
        lson.len = (lson.len * nd.laz3 % MOD * nd.laz3) % MOD;
        lson.laz1 = (lson.laz1 * nd.laz3 + nd.laz1) % MOD;
        lson.laz2 = (nd.laz2 * lson.laz3 + lson.laz2) % MOD;
        lson.laz3 = (lson.laz3 * nd.laz3) % MOD;

        rson.sum = (nd.laz1 * rson.width() + rson.sum * nd.laz3 + nd.laz2 * rson.len % MOD * nd.laz3) % MOD;
        rson.len = (rson.len * nd.laz3 % MOD * nd.laz3) % MOD;
        rson.laz1 = (rson.laz1 * nd.laz3 + nd.laz1) % MOD;
        rson.laz2 = (nd.laz2 * rson.laz3 + rson.laz2) % MOD;
        rson.laz3 = (rson.laz3 * nd.laz3) % MOD;

        nd.laz1 = nd.laz2 = 0;
        nd.laz3 = 1;
    }

    void build(int k, int l, int r) {
        TreeNode &nd = node[k];
        nd.init(l, r, tot + 1, tot + 2);
        tot += 2;
        if (l == r) {
            return;
        }
        int mid = nd.mid();
        build(nd.lson, nd.l, mid);
        build(nd.rson, mid + 1, nd.r);
        up(k);
    }

    void change(int k, int l, int r, ll val) {
        TreeNode &nd = node[k];
        if (nd.l == l && nd.r == r) {
            nd.add(val);
            return;
        }
        push(k);
        int mid = nd.mid();
        if (r <= mid) {
            change(nd.lson, l, r, val);
        } else if (l > mid) {
            change(nd.rson, l, r, val);
        } else {
            change(nd.lson, l, mid, val);
            change(nd.rson, mid + 1, r, val);
        }
        up(k);
    }

    ll query(int k, int l, int r) {
        TreeNode &nd = node[k];
        if (nd.l == l && nd.r == r) {
            return nd.sum;
        }
        push(k);
        int mid = nd.mid();
        ll ans = 0;
        if (r <= mid) {
            ans += query(nd.lson, l, r);
        } else if (l > mid) {
            ans += query(nd.rson, l, r);
        } else {
            ans += query(nd.lson, l, mid);
            ans += query(nd.rson, mid + 1, r);
        }
        return ans % MOD;
    }
} tree;\end{lstlisting}
\subsection{二维线段树}
\begin{lstlisting}
const int N = int(2e3 + 48);
 
ll p[N][N];
 
struct SegTree {
 
    inline int static get(int l, int r) {
        return (l + r) | (l != r);
    }
 
    struct node {
        int l, r;
 
        inline int mid() { return (l + r) >> 1; }
 
        inline bool point() { return l == r; }
 
        inline node operator--(int) { return {l, mid()}; }
 
        inline node operator++(int) { return {mid() + 1, r}; }
 
        bool operator==(node t) const { return l == t.l && r == t.r; }
 
        operator int() { return get(l, r); }
    };
 
    ll s[N * 2][N * 2];
 
    inline void upX(node x, node y) {
        s[x][y] = max(s[x--][y], s[x++][y]);
    }
 
    inline void upY(node x, node y) {
        s[x][y] = max(s[x][y--], s[x][y++]);
    }
 
    void buildY(node x, node y) {
        if (y.point()) {
            s[x][y] = p[x.l][y.l];
            return;
        }
        buildY(x, y--);
        buildY(x, y++);
        upY(x, y);
        return;
    }
 
    void mergeY(node x, node y) {
        if (y.point()) {
            upX(x, y);
            return;
        }
        mergeY(x, y--);
        mergeY(x, y++);
        upX(x, y);
        return;
    }
 
    void build(node x, node y) {
        if (x.point()) {
            buildY(x, y);
            return;
        }
        build(x--, y);
        build(x++, y);
        mergeY(x, y);
        return;
    }
 
    ll queryY(node x, node y, node a, node b) {
        if (y == b) {
            return s[x][y];
        }
        int mid = y.mid();
        if (b.r <= mid) {
            return queryY(x, y--, a, b);
        } else if (b.l > mid) {
            return queryY(x, y++, a, b);
        } else {
            return max(queryY(x, y--, a, {b.l, mid}),
                       queryY(x, y++, a, {mid + 1, b.r}));
        }
    }
 
    ll query(node x, node y, node a, node b) {
        if (x == a) {
            return queryY(x, y, a, b);
        }
        int mid = x.mid();
        if (a.r <= mid) {
            return query(x--, y, a, b);
        } else if (a.l > mid) {
            return query(x++, y, a, b);
        } else {
            return max(query(x--, y, {a.l, mid}, b),
                       query(x++, y, {mid + 1, a.r}, b));
        }
    }
} tree;\end{lstlisting}
\subsection{树状数组求逆序对}
\begin{lstlisting}
BITree t;
int n;
pii a[N];

void solve() {
    t.init(n);
    for (int i = 1; i <= n; i++) {
        int x;
        cin >> x;
        a[i] = make_pair(x, i);
    }
    sort(a + 1, a + n + 1);
    ll ans = 0;
    for (int i = 1; i <= n; i++) {
        t.change(a[i].second, 1);
        ans += (i - t.query(a[i].second));
    }
    cout << ans << endl;
}

\end{lstlisting}
\subsection{ST}
\begin{lstlisting}
// 只需要取值
struct ST {
    int ck[N];
    int dp[20][N];

    void init(int n, int squ[]) {
        ++n;
        ck[0] = -1;
        for (int i = 1; i <= n; i++) {
            ck[i] = ck[i - 1] + ((i & (i - 1)) == 0 ? 1 : 0);
        }
        for (int i = 0; i < n; i++) {
            dp[0][i] = squ[i];
        }
        for (int k = 1; k <= ck[n]; k++) {
            int dk = k - 1;
            for (int i = 0; i + (1 << k) - 1 < n; i++) {
                dp[k][i] = max(dp[dk][i], dp[dk][i + (1 << dk)]);
            }
        }
    }

    int query(int l, int r) {
        if (l > r) swap(l, r);
        int k = ck[r - l + 1];
        return max(dp[k][l], dp[k][r - (1 << k) + 1]);
    }
};

// 可得到下标
struct ST {
    int ck[N];
    int rmq[N];
    int dp[20][N];

    void init(int n, int squ[]) {
        ++n;
        ck[0] = -1;
        for (int i = 1; i <= n; i++) {
            ck[i] = ck[i - 1] + ((i & (i - 1)) == 0 ? 1 : 0);
        }
        memcpy(rmq, squ, sizeof(int) * n);
        for (int i = 0; i < n; i++) {
            dp[0][i] = i;
        }
        for (int k = 1; k <= ck[n]; k++) {
            int dk = k - 1;
            for (int i = 0; i + (1 << k) - 1 < n; i++) {
                int a = dp[dk][i];
                int b = dp[dk][i + (1 << dk)];
                dp[k][i] = rmq[a] < rmq[b] ? a : b;
            }
        }
    }

    int query(int l, int r) {
        if (l > r) swap(l, r);
        int k = ck[r - l + 1];
        int a = dp[k][l];
        int b = dp[k][r - (1 << k) + 1];
        return rmq[a] < rmq[b] ? a : b;
    }
};\end{lstlisting}
\subsection{笛卡尔树}
\begin{lstlisting}
// 笛卡尔树,静态建树,区间最值跳转
struct CartesianTree {
    int rt; // 根节点
    pii ch[N]; // 左右儿子
    int st[N]; // 单调栈

    void build(int n, int p[]) {
        rt = 0;
        int t = 0;
        for (int i = 1; i <= n; i++) {
            ch[i] = {0, 0};
            // 决定了大于还是小于
            while (t && p[st[t]] > p[i]) --t;
            if (t) { 
                // 上一个点的右儿子作为自己的左儿子
				// 成为上一个点的右儿子
                ch[i].first = ch[st[t]].second;
                ch[st[t]].second = i;
            } else { // 自己作为根节点
                ch[i].first = rt;
                rt = i;
            }
            st[++t] = i;
        }
    }
} dika;
\end{lstlisting}
\subsection{DancingLinks}
\begin{lstlisting}
// Dancing Links
struct DLX {
    int n, m, size;
    int U[MaxNode], D[MaxNode], L[MaxNode], R[MaxNode], Row[MaxNode], Col[MaxNode];
    int H[MaxN], S[MaxM];
    int ansd, ans[MaxN];

    void init(int _n, int _m) {
        n = _n;
        m = _m;
        for (int i = 0; i <= m; i++) {
            S[i] = 0;
            U[i] = D[i] = i;
            L[i] = i - 1;
            R[i] = i + 1;
        }
        R[m] = 0;
        L[0] = m;
        size = m;
        for (int i = 0; i <= n; i++) {
            H[i] = -1;
        }
    }

    void Link(int r, int c) {
        ++S[Col[++size] = c];
        Row[size] = r;
        D[size] = D[c];
        U[D[c]] = size;
        U[size] = c;
        D[c] = size;
        if (H[r] < 0) {
            H[r] = L[size] = R[size] = size;
        } else {
            R[size] = R[H[r]];
            L[R[H[r]]] = size;
            L[size] = H[r];
            R[H[r]] = size;
        }
    }

    void remove(int c) {
        L[R[c]] = L[c];
        R[L[c]] = R[c];
        for (int i = D[c]; i != c; i = D[i]) {
            for (int j = R[i]; j != i; j = R[j]) {
                U[D[j]] = U[j];
                D[U[j]] = D[j];
                --S[Col[j]];
            }
        }
    };

    void resume(int c) {
        for (int i = U[c]; i != c; i = U[i])
            for (int j = L[i]; j != i; j = L[j])
                ++S[Col[U[D[j]] = D[U[j]] = j]];
        L[R[c]] = R[L[c]] = c;
    }

    bool Dance(int d) {
        if (R[0] == 0) {
            for (int i = 0; i < d; i++) {
                printf("%d%c", ans[i], " \n"[i == d - 1]);
            }
            return true;
        }
        int c = R[0];
        for (int i = R[0]; i != 0; i = R[i]) if (S[i] < S[c]) c = i;
        remove(c);
        for (int i = D[c]; i != c; i = D[i]) {
            ans[d] = Row[i];
            for (int j = R[i]; j != i; j = R[j])remove(Col[j]);
            if (Dance(d + 1))return true;
            for (int j = L[i]; j != i; j = L[j])resume(Col[j]);
        }
        resume(c);
        return false;
    }
};\end{lstlisting}
\subsection{静态主席树}
\begin{lstlisting}
// m=update count,MAXN>=m*log(n)
const int N = int(2e5 + 10);
const int MAXN = int(1e7 + 10);

struct PSegTree {
    const int *a;
    pii ran; // Tree Range
    int c[MAXN];
    int tot = 0;
    int lson[MAXN], rson[MAXN];

    int build(int l, int r) {
        int k = ++tot;
        c[k] = 0;
        if (l == r) {
            return k;
        }
        int mid = (l + r) >> 1;
        lson[k] = build(l, mid);
        rson[k] = build(mid + 1, r);
        return k;
    }

    int init(int l, int r, const int num[]) {
        tot = 0;
        a = num;
        ran = {l, r};
        return build(l, r);
    }

    int update(int rt, int p, int v) {
        int k = ++tot, rst = k;
        int l, r;
        tie(l, r) = ran;
        // calc
        c[k] = c[rt] + v;
        while (l < r) {
            int mid = (l + r) >> 1;
            // 下面的逗号表达式顺序不能换
            if (p <= mid) {
                // go left
                rson[k] = rson[rt], rt = lson[rt], k = lson[k] = ++tot;
                r = mid;
            } else {
                // go right
                lson[k] = lson[rt], rt = rson[rt], k = rson[k] = ++tot;
                l = mid + 1;
            }
            // calc
            c[k] = c[rt] + v;
        }
        return rst;
    }

    // r1=right_root,r2=left_root,kth number
    int query(int r1, int r2, int k) {
        int l, r;
        tie(l, r) = ran;
        while (l < r) {
            int mid = (l + r) >> 1;
            int cnt = c[lson[r1]] - c[lson[r2]];
            if (cnt >= k) {
                r1 = lson[r1], r2 = lson[r2];
                r = mid;
            } else {
                k -= cnt;
                r1 = rson[r1], r2 = rson[r2];
                l = mid + 1;
            }
        }
        return l;
    }
	
	// r1=right_root,r2=left_root, sum of num<=k
    int sum(int r1, int r2, int k) {
        int l, r;
        tie(l, r) = ran;
        int ans = 0;
        while (l < r) {
            int mid = (l + r) >> 1;
            int cnt = c[lson[r1]] - c[lson[r2]];
            if (k <= mid) {
                r1 = lson[r1], r2 = lson[r2];
                r = mid;
            } else {
                ans += cnt;
                r1 = rson[r1], r2 = rson[r2];
                l = mid + 1;
            }
        }
        ans += c[r1] - c[r2];
        return ans;
    }
} tree;\end{lstlisting}
\subsection{动态主席树}
\begin{lstlisting}
// m: update count,MAXN>=m*log(n)^2
const int N = int(2e5 + 10);
const int MAXN = int(2e7 + 10);
const int MAXM = int(2e7 + 10);
const int LN = 40;

struct PSegTree {
    const int *a;
    pii ran;
    int n;
    ll c[MAXN];
    int tot = 0;
    int lson[MAXN], rson[MAXN];
    // t: static root, s: dynamic root
    int s[N], t[N];

    // SegTree Range and n points, num can be nullptr
    void init(int l, int r, int _n, const int num[]) {
        tot = 0;
        a = num;
        ran = {l, r};
        n = _n;
        for (int i = 0; i <= n; i++) s[i] = t[i] = 0;
        // init node 0
        lson[0] = rson[0] = 0;
        c[0] = 0;
    }

    inline int get(int &k, bool flag) {
        if (k == 0 || flag) {
            k = ++tot;
            lson[k] = rson[k] = 0;
        }
        return k;
    }

    // update the root in k
    void update(int k[], int rt[], int cnt, int p, int v, bool flag = false) {
        // calc
        for (int i = 0; i < cnt; i++)
            c[k[i]] = c[rt[i]] + v;
        int l, r;
        tie(l, r) = ran;
        while (l < r) {
            int mid = (l + r) >> 1;
            // 下面的逗号表达式顺序不能换
            if (p <= mid) {
                // go left
                for (int i = 0; i < cnt; i++) {
                    rson[k[i]] = rson[rt[i]];
                    rt[i] = lson[rt[i]], k[i] = get(lson[k[i]], flag);
                }
                r = mid;
            } else {
                // go right
                for (int i = 0; i < cnt; i++) {
                    lson[k[i]] = lson[rt[i]];
                    rt[i] = rson[rt[i]], k[i] = get(rson[k[i]], flag);
                }
                l = mid + 1;
            }
            // calc
            for (int i = 0; i < cnt; i++)
                c[k[i]] = c[rt[i]] + v;
        }
    }

    // build static tree
    void change(int pos, int p, int v) {
        // use int as the int[]
        // must use variable because I use the pointer
        int rt = t[pos - 1];
        int k = t[pos] = ++tot;
        update(&k, &rt, 1, p, v, true);
    }

    int use1[LN], use2[LN];

    // edit dynamic tree
    void add(int pos, int p, int v) {
        // memory reuse
        int *k = use1, *rt = use2;
        int cnt = 0;
        for (int i = pos; i <= n; i += (i & -i), cnt++) {
            rt[cnt] = s[i], k[cnt] = get(s[i], false);
        }
        update(k, rt, cnt, p, v);
    }

    // calc lson value in use
    inline ll sum(int use[], int cnt) {
        ll ans = 0;
        for (int i = 0; i < cnt; i++)
            ans += c[lson[use[i]]];
        return ans;
    }

    // calc value in use
    inline ll calc(int use[], int cnt) {
        ll ans = 0;
        for (int i = 0; i < cnt; i++)
            ans += c[use[i]];
        return ans;
    }

    // ans=p1-p2
    ll querySum(int p1, int p2, int k) {
        int rt1 = t[p1], rt2 = t[p2];
        ll cnt1 = 0, cnt2 = 0;
        // calc root in need
        for (int i = p1; i; i -= (i & -i)) use1[cnt1++] = s[i];
        for (int i = p2; i; i -= (i & -i)) use2[cnt2++] = s[i];
        int l, r;
        tie(l, r) = ran;
        ll ans = 0;
        while (l < r) {
            int mid = (l + r) >> 1;
            if (k <= mid) {
                // go left
                rt1 = lson[rt1], rt2 = lson[rt2];
                for (int i = 0; i < cnt1; i++) use1[i] = lson[use1[i]];
                for (int i = 0; i < cnt2; i++) use2[i] = lson[use2[i]];
                r = mid;
            } else {
                // go right
                ll cnt = sum(use1, cnt1) - sum(use2, cnt2) + c[lson[rt1]] - c[lson[rt2]];
                ans += cnt;
                rt1 = rson[rt1], rt2 = rson[rt2];
                for (int i = 0; i < cnt1; i++) use1[i] = rson[use1[i]];
                for (int i = 0; i < cnt2; i++) use2[i] = rson[use2[i]];
                l = mid + 1;
            }
        }
        ll cnt = calc(use1, cnt1) - calc(use2, cnt2) + c[rt1] - c[rt2];
        ans += cnt;
        return ans;
    }

    // query k
    int queryK(int p1, int p2, int k) {
        int r1 = t[p1], r2 = t[p2];
        int cnt1 = 0, cnt2 = 0;
        // calc root in need
        for (int i = p1; i; i -= (i & -i)) use1[cnt1++] = s[i];
        for (int i = p2; i; i -= (i & -i)) use2[cnt2++] = s[i];
        int l, r;
        tie(l, r) = ran;
        while (l < r) {
            int mid = (l + r) >> 1;
            int cnt = c[lson[r1]] - c[lson[r2]] + sum(use1, cnt1) - sum(use2, cnt2);
            if (cnt >= k) {
                // go left
                r1 = lson[r1], r2 = lson[r2];
                for (int i = 0; i < cnt1; i++) use1[i] = lson[use1[i]];
                for (int i = 0; i < cnt2; i++) use2[i] = lson[use2[i]];
                r = mid;
            } else {
                // go right
                k -= cnt;
                r1 = rson[r1], r2 = rson[r2];
                for (int i = 0; i < cnt1; i++) use1[i] = rson[use1[i]];
                for (int i = 0; i < cnt2; i++) use2[i] = rson[use2[i]];
                l = mid + 1;
            }
        }
        return l;
    }
} tree;\end{lstlisting}
\subsection{伸展树Splay}
\begin{lstlisting}
const int N = int(1e6 + 100);

struct Node;
Node *tail, *null;
// 回收栈
Node *st[N], **stop;

struct Node {
    // 内存池
    static Node pool[N];

    Node *ch[2], *fa;
    bool rev, edit;
    int size, val;
    ll sum;
    ll dpl, dpr, dpm; // 左端开始最大,右端开始最大,实际最大

    // 初始化内存池
    static void init() {
        null = tail = pool;
        null->clear();
        stop = st;
    }

    // 清空当前
    inline void clear(int _val = 0, int _size = 0) {
        edit = rev = false;
        sum = val = _val;
        size = _size;
        fa = ch[0] = ch[1] = null;
        dpl = dpr = dpm = _val;
    }

    inline void push_up() {
        if (this == null) return;
        size = ch[0]->size + ch[1]->size + 1;
        sum = ch[0]->sum + ch[1]->sum + val;
        Node *x = ch[0], *y = ch[1];
        if (size == 1) {
            dpl = dpr = dpm = val;
        } else if (y == null) {
            dpl = max(x->dpl, x->sum + val);
            dpr = max(1ll * val, val + x->dpr);
            dpm = max(x->dpm, val + max(x->dpr, 0ll));
        } else if (x == null) {
            dpl = max(1ll * val, val + y->dpl);
            dpr = max(y->dpr, y->sum + val);
            dpm = max(y->dpm, val + max(y->dpl, 0ll));
        } else {
            dpl = max(x->dpl, max(x->sum + val, x->sum + val + y->dpl));
            dpr = max(y->dpr, max(y->sum + val, y->sum + val + x->dpr));
            dpm = max(max(x->dpm, y->dpm), val + max(x->dpr, 0ll) + max(y->dpl, 0ll));
        }
    }

    // 设置儿子
    inline void setc(Node *p, int d) { ch[d] = p, p->fa = this; }

    // 获取方向
    inline bool d() { return fa->ch[1] == this; }

    inline void update_rev() {
        if (this == null) return;
        swap(ch[0], ch[1]);
        swap(dpl, dpr);
        rev ^= 1;
    }

    inline void update_val(int v) {
        if (this == null) return;
        val = v;
        sum = 1ll * size * v;
        edit = 1;
        dpl = dpr = dpm = max(1ll * val, sum);
    }

    inline void push_down() {
        if (this == null) return;
        if (rev) {
            ch[0]->update_rev(), ch[1]->update_rev();
            rev = 0;
        }
        if (edit) {
            ch[0]->update_val(val), ch[1]->update_val(val);
            edit = 0;
        }
    }

    inline bool isroot() {
        return fa == null || (this != fa->ch[0] && this != fa->ch[1]);
    }
};

Node Node::pool[N];

// 获取第k个
Node *get_kth(Node *x, int k) {
    while (x->ch[0]->size + 1 != k) {
        x->push_down();
        if (x->ch[0]->size >= k) {
            x = x->ch[0];
        } else {
            k -= x->ch[0]->size + 1;
            x = x->ch[1];
        }
    }
    return x;
}

void rotate(Node *x) {
    Node *f = x->fa, *ff = f->fa;
    f->push_down();
    x->push_down();
    int c = x->d(), cc = f->d();
    f->setc(x->ch[!c], c);
    x->setc(f, !c);
    if (ff->ch[cc] == f) ff->setc(x, cc);
    else x->fa = ff;
    f->push_up();
}

void splay(Node *&root, Node *x, Node *goal) {
    for (Node *f; (f = x->fa) != goal; rotate(x)) {
        if (f->fa == goal) continue;
        f->fa->push_down();
        f->push_down();
        x->push_down();
        rotate(x->d() == f->d() ? f : x);
    }
    x->push_up();
    if (goal == null) root = x;
}

int a[N];

// 获取有效节点
inline Node *getNode() { return stop == st ? ++tail : *--stop; }

Node *build(int l, int r) {
    int mid = (l + r) >> 1;
    Node *root = getNode();
    root->clear();
    root->val = a[mid];
    root->size = 1;
    if (l < mid) root->setc(build(l, mid - 1), 0);
    if (r > mid) root->setc(build(mid + 1, r), 1);
    root->push_up();
    return root;
}

// l->root, r->root.ch[1]
inline Node *make(Node *&root, int l, int r) {
    Node *x = get_kth(root, l);
    splay(root, x, null);
    Node *y = get_kth(root, r);
    splay(root, y, root);
    return y;
}

// 插入p子树
inline void insert(Node *&root, int l, int r, Node *p) {
    Node *x = make(root, l, r);
    x->setc(p, 0);
    x->push_up();
    x->fa->push_up();
}

// 回收内存
inline void clear(Node *&x) {
    if (x == null) return;
    auto now = stop;
    *stop++ = x;
    while (now != stop) {
        x = *now++;
        if (x->ch[0] != null) *stop++ = x->ch[0];
        if (x->ch[1] != null) *stop++ = x->ch[1];
    }
}

// 释放子树
inline void del(Node *&root, int l, int r) {
    Node *x = make(root, l, r);
    clear(x->ch[0]);
    x->ch[0] = null;
    x->push_up();
    x->fa->push_up();
}

// 转向
inline void reverse(Node *&root, int l, int r) {
    Node *x = make(root, l, r);
    x->ch[0]->update_rev();
    x->push_up();
    x->fa->push_up();
}

// set value
inline void make_same(Node *&root, int l, int r, int val) {
    Node *x = make(root, l, r);
    x->ch[0]->update_val(val);
    x->push_up();
    x->fa->push_up();
}

// 求和
inline ll sum(Node *&root, int l, int r) {
    Node *x = make(root, l, r);
    return x->ch[0]->sum;
}

// 求最大子串和
inline ll maxsum(Node *&root, int l, int r) {
    Node *x = make(root, l, r);
    return x->ch[0]->dpm;
}

void debug(Node *root) {
    function<void(Node *)> dfs = [&dfs](Node *x) {
        if (x == null) return;
        x->push_down();
        dfs(x->ch[0]);
        cout << x->val << ' ';
        dfs(x->ch[1]);
    };
    dfs(root);
    cout << endl;
}\end{lstlisting}
\subsection{带权并查集}
\begin{lstlisting}
const int N = 100005;
int f[N], deep[N], d[N];

void init(int n) {
    for (int i = 0; i <= n; i++) f[i] = i, d[i] = 0;
}

int fa(int x) {
    if (f[x] == x) return x;
    int fx = fa(f[x]);
    (d[x] += d[f[x]]) %= 3;
    return f[x] = fx;
}

bool un(int x, int y, int dis) {
    int fx = fa(x), fy = fa(y);
    if (fx == fy) {
        return ((d[y] - d[x] + 3) % 3 != dis);
    }
    int w = (3 + (d[x] - d[y]) + dis) % 3;
    // 按轶合并
    if (deep[fx] < deep[fy]) {
        swap(fx, fy);
        w = (3 - w) % 3;
    }
    f[fy] = fx;
    d[fy] = w;
    // 按轶合并
    deep[fx] += deep[fx] == deep[fy];
    return true;
}\end{lstlisting}
\subsection{可持久化并查集}
\begin{lstlisting}
const int N = int(2e5 + 10);
const int MAXN = N * 150;

struct PArray {
    int tot = 0;
    struct node {
        int val, deep;
    } c[MAXN];
    int lson[MAXN], rson[MAXN];
    pii ran;

    int build(int l, int r) {
        int k = tot++;
        if (l == r) {
            c[k].val = l;
            return k;
        }
        int mid = (l + r) >> 1;
        lson[k] = build(l, mid);
        rson[k] = build(mid + 1, r);
        return k;
    }

    int init(int l, int r) {
        ran = {l, r};
        return build(l, r);
    }

    int update(int rt, int p, node w) {
        int l = ran.first, r = ran.second;
        int k = tot++, rst = k;
        while (l < r) {
            lson[k] = lson[rt], rson[k] = rson[rt];
            int mid = (l + r) >> 1;
            if (p <= mid) {
                rt = lson[rt];
                k = lson[k] = tot++;
                r = mid;
            } else {
                rt = rson[rt];
                k = rson[k] = tot++;
                l = mid + 1;
            }
        }
        c[k].val = w.val;
        c[k].deep = c[rt].deep + w.deep;
        return rst;
    }

    node query(int rt, int p) {
        int l = ran.first, r = ran.second;
        while (l < r) {
            int mid = (l + r) >> 1;
            if (p <= mid) {
                rt = lson[rt];
                r = mid;
            } else {
                rt = rson[rt];
                l = mid + 1;
            }
        }
        return c[rt];
    }
};

struct PDSU {
    PArray tree;
    int root[N];

    void init(int n) {
        root[0] = tree.init(1, n);
    }

    PArray::node fa(int rt, int x) {
        PArray::node u = tree.query(root[rt], x);
        while (u.val != x) {
            x = u.val;
            u = tree.query(root[rt], x);
        }
        return u;
    }

    inline void change(int rt, int k) {
        root[k] = root[rt];
    }

    inline void un(int rt, int x, int y) {
        auto fx = fa(rt, x), fy = fa(rt, y);
        if (fx.val == fy.val) return;
        if (fx.deep > fy.deep) swap(fx, fy);
        root[rt] = tree.update(root[rt], fx.val, {fy.val, fx.deep});
        if (fx.deep == fy.deep) root[rt] = tree.update(root[rt], fy.val, {fy.val, fy.deep + 1});
    }
} dsu;\end{lstlisting}
\clearpage\section{图论}
\subsection{Graph}
\begin{lstlisting}
#define forg(i, h, eg) for(int i = (h); ~i; i = (eg[i]).first)

struct Edge {
    int to;
    ll w;
};

inline bool operator<(Edge a, Edge b) {
    return a.w < b.w;
}

const ll INF = ll(1e11);
const int N = int(1e6 + 10);
const int M = int(2e6 + 10);

struct Graph {
    pair<int, Edge> eg[M];
    int head[N];
    int tot;

    void init(int n) {
        memset(head, -1, sizeof(int) * ++n);
        tot = 0;
    }

    inline void addEdge(int x, Edge g) {
        eg[tot] = {head[x], g};
        head[x] = tot++;
    }
    
    Edge &operator[](int p) {
        return eg[p].second;
    }
} gh;\end{lstlisting}
\subsection{Dijkstra}
\begin{lstlisting}
int dist[N];
int path[N];

void bfs(int s, int n) {
    n++;
    rep(i, 0, n) dist[i] = INF;
    memset(path, -1, sizeof(int) * n);
    dist[s] = 0;
    path[s] = s;
	// 注意优先队列默认less运算,但选择最大的作为top,注意cmp!!!
    priority_queue<Edge, vector<Edge>, greater<Edge>> q;
    q.push(Edge(s, dist[s]));
    while (!q.empty()) {
        Edge f = q.top();
        q.pop();
        for (int i = gh.head[f.e]; ~i; i = gh.eg[i].nxt) {
            Edge &t = gh.eg[i];
            if (dist[t.e] > f.v + t.v) {
                dist[t.e] = f.v + t.v;
                path[t.e] = f.e;
                q.push(Edge(t.e, dist[t.e]));
            }
        }
    }
}

#include <ext/pb_ds/priority_queue.hpp>
#include <ext/pb_ds/assoc_container.hpp>
typedef __gnu_pbds::priority_queue<Edge, greater<Edge>> heap;
// 使用该模板,需要注意因为使用了greater,所以需要重载大于运算
// 默认pairing_heap_tag
// push O(1), pop O(logn) modify O(logn) erase O(logn) join O(1)
// 可选thin_heap_tag
// push O(1), pop O(logn) modify O(1) erase O(logn) join O(n)

heap::point_iterator its[N];
int cnt[N];

void bfs(int s, int n) {
    n++;
    rep(i, 0, n) dist[i] = INF;
    memset(cnt, 0, sizeof(int) * n);
    dist[s] = 0;
    cnt[s] = 1;
    heap q;
    its[s] = q.push(Edge(s, dist[s]));
    while (!q.empty()) {
        Edge f = q.top();
        q.pop();
        for (int i = gh.head[f.e]; ~i; i = gh.eg[i].nxt) {
            Edge &t = gh.eg[i];
            its[t.e] = 0;
            int v = f.v + t.v;
            if (dist[t.e] > v) {
                dist[t.e] = v;
                if (its[t.e] != 0) {
                    q.modify(its[t.e], Edge(t.e, dist[t.e]));
                } else {
                    its[t.e] = q.push(Edge(t.e, dist[t.e]));
                }
                cnt[t.e] = cnt[f.e];
            } else if (dist[t.e] == v) {
                (cnt[t.e] += cnt[f.e]) %= 100003;
            }
        }
    }
}\end{lstlisting}
\subsection{spfa}
\begin{lstlisting}
vector<int> dist;
vector<vector<node>> eg;
vector<int> path;

bool spfa(int n, int start) {
    dist.assign(n, INF);
    dist[start] = 0;
    deque<int> q;
    q.push_back(start);
    path.assign(n, -1);
    vector<int> cnt(n, 0);
    vector<bool> flag(n, false);
    cnt[start] = flag[start] = true;
    while (!q.empty()) {
        const int now = q.front();
        q.pop_front();
        flag[now] = false;
        for (auto i: eg[now]) {
            if (dist[i.x] > dist[now] + i.d) {
                dist[i.x] = dist[now] + i.d;
                path[i.x] = now;
                if (!flag[i.x]) {
                    if (n == ++cnt[i.x]) return false;
                    //队列非空且优于队首(SLF)
                    if (!q.empty() && dist[i.x] < dist[q.front()]) {
                        q.push_front(i.x);
                    } else {
                        q.push_back(i.x);
                    }
                    flag[i.x] = true;
                }
            }
        }
    }
    return true;
}
\end{lstlisting}
\subsection{Dinic}
\begin{lstlisting}
struct Graph {
    pair<int, Edge> eg[M];
    int head[N];
    int tot;

    void init(int n) {
        memset(head, -1, sizeof(int) * ++n);
        tot = 0;
    }

    inline void addEdge(int x, Edge g) {
        eg[tot] = {head[x], g};
        head[x] = tot++;
    }
} gh;

struct Dinic {
    Graph gh;
    // 点的范围[0, n)
    int n;
    // 弧优化
    int cur[N], dis[N];

    Dinic() {};

    // 设置N
    void init(int _n) {
        n = _n + 1;
        gh.init(n);
    }

    // 加流量
    void addFlow(int x, int y, ll f) {
        printf("%d->%d: %lld\n", x, y, f);
        gh.addEdge(x, {y, f});
        gh.addEdge(y, {x, 0});
    }

    bool bfs(int s, int e) {
        memset(dis, -1, sizeof(int) * n);
        int q[N];
        int l, r;
        l = r = 0;
        dis[s] = 0;
        q[r++] = s;
        while (l < r) {
            int f = q[l++];
            for (int i = gh.head[f]; ~i; i = gh.eg[i].first) {
                auto &eg = gh.eg[i].second;
                int to = eg.to;
                if (eg.w > 0 && dis[to] == -1) {
                    dis[to] = dis[f] + 1;
                    q[r++] = to;
                }
            }
        }
        return dis[e] > 0;
    }

    ll dfs(int s, int e, ll mx) {
        if (s == e || mx == 0) {
            return mx;
        }
        ll flow = 0;
        for (int &k = cur[s]; ~k; k = gh.eg[k].first) {
            auto &eg = gh.eg[k].second;
            ll a;
            if (eg.w > 0 && dis[eg.to] == dis[s] + 1 && (a = dfs(eg.to, e, min(eg.w, mx)))) {
                eg.w -= a;
                gh.eg[k ^ 1].second.w += a;
                flow += a;
                mx -= a;
                if (mx <= 0) break;
            }
        }
        return flow;
    }

    ll max_flow(int s, int e) {
        ll ans = 0;
        while (bfs(s, e)) {
            memcpy(cur, gh.head, sizeof(int) * n);
            ans += dfs(s, e, INF);
        }
        return ans;
    }
} dinic;\end{lstlisting}
\subsection{hungry}
\begin{lstlisting}
#define N 105
#define M 10005
int n, m, k;
pii eg[M * 2];
int result[N * 2];
int head[N * 2];
int cnt = 0;

void addEdge(int x, int y) {
    eg[cnt].first = y;
    eg[cnt].second = head[x];
    head[x] = cnt++;
}

bool vis[M * 2] = {false};

int dfs(int x) {
    for (int i = head[x]; ~i; i = eg[i].second) {
        int y = eg[i].first;
        if (!vis[y]) {
            vis[y] = true;
            if (result[y] == -1 || dfs(result[y])) {
                result[y] = x;
                return 1;
            }
        }
    }
    return 0;
}

int MaxMatch() {
    int ans = 0;
    memset(result, -1, sizeof(result));
    rep(i, 1, n + 1) {
        memset(vis, 0, sizeof(vis));
        ans += dfs(i);
    }
    return ans;
}

void solve() {
    scanf("%d%d", &m, &k);
    memset(head, -1, sizeof(head));
    cnt = 0;
    rep(i, 0, k) {
        int x, y;
        scanf("%d%d", &x, &y);
        addEdge(x, y);
    }
    int ans = MaxMatch();
    printf("%d\n", ans);
}\end{lstlisting}
\subsection{MinSpanTree}
\begin{lstlisting}
/*
* Prim 求 MST
* 耗费矩阵 cost[][],标号从 0 开始,0∼n-1
* 返回最小生成树的权值,返回 -1 表示原图不连通
 */
const int INF = 0x3f3f3f3f;
const int N = 110;
bool vis[N];
int lowc[N]; //点是 0 n-1 
int prim(int cost[][N], int n) {
    int ans = 0;
    memset(vis, false, sizeof(vis));
    vis[0] = true;
    for (int i = 1; i < n; i++)lowc[i] = cost[0][i];
    for (int i = 1; i < n; i++) {
        int minc = INF;
        int p =−1;
        19
        for (int j = 0; j < n; j++)
            if (!vis[j] && minc > lowc[j]) {
                minc = lowc[j];
                p = j;
            }
        if (minc == INF)return −1;//原图不连通 
        ans += minc;
        vis[p] = true;
        for (int j = 0; j < n; j++)
            if (!vis[j] && lowc[j] > cost[p][j])
                lowc[j] = cost[p][j];
    }
    return ans;
}\end{lstlisting}
\subsection{MinCostMaxFlow}
\begin{lstlisting}
struct Edge {
    int e, nxt;
    ll flow, cost;

    Edge() {};

    Edge(int a, ll b, ll c, int d = 0) : e(a), flow(b), cost(c), nxt(d) {}
};

const ll INF = ll(1e15);
const int N = int(2e3 + 10);
const int M = int(1e6 + 10);

//前向星
struct Graph {
    Edge eg[M];
    int head[N];
    int cnt;

    void init(int n) {
        memset(head, -1, sizeof(int) * ++n);
        cnt = 0;
    }

    inline void addEdge(int x, int y, ll v, ll c) {
        eg[cnt] = Edge(y, v, c, head[x]);
        head[x] = cnt++;
    }
};

struct MinCostMaxFlow {
    Graph gh;
    // 点的范围[0, n)
    int n;

    // 设置N
    void init(int _n) {
        n = _n + 1;
        gh.init(n);
    }

    // 加流量,反向是负的花费
    void addFlow(int x, int y, ll f, ll c) {
        // printf("%d->%d: %lld\t%lld\n", x, y, f, c); fflush(stdout);
        gh.addEdge(x, y, f, c);
        gh.addEdge(y, x, 0, -c);
    }

    // 该pre存的是边
    int pre[N];
    ll dis[N];
    bool vis[N];

    bool spfa(int s, int e) {
        queue<int> q;
        for (int i = 0; i < n; i++) {
            dis[i] = INF;
            vis[i] = false;
            pre[i] = -1;
        }
        dis[s] = 0;
        vis[s] = true;
        q.push(s);
        while (!q.empty()) {
            int u = q.front();
            q.pop();
            vis[u] = false;
            for (int i = gh.head[u]; ~i; i = gh.eg[i].nxt) {
                Edge &eg = gh.eg[i];
                if (eg.flow > 0 && dis[eg.e] > dis[u] + eg.cost) {
                    dis[eg.e] = dis[u] + eg.cost;
                    pre[eg.e] = i;
                    if (!vis[eg.e]) {
                        vis[eg.e] = true;
                        q.push(eg.e);
                    }
                }
            }
        }
        return pre[e] != -1;
    }

    pll cal(int s, int e) {
        ll flow = 0, cost = 0;
        while (spfa(s, e)) {
            ll f = INF;
            for (int i = pre[e]; ~i; i = pre[gh.eg[i ^ 1].e]) {
                f = min(f, gh.eg[i].flow);
            }
            for (int i = pre[e]; ~i; i = pre[gh.eg[i ^ 1].e]) {
                gh.eg[i].flow -= f;
                gh.eg[i ^ 1].flow += f;
                cost += gh.eg[i].cost;
            }
            flow += f;
        }
        return make_pair(flow, cost);
    }

} network;

const int N = int(2e3 + 10);
const ll INF = ll(1e15);

struct Edge {
    int e, nxt;
    ll flow, cost;

    Edge() {};

    Edge(int a, ll b, ll c, int d = 0) : e(a), flow(b), cost(c), nxt(d) {}
};

// vector图存
struct MinCostMaxFlow {
    vector<Edge> g[N];
    // 点的范围[0, n)
    int n = 0;

    // 设置N
    void init(int _n) {
        n = _n + 1;
        rep(i, 0, n) {
            g[i].clear();
        }
    }

    // 加流量,反向是负的花费
    void addFlow(int x, int y, ll f, ll c) {
        g[x].push_back(Edge(y, f, c, g[y].size()));
        g[y].push_back(Edge(x, 0, -c, g[x].size() - 1));
    }

    // 该pre存的是(点,边)
    pii pre[N];
    ll dis[N];
    bool vis[N];
    ll h[N];

    bool bfs(int s, int e) {
        using pli = pair<ll, int>;
        priority_queue<pli, vector<pli>, greater<pli>> q;
        fill_n(dis, n, INF);
        fill_n(vis, n, false);
        fill_n(pre, n, pii(-1, -1));
        dis[s] = 0;
        q.push(pii(0, s));
        while (!q.empty()) {
            pli f = q.top();
            int u = f.second;
            q.pop();
            if (f.first != dis[u]) continue;
            for (int i = 0; i < g[u].size(); i++) {
                auto &eg = g[u][i];
                if (eg.flow == 0) continue;
                int v = eg.e;
                ll cost = eg.cost + dis[u] + h[u] - h[v];
                if (dis[v] > cost) {
                    dis[v] = cost;
                    pre[v] = pii(u, i);
                    q.push(pii(dis[v], v));
                }
            }
        }
        for (int i = 0; i < n; i++) {
            h[i] += dis[i];
        }
        return pre[e].second != -1;
    }

    pll cal(int s, int e, ll limit) {
        ll flow = 0, cost = 0;
        fill_n(h, n, 0);
        while (limit) {
            if (!bfs(s, e)) break;
            ll f = INF;
            for (int i = e; ~pre[i].second; i = pre[i].first) {
                f = min(f, g[pre[i].first][pre[i].second].flow);
            }
            for (int i = e; ~pre[i].second; i = pre[i].first) {
                g[pre[i].first][pre[i].second].flow -= f;
                g[i][g[pre[i].first][pre[i].second].nxt].flow += f;
            }
            cost += f * h[e];
            flow += f;
            limit -= f;
        }
        return make_pair(flow, cost);
    }
} network;\end{lstlisting}
\subsection{ISAP}
\begin{lstlisting}
struct ISAP {
    Graph gh;
    // 点的范围[0, n)
    int n;
    // 弧优化
    int cur[N], dis[N];
    ISAP() {};
    // 设置N
    void init(int _n) {
        n = _n;
        gh.init(n);
    }

    // 加流量
    inline void addFlow(int x, int y, ll f) {
        gh.addEdge(x, y, f);
        gh.addEdge(y, x, 0);
    }

    int dep[N]; // 记录距离标号
    int gap[N]; // gap常数优化
    int q[N]; // 数组模拟队列

    void bfs(int s, int e) {
        memset(dep, -1, sizeof(int) * n);
        memset(gap, 0, sizeof(int) * n);
        gap[0] = 1;
        dep[e] = 0;
        int l = 0, r = 0;
        q[r++] = e;
        while (l < r) {
            int u = q[l++];
            for (int i = gh.head[u]; ~i; i = gh.eg[i].nxt) {
                int v = gh.eg[i].e;
                if (~dep[v]) continue;
                q[r++] = v;
                dep[v] = dep[u] + 1;
                gap[dep[v]]++;
            }
        }
    }

    ll st[N]; // 栈优化

    ll max_flow(int s, int e) {
        bfs(s, e);
        memcpy(cur, gh.head, sizeof(int) * n);
        int top = 0;
        int u = s;
        ll ans = 0;
        while (dep[s] < N) {
            if (u == e) {
                ll mf = INF;
                int sel = 0;
                for (int i = 0; i < top; i++) {
                    if (mf > gh.eg[st[i]].v) {
                        mf = gh.eg[st[i]].v;
                        sel = i;
                    }
                }

                for (int i = 0; i < top; i++) {
                    gh.eg[st[i]].v -= mf;
                    gh.eg[st[i] ^ 1].v += mf;
                }
                ans += mf;
                top = sel;
                u = gh.eg[st[top] ^ 1].e;
                continue;
            }
            bool flag = false;
            int v = 0;
            for (int i = cur[u]; ~i; i = gh.eg[i].nxt) {
                v = gh.eg[i].e;
                if (gh.eg[i].v > 0 && dep[v] + 1 == dep[u]) {
                    flag = true;
                    cur[u] = i;
                    break;
                }
            }
            if (flag) {
                st[top++] = cur[u];
                u = v;
                continue;
            }
            int mind = N;
            for (int i = gh.head[u]; ~i; i = gh.eg[i].nxt) {
                if (gh.eg[i].v > 0 && dep[gh.eg[i].e] < mind) {
                    mind = dep[gh.eg[i].e];
                    cur[u] = i;
                }
            }
            gap[dep[u]]--; // 当前层无法连通,降层
            if (!gap[dep[u]]) return ans; // 断层结束运算
            dep[u] = mind + 1; // 进入更高层
            gap[dep[u]]++;
            if (u != s)  u = gh.eg[st[--top] ^ 1].e;
        }
        return ans;
    }
} isap;\end{lstlisting}
\subsection{树链剖分}
\begin{lstlisting}
struct TreeChain {
    int top[N]; // 链条顶端点ID
    int fa[N]; // 父亲节点
    int son[N]; // 重儿子
    int deep[N]; // 深度
    int num[N]; // 儿子节点数(包括自己)

    int p[N]; // 在线段树中的ID,
    int fp[N]; // 线段树中ID对应的点
    int tot;

    void dfs(int u, int pre, int d) {
        num[u] = 1;
        deep[u] = d;
        fa[u] = pre;
        son[u] = -1;
        for (int i = gh.head[u]; ~i; i = gh.eg[i].first) {
            int v = gh.eg[i].second.to;
            if (v == pre) continue;
            dfs(v, u, d + 1);
            num[u] += num[v];
            if (son[u] == -1 || num[v] > num[son[u]]) {
                son[u] = v;
            }
        }
    }

    void getpos(int u, int sp) {
        top[u] = sp;
        p[u] = tot++;
        fp[p[u]] = u;
        if (son[u] == -1) return;
        getpos(son[u], sp);
        for (int i = gh.head[u]; ~i; i = gh.eg[i].first) {
            int v = gh.eg[i].second.to;
            if (v == son[u] || v == fa[u]) continue;
            getpos(v, v);
        }
    }
    
    // start是线段树中的ID起始数值
    void build(int n, int root = 1, int start = 1) {
        tot = start;
        // deep起始值如果是0,则遇到数据错误超出这棵树,有可能死循环
        // dfs(root, 0, 0);
        dfs(root, 0, 1);
        getpos(root, root);
    }

    int lca(int u, int v) {
        int f1 = top[u];
        int f2 = top[v];
        while (f1 != f2) {
            if (deep[f1] < deep[f2]) {
                swap(f1, f2);
                swap(u, v);
            }
            u = fa[f1];
            f1 = top[u];
        }
        return deep[u] < deep[v] ? u : v;
    }
} tch;

// 树状数组,如果无需在线查询可以使用差分树
BITree tree;

// 点权修改
void change1(int u, int v, ll val) {
    int f1 = tch.top[u];
    int f2 = tch.top[v];
    while (f1 != f2) {
        if (tch.deep[f1] < tch.deep[f2]) {
            swap(f1, f2);
            swap(u, v);
        }
        tree1.update(tch.p[f1], tch.p[u], val);
        u = tch.fa[f1];
        f1 = tch.top[u];
    }
    if (tch.deep[u] > tch.deep[v]) {
        swap(u, v);
    }
    tree1.update(tch.p[u], tch.p[v], val);
}

// 边权修改
void change2(int u, int v, ll val) {
    int f1 = tch.top[u];
    int f2 = tch.top[v];
    while (f1 != f2) {
        if (tch.deep[f1] < tch.deep[f2]) {
            swap(f1, f2);
            swap(u, v);
        }
        tree2.update(tch.p[f1], tch.p[u], val);
        u = tch.fa[f1];
        f1 = tch.top[u];
    }
    if (tch.deep[u] > tch.deep[v]) {
        swap(u, v);
    }
    tree2.update(tch.p[tch.son[u]], tch.p[v], val);
}\end{lstlisting}
\subsection{倍增LCA}
\begin{lstlisting}
const int MAX_DEP = 20;

// 倍增2^k的父亲
int fa[N][MAX_DEP];
int dep[N];

// 倍增LCA
int lca(int u, int v) {
    if (dep[u] > dep[v]) {
        swap(u, v);
    }
    int hu = dep[u], hv = dep[v];
    int tu = u, tv = v;
    for (int det = hv - hu, i = 0; det; det >>= 1, i++) {
        if (det & 1)
            tv = fa[tv][i];
    }
    if (tu == tv) {
        return tu;
    }
    for (int i = MAX_DEP - 1; i >= 0; i--) {
        if (fa[tu][i] == fa[tv][i]) {
            continue;
        }
        tu = fa[tu][i];
        tv = fa[tv][i];
    }
    return fa[tu][0];
}

// 动态更新节点的父亲属性
void lineFa(int u, int v) {
    fa[u][0] = v;
    for (int i = 1; i < MAX_DEP; i++) {
        v = fa[u][i] = fa[v][i - 1];
    }
}\end{lstlisting}
\subsection{Tarjan}
\begin{lstlisting}
int dfn[N], low[N], st[N], belong[N], num[N];
bool inst[N];
int idx, top, scc;

void tarjan(int u) {
    dfn[u] = low[u] = ++idx;
    st[top++] = u;
    inst[u] = true;
    for (int i = gh.head[u]; i != -1; i = gh.eg[i].nxt) {
        int v = gh.eg[i].e;
        if (!dfn[v]) {
            tarjan(v);
            low[u] = min(low[u], low[v]);
        } else if (inst[v]) {
            low[u] = min(low[u], dfn[v]);
        }
    }
    int v;
    if (dfn[u] == low[u]) {
        scc++;
        do {
            v = st[--top];
            inst[v] = false;
            belong[v] = scc;
            num[scc]++;
        } while (u != v);
    }
}

void work() {
    fill_n(dfn, n + 1, 0);
    fill_n(num, n + 1, 0);
    fill_n(inst, n + 1, false);
    idx = top = scc = 0;
    for (int i = 1; i <= n; i++) {
        if (!dfn[i])
            tarjan(i);
    }
}\end{lstlisting}
\subsection{支配树}
\begin{lstlisting}
const int MAX_DEP = 20;

// 注意0,1点的边界问题
struct DominatorTree {
    int deg[N]; // 入度
    int dep[N]; //
    int dfn[N];
    int st[N];
    int tot;

    // 拓扑序, 要保证root是入度为0
    void bfs(Graph &gh, int root) {
        queue<int> q;
        q.push(root);
        tot = 0;
        while (!q.empty()) {
            int u = q.front();
            q.pop();
            dfn[u] = ++tot;
            st[tot] = u;
            forg(i, gh.head[u], gh.eg) {
                int v = gh.eg[i].e;
                if ((--deg[v]) == 0) {
                    q.push(v);
                }
            }
        }
    }

    // 倍增2^k的父亲
    int fa[N][MAX_DEP];

    // 倍增LCA
    int lca(int u, int v) {
        if (dep[u] > dep[v]) {
            swap(u, v);
        }
        int hu = dep[u], hv = dep[v];
        int tu = u, tv = v;
        for (int det = hv - hu, i = 0; det; det >>= 1, i++) {
            if (det & 1)
                tv = fa[tv][i];
        }
        if (tu == tv) {
            return tu;
        }
        for (int i = MAX_DEP - 1; i >= 0; i--) {
            if (fa[tu][i] == fa[tv][i]) {
                continue;
            }
            tu = fa[tu][i];
            tv = fa[tv][i];
        }
        return fa[tu][0];
    }

    // 动态更新节点的父亲属性
    void lineFa(int u, int v) {
        fa[u][0] = v;
        for (int i = 1; i < MAX_DEP; i++) {
            v = fa[u][i] = fa[v][i - 1];
        }
    }

    // 建树, op是gh的反向图,用来寻找其父亲
    void build(Graph &gh, Graph &op, int n, int root) {
        memcpy(deg, gh.deg, sizeof(int) * (n + 1));
        bfs(gh, root);
        for (int k = 1; k <= tot; k++) {
            int u = st[k], fath = -1;
            dep[u] = 0;
            for (int i = op.head[u]; ~i; i = op.eg[i].nxt) {
                int v = op.eg[i].e;
                if (dfn[v] > dfn[u]) continue;
                fath = (fath == -1 ? v : lca(fath, v));
            }
            if (fath == -1) fath = u;
            lineFa(u, fath);
            dep[u] = dep[fath] + 1;
        }
    }
} dtree;\end{lstlisting}
\subsection{Hopcroft-Karp}
\begin{lstlisting}
int dis;
// linkx: x链接的y,linky: y链接的x
int linkx[N], linky[N];
int dx[N], dy[N];
bool vis[N];

bool searchP(int n) {
    queue<int> q;
    dis = INF;
    mst(dx, -1, n);
    mst(dy, -1, n);
    for (int i = 0; i < n; i++) {
        if (linkx[i] == -1) {
            q.push(i);
            dx[i] = 0;
        }
    }
    while (!q.empty()) {
        int u = q.front();
        q.pop();
        if (dx[u] > dis) break;
        for (int i = gh.head[u]; ~i; i = gh.eg[i].nxt) {
            int v = gh.eg[i].e;
            if (dy[v] == -1) {
                dy[v] = dx[u] + 1;
                if (linky[v] == -1) {
                    dis = dy[v];
                } else {
                    dx[linky[v]] = dy[v] + 1;
                    q.push(linky[v]);
                }
            }
        }
    }
    return dis != INF;
}

bool dfs(int u) {
    for (int i = gh.head[u]; ~i; i = gh.eg[i].nxt) {
        int v = gh.eg[i].e;
        if (!vis[v] && dy[v] == dx[u] + 1) {
            vis[v] = true;
            if (linky[v] != -1 && dy[v] == dis) continue;
            if (linky[v] == -1 || dfs(linky[v])) {
                linky[v] = u;
                linkx[u] = v;
                return true;
            }
        }
    }
    return false;
}

int MaxMatch(int n) {
    int rst = 0;
    mst(linkx, -1, n);
    mst(linky, -1, n);
    while (searchP(n)) {
        mst(vis, false, n);
        for (int i = 0; i < n; i++) {
            if (linkx[i] == -1 && dfs(i)) {
                rst++;
            }
        }
    }
    return rst;
}\end{lstlisting}
\subsection{Kuhn-Munkres}
\begin{lstlisting}
const int N = 1005;
const ll INF = ll(2e11);

ll cost[N][N];
// lx, ly 表示当前期望值, slack是得到链接的期望值
ll lx[N], ly[N], slack[N];
// prey->alternating tree, match是y->x的链接
int prey[N], match[N];
bool vy[N]; // 是否已经变化的点

void augment(int n, int root) {
    fill_n(vy + 1, n, false);
    fill_n(slack + 1, n, INF);
    int py = 0;
    match[py] = root;
    do {
        vy[py] = true;
        int x = match[py], yy = 0;
        ll delta = INF;
        for (int y = 1; y <= n; y++) {
            if (vy[y]) continue;
            ll w = lx[x] + ly[y] - cost[x][y];
            if (w < slack[y])
                slack[y] = w, prey[y] = py;
            if (slack[y] < delta) delta = slack[y], yy = y;
        }
        for (int y = 0; y <= n; y++) {
            if (vy[y])
                lx[match[y]] -= delta, ly[y] += delta;
            else
                slack[y] -= delta;
        }
        py = yy;
    } while (match[py] != -1);
    do {
        int pre = prey[py];
        match[py] = match[pre], py = pre;
    } while (py);
}

ll KM(int n) {
    fill_n(match + 1, n, -1);
    fill_n(ly + 1, n, 0);
    for (int i = 1; i <= n; i++) {
        lx[i] = *max_element(cost[i] + 1, cost[i] + n + 1);
    }
    ll ans = 0;
    for (int root = 1; root <= n; root++) augment(n, root);
    for (int i = 1; i <= n; i++) ans += lx[i], ans += ly[i];
    return ans;
}\end{lstlisting}
\clearpage\section{博弈}
\subsection{GameProblem}
\begin{lstlisting}
// 巴什博奕, 是否先手必胜
inline bool bash_game(int n, int m) {
    //一堆东西, n个物品,最多选m个
    return n % (m + 1);
}

// 威佐夫博弈, 是否先手必胜
// 有两堆各若干的物品,两人轮流从其中一堆取至少一件物品,至多不限,或从两堆中同时取相同件物品,规定最后取完者胜利。
inline bool wythoff_game(int n, int m) {
    if (n > m) {
        swap(n, m);
    }
    int temp = floor((n2 - n1) * (1 + sqrt(5.0)) / 2.0);
    return temp != n1;
}
// SG函数
#define N 1001
//f[]:可以取走的石子个数
//sg[]:0~n的SG函数值
int f[N], sg[N], mex[N];

void getSG(int n) {
    int i, j;
    memset(sg, 0, sizeof(sg));
    for (i = 1; i <= n; i++) {
        memset(mex, 0, sizeof(mex));
        for (j = 1; f[j] <= i; j++)
            mex[sg[i - f[j]]] = 1;
        for (j = 0; j <= n; j++) { //求mes{}中未出现的最小的非负整数
            if (mex[j] == 0) {
                sg[i] = j;
                break;
            }
        }
    }
}

// Auti-nim 反尼姆游戏
// 当先拿完所有石子时候输
// 当如下条件时,先手必胜
// ①:所有堆的石子数均=1,且有偶数堆。 
// ②:至少有一个堆的石子数>1,且石子堆的异或和≠0。
\end{lstlisting}
\clearpage\section{分治}
\subsection{IntegerFastPower}
\begin{lstlisting}
ll fpow(ll x, ll k) {
    ll base = x, r = 1;
    for (; k; k >>= 1) {
        if (k & 1) r = r * base;
        base = base * base;
    }
    return r;
}\end{lstlisting}
\subsection{MatrixFastPower}
\begin{lstlisting}
#define MAX_N 10
#define mod_num 9973

struct Mat {
	long long mat[MAX_N][MAX_N];
	long long n;
	Mat() {
		memset(mat, 0, sizeof(mat));
		n = 0;
	}
	Mat(long long n) {
		memset(mat, 0, sizeof(mat));
		this->n = n;
	}
	void init() {
		for (int i = 0; i < n; ++i) {
			mat[i][i] = 1;
		}
	}
	Mat(const long long ** list, long long n) {
		this->n = n;
		for (int i = 0; i < n; ++i) {
			for (int j = 0; j < n; ++j) {
				mat[i][j] = list[i][j];
			}
		}
	}
};

Mat operator * (Mat a, Mat b) {
	long long n = a.n;
	Mat c(n);
	memset(c.mat, 0, sizeof(c.mat));
	for (int i = 0; i < n; ++i) {
		for (int j = 0; j < n; ++j) {
			for (int k = 0; k < n; ++k) {
				c.mat[i][j] += (a.mat[i][k] * b.mat[k][j]) % mod_num;
				c.mat[i][j] %= mod_num;
			}
		}
	}
	return c;
}

Mat operator ^ (Mat a, int k) {
	long long n = a.n;
	Mat c(n);
	c.init();
	for (; k; k >>= 1) {
		if (k & 1)   c = c * a;
		a = a * a;
	}
	return c;
}\end{lstlisting}
\subsection{三维CDQ}
\begin{lstlisting}
struct node {
    // time: 时间 | id: 0,1 是否修改 | f: 正负 | x是下标 | y是权值
    int time, id, f, x, y;
};

bool cmp(const node &a, const node &b) {
    return a.x < b.x;
}

int ans[N];
node p[N << 2], et[N << 2];

void cdq(int l, int r) {
    if (l + 1 == r) return;
    int mid = (l + r) >> 1;
    cdq(l, mid), cdq(mid, r);
    int t = l;
    for (int i = mid; i < r; i++) {
        // if edit continue
        if (p[i].id)continue;
        for (; t < mid && p[t].x <= p[i].x; t++) {
            if (p[t].id) tree.change(p[t].y, p[t].f);;
        }
        int f = p[i].f;
        int cnt = tree.query(p[i].y);
        ans[p[i].time] += f * cnt;
    }
    // 逆操作p[t].y
    while (--t >= l) {
        if (p[t].id) tree.change(p[t].y, -p[t].f);
    }
    // 归并排序
    int t1 = l, t2 = mid, k = 0;
    while (t1 < mid && t2 < r) {
        et[k++] = p[t1].x < p[t2].x ? p[t1++] : p[t2++];
    }
    copy(p + t1, p + mid, et + k);
    copy(p + t2, p + r, et + k);
    copy(et, et + (r - l), p + l);
}\end{lstlisting}
\subsection{树分治-点分治}
\begin{lstlisting}
// 题意: n个节点的树,存在边权,范围1e18
// 求任意两点之间点集的子集中两点之间路径异或和为0的个数
// u<v,u'<v',(u',v') ∈ path(u,v),求path(u', v')异或和==0

struct Edge {
    int to, nxt;
    ll w;
};
const int N = int(1e5 + 10);
const int M = N << 1;

struct Grahp {
    int head[N];
    Edge eg[M];
    int tot;

    void init(int n) {
        memset(head, -1, sizeof(int) * ++n);
    }

    inline void addEdge(int u, int v, ll w) {
        eg[tot] = {v, head[u], w};
        head[u] = tot++;
    }
} gh;

bool vis[N];
// q队列, fa祖先, sz是子树大小, smx是子树最大
int q[N], fa[N], sz[N], smx[N];

int froot(int s) {
    int l, r, mn = N, rt = 0;
    q[l = r = 1] = s;
    while (l <= r) {
        int u = q[l++];
        sz[u] = 1;
        smx[u] = 0;
        for (int i = gh.head[u]; ~i; i = gh.eg[i].nxt) {
            int v = gh.eg[i].to;
            if (v == fa[u] || vis[v]) continue;
            fa[v] = u;
            q[++r] = v;
        }
    }
    // 反向遍历所有点算size
    while (--l) {
        int u = q[l];
        int mx = max(smx[u], r - sz[u]);
        if (mx < mn) mn = mx, rt = u;
        if (l == 1) break; // 根节点没有fa
        sz[fa[u]] += sz[u];
        smx[fa[u]] = max(smx[fa[u]], sz[u]);
    }
    return rt;
}

// sons子树方向节点个数, val根到该节点异或和, gc边后继方向的节点个数
int sons[N], gc[M];
ll val[N];
ll ans = 0;
int n;

const int MOD = int(1e9 + 7);

ll nums[N];
int cnt[N];

void go(int s, int rt) {
    fa[s] = rt;
    val[s] = 0;
    int l, r;
    // 不计算s
    q[l = r = 0] = s;
    int m = 0;
    while (l <= r) {
        int u = q[l++];
        nums[m++] = val[u];
        for (int i = gh.head[u]; ~i; i = gh.eg[i].nxt) {
            int v = gh.eg[i].to;
            if (v == fa[u] || vis[v]) continue;
            fa[v] = u;
            q[++r] = v;
            val[v] = val[u] ^ gh.eg[i].w;
            // 这个点方向后面有多少点
            sons[v] = gc[i];
        }
    }
    sort(nums, nums + m);
    m = unique(nums, nums + m) - nums;
    mst(cnt, 0, m);
    // 遍历分支
    for (int j = gh.head[s]; ~j; j = gh.eg[j].nxt) {
        // 分支的根
        int du = gh.eg[j].to;
        if (vis[du]) continue;
        q[l = r = 1] = du;
        while (l <= r) {
            int u = q[l++];
            int k = lower_bound(nums, nums + m, val[u]) - nums;
            (ans += 1ll * sons[u] * cnt[k] % MOD) %= MOD;
            if (val[u] == 0) {
                (ans += 1ll * sons[u] * (n - gc[j]) % MOD) %= MOD;
            }
            for (int i = gh.head[u]; ~i; i = gh.eg[i].nxt) {
                int v = gh.eg[i].to;
                if (v == fa[u] || vis[v]) continue;
                q[++r] = v;
            }
        }
        // 增加这个方向的值
        while (--l) {
            int u = q[l];
            int k = lower_bound(nums, nums + m, val[u]) - nums;
            (cnt[k] += sons[u]) %= MOD;
        }
    }
}

void work(int u) {
    // 换根
    u = froot(u);
    vis[u] = true;
    go(u, 0);
    for (int i = gh.head[u]; ~i; i = gh.eg[i].nxt) {
        int v = gh.eg[i].to;
        if (vis[v]) continue;
        work(v);
    }
}

// 预处理边后继节点个数
int pdfs(int u, int f) {
    int fg_id = -1;
    int s = 1;
    for (int i = gh.head[u]; ~i; i = gh.eg[i].nxt) {
        int v = gh.eg[i].to;
        if (v == f) { // 记录父边ID
            fg_id = i;
            continue;
        }
        int c = pdfs(v, u);
        gc[i] = c;
        s += c;
    }
    // 存在父边
    if (~fg_id) gc[fg_id] = n - s;
    return s;
}

void solve() {
    while (cin >> n) {
        gh.init(n);
        for (int i = 2; i <= n; i++) {
            int u, v;
            ll w;
            u = i;
            cin >> v >> w;
            gh.addEdge(u, v, w);
            gh.addEdge(v, u, w);
        }
        mst(vis, false, n + 1);
        pdfs(1, 0);
        ans = 0;
        work(1);
        cout << ans << endl;
    }
}\end{lstlisting}
\subsection{DsuOnTree1}
\begin{lstlisting}
/*
DSU On Tree 树上启发式合并
进行轻重链剖分,每次先递归轻链,然后消除轻儿子影响,然后计算重儿子,并且保留重儿子结果
然后以常数级别复杂度将重儿子状态修改成当前节点状态!!!(<------重点)
然后递归计算当前子树结果
复杂度O(nlogn)*O(计算/更新)

[CF741D]一棵有根树,边上有字母a~v,求每个子树中最长的边,满足这个边上的所有字母重排后可以构成回文。
很容易想到点分治,但由于是有根树,所以点分治是做不了的。
先对没给点记录xn[x]表示x到根的路径的各个字母出现奇偶性,然后b[S]记录满足xn[x]=S的所有点x的最大深度。
类似点分治,一棵子树一棵子树地处理,每次先对子树内每个点查找之前子树中是否有能与它拼成回文串的点与它的最大深度,再更新b[]。
注意查找和更新不能同时进行,要整棵子树都查询完毕之后再进行更新操作。注意查找与更新当前子树的根。
我试图去写了一个一次进出完整查询和更新的代码,逻辑是先查询后进入最后更新,但没有合适的写法,难度太大
 */

const int N = 500005;
const int M = N << 1;
struct Edge {
    int to, next;
};

struct Gragh {
    int head[N];
    Edge eg[M];
    int tot;

    void init(int n) {
        tot = 0;
        memset(head, -1, sizeof(int) * ++n);
    }

    inline void addEdge(int u, int v) {
        eg[tot] = {v, head[u]};
        head[u] = tot++;
    }
} gh;

int tsz[N], son[N], cson;
int dep[N] = {0};

void pdfs(int u, int f) {
    dep[u] = dep[f] + 1;
    tsz[u] = 1;
    son[u] = -1;
    for (int i = gh.head[u]; ~i; i = gh.eg[i].next) {
        int v = gh.eg[i].to;
        if (v == f) continue;
        pdfs(v, u);
        tsz[u] += tsz[v];
        if (son[u] == -1 || tsz[v] > tsz[son[u]]) son[u] = v;
    }
}

bool ans[N];
char s[N];
int p[N];

void calc(int u, int f, int d) {
    p[dep[u]] ^= (1 << (s[u] - 'a'));
    for (int i = gh.head[u]; ~i; i = gh.eg[i].next) {
        int v = gh.eg[i].to;
        if (v == f || v == cson) continue;
        calc(v, u, d);
    }
}

void dfs(int u, int f, int d) {
    // 遍历轻儿子,结果删除
    for (int i = gh.head[u]; ~i; i = gh.eg[i].next) {
        int v = gh.eg[i].to;
        if (v == f || v == son[u]) continue;
        dfs(v, u, 0);
    }
    if (~son[u]) { // 计算重儿子,结果保留
        dfs(son[u], u, 1);
        cson = son[u];
    }
    calc(u, f, 1);
    cson = -1;
    for (int i = qry.head[u]; ~i; i = qry.eg[i].next) {
        int h = qry.eg[i].to;
        ans[i] = __builtin_popcount(p[h]) <= 1;
    }
    if (d == 0) { // 删除当前子树结果
        calc(u, f, -1);
    }
}

void solve() {
    int n, m;
    while (cin >> n >> m) {
        gh.init(n);
        qry.init(n);
        for (int i = 2; i <= n; i++) {
            int u;
            cin >> u;
            gh.addEdge(u, i);
        }
        s[0] = ' ';
        cin >> (s + 1);
        rep(i, 0, m) {
            int v, h;
            cin >> v >> h;
            qry.addEdge(v, h);
        }
        pdfs(1, 0);
        dfs(1, 0, 0);
        rep(i, 0, m) {
            cout << (ans[i] ? "Yes" : "No") << endl;
        }
    }
}

const int N = 500005;
const int M = N << 1;
struct Edge {
    int to, next;
};

struct Gragh {
    int head[N];
    Edge eg[M];
    int tot;

    void init(int n) {
        tot = 0;
        memset(head, -1, sizeof(int) * ++n);
    }

    inline void addEdge(int u, int v) {
        eg[tot] = {v, head[u]};
        head[u] = tot++;
    }
} gh, qry;

int tsz[N], son[N], dep[N] = {0}, fa[N], cson;
int xn[N];
int a[N];

void pdfs(int u, int f) {
    fa[u] = f;
    dep[u] = dep[f] + 1;
    tsz[u] = 1;
    son[u] = -1;

    xn[u] = xn[f] ^ a[u];
    for (int i = gh.head[u]; ~i; i = gh.eg[i].next) {
        int v = gh.eg[i].to;
        if (v == f) continue;
        pdfs(v, u);
        tsz[u] += tsz[v];
        if (son[u] == -1 || tsz[v] > tsz[son[u]]) son[u] = v;
    }
}


int b[1 << 22];
int rst, ans[N], mx, root;
int num[N], tot = 0;

void check(int u) {
    int bs = xn[u];
    if (b[bs]) {
        rst = max(rst, dep[u] + b[bs] - dep[root] * 2);
    }
    for (int i = 0; i < 22; i++) {
        int bt = bs ^(1 << i);
        if (b[bt]) {
            rst = max(rst, dep[u] + b[bt] - dep[root] * 2);
        }
    }
    bs ^= xn[root];
    if (bs == (bs & (-bs))) {
        rst = max(rst, dep[u] - dep[root]);
    }
}

void calc(int u) {
    check(u);
    for (int i = gh.head[u]; ~i; i = gh.eg[i].next) {
        int v = gh.eg[i].to;
        if (v == fa[u]) continue;
        calc(v);
    }
}

void add(int u) {
    if (b[xn[u]]) {
        b[xn[u]] = max(b[xn[u]], dep[u]);
    } else {
        b[xn[u]] = dep[u];
        num[tot++] = xn[u];
    }
    if (u == cson) return;
    for (int i = gh.head[u]; ~i; i = gh.eg[i].next) {
        int v = gh.eg[i].to;
        if (v == fa[u]) continue;
        add(v);
    }
}

void go(int u) {
    rst = 0;
    if (~son[u]) {
        rst = 1;
        root = son[u];
        check(son[u]);
        add(son[u]);
    }
    root = u;
    check(u);
    for (int i = gh.head[u]; ~i; i = gh.eg[i].next) {
        int v = gh.eg[i].to;
        if (v == fa[u] || v == cson) continue;
        calc(v);
        add(v);
    }
}

void del() {
    for (int i = 0; i < tot; b[num[i++]] = 0);
    tot = 0;
}

void dfs(int u, int d) {
    // 遍历轻儿子,结果删除
    for (int i = gh.head[u]; ~i; i = gh.eg[i].next) {
        int v = gh.eg[i].to;
        if (v == fa[u] || v == son[u]) continue;
        dfs(v, 0);
    }
    if (~son[u]) { // 计算重儿子,结果保留
        dfs(son[u], 1);
        cson = son[u];
    }
    go(u); // 计算当前
    cson = -1;
    ans[u] = rst;
    if (d == 0) {
        del();
    }
}

void dfs2(int u) {
    for (int i = gh.head[u]; ~i; i = gh.eg[i].next) {
        int v = gh.eg[i].to;
        dfs2(v);
        ans[u] = max(ans[u], ans[v]);
    }
}

void input(int n) {
    a[1] = 0;
    for (int i = 2; i <= n; i++) {
        int f;
        char c;
        cin >> f >> c;
        gh.addEdge(f, i);
        a[i] = 1 << (c - 'a');
    }
}

void solve() {
    int n;
    while (cin >> n) {
        gh.init(n);
        input(n);
        memset(b, 0, sizeof(b));
        pdfs(1, 0);
        dfs(1, 0);
        dfs2(1);
        for (int i = 1; i <= n; i++) {
            printf("%d%c", ans[i], " \n"[i == n]);
        }
    }
}\end{lstlisting}
\subsection{DsuOnTree2}
\begin{lstlisting}
/*
[CF570D]一棵树,询问某棵子树指定深度(对于树根的深度)的点能否构成回文。
用二进制记录每个深度的每个字母出现次数的奇偶性即可。
 */

const int N = 500005;
const int M = N << 1;
struct Edge {
    int to, next;
};

struct Gragh {
    int head[N];
    Edge eg[M];
    int tot;

    void init(int n) {
        tot = 0;
        memset(head, -1, sizeof(int) * ++n);
    }

    inline void addEdge(int u, int v) {
        eg[tot] = {v, head[u]};
        head[u] = tot++;
    }
} gh, qry;

int tsz[N], son[N], cson;
int dep[N] = {0};

void pdfs(int u, int f) {
    dep[u] = dep[f] + 1;
    tsz[u] = 1;
    son[u] = -1;
    for (int i = gh.head[u]; ~i; i = gh.eg[i].next) {
        int v = gh.eg[i].to;
        if (v == f) continue;
        pdfs(v, u);
        tsz[u] += tsz[v];
        if (son[u] == -1 || tsz[v] > tsz[son[u]]) son[u] = v;
    }
}

bool ans[N];
char s[N];
int p[N];

void calc(int u, int f, int d) {
    p[dep[u]] ^= (1 << (s[u] - 'a'));
    for (int i = gh.head[u]; ~i; i = gh.eg[i].next) {
        int v = gh.eg[i].to;
        if (v == f || v == cson) continue;
        calc(v, u, d);
    }
}

void dfs(int u, int f, int d) {
    // 遍历轻儿子,结果删除
    for (int i = gh.head[u]; ~i; i = gh.eg[i].next) {
        int v = gh.eg[i].to;
        if (v == f || v == son[u]) continue;
        dfs(v, u, 0);
    }
    if (~son[u]) { // 计算重儿子,结果保留
        dfs(son[u], u, 1);
        cson = son[u];
    }
    calc(u, f, 1);
    cson = -1;
    for (int i = qry.head[u]; ~i; i = qry.eg[i].next) {
        int h = qry.eg[i].to;
        ans[i] = __builtin_popcount(p[h]) <= 1;
    }
    if (d == 0) { // 删除当前子树结果
        calc(u, f, -1);
    }
}

void solve() {
    int n, m;
    while (cin >> n >> m) {
        gh.init(n);
        qry.init(n);
        for (int i = 2; i <= n; i++) {
            int u;
            cin >> u;
            gh.addEdge(u, i);
        }
        s[0] = ' ';
        cin >> (s + 1);
        rep(i, 0, m) {
            int v, h;
            cin >> v >> h;
            qry.addEdge(v, h);
        }
        pdfs(1, 0);
        dfs(1, 0, 0);
        rep(i, 0, m) {
            cout << (ans[i] ? "Yes" : "No") << endl;
        }
    }
}\end{lstlisting}
\clearpage\section{数论}
\subsection{线性基}
\begin{lstlisting}
#define rep(i, l, r) for(int i=(l);i<(r);++i)
#define per(i, l, r) for(int i=(r)-1;i>=(l);--i)
#define pw(x) (1ll << (x))
#define bt(x, i) ((x >> i) & 1)
const int LN = 61;
struct LB {
    ll d[LN] = {0}, p[LN] = {0};
    int g[LN] = {0};
    int cnt = 0;

    LB() = default;

    ll &operator[](int pos) {
        return d[pos];
    }

    const ll &operator[](int pos) const {
        return d[pos];
    }

    void insert(ll val, int pos) {
        per(i, 0, LN) {
            if (val & pw(i)) {
                if (!d[i]) {
                    d[i] = val;
                    g[i] = pos;
                    return;
                }
				// 贪心保留最右
                if (pos > g[i]) {
                    swap(pos, g[i]);
                    swap(val, d[i]);
                }
                val ^= d[i];
            }
        }
    }

    ll query_max(int l) {
        ll ret = 0;
        per(i, 0, LN) {
            if (g[i] >= l)
                ret = max(ret, ret ^ d[i]);
        }
        return ret;
    }

    ll query_max() {
        ll ret = 0;
        per(i, 0, LN) {
            ret = max(ret, ret ^ d[i]);
        }
        return ret;
    }

    ll query_min() {
        rep(i, 0, LN) {
            if (d[i]) return d[i];
        }
        return 0;
    }
	
	bool test(ll x) const {
        per(i, 0, LN) {
            if (bt(x, i)) {
                if (!d[i]) return false;
                x ^= d[i];
            }
        }
        return true;
    }

    void rebuild() {
        per(i, 0, LN) {
            per(j, 0, i) {
                if (d[i] & (1LL << j)) d[i] ^= d[j];
            }
        }
        rep(i, 0, LN) {
            if (d[i]) p[cnt++] = d[i];
        }
    }

    ll kth_query(ll k) {
        int ret = 0;
        if (k >= pw(cnt)) return -1;
        per(i, 0, LN) {
            if (bt(k, 1)) ret ^= p[i];
        }
        return ret;
    }
};
// 求并集
LB operator+(const LB &n1, const LB &n2) {
    LB ret = n1;
    per(i, 0, LN)
        if (n2.d[i])
            ret.insert(n1.d[i], n1.g[i]);
    return ret;
}
// 求交集
LB operator^(const LB &n1, const LB &n2) {
    LB ans = {}, c = n2, d = n2;
    rep(i, 0, LN) {
        ll x = n1[i];
        if (!x) continue;
        int p = i;
        ll T = 0;
        per(j, 0, p + 1) {
            if (bt(x, j)) {
                if (c[j]) {
                    x ^= c[j];
                    T ^= d[j];
                } else {
                    p = j;
                    break;
                }
            }
        }
        if (!x) {
            ans[i] = T;
        } else {
            c[p] = x;
            d[p] = T;
        }
    }
    return ans;
}\end{lstlisting}
\subsection{FWT}
\begin{lstlisting}
void FWT(int p[], int n) { // 如果要取模运算记得-x+MOD,*inv
    for (int i = 1; i < n; i <<= 1) {
        for (int j = 0; j < n; j += (i << 1)) {
            for (int k = 0; k < i; k++) {
                // or{
                p[i + j + k] += +p[j + k];
                // and
                p[j + k] += p[i + j + k];
                // xor
                tie(p[j + k], p[i + j + k]) = make_pair(p[j + k] + p[i + j + k], p[j + k] - p[i + j + k]);
            }
        }
    }
}

void IFWT(int p[], int n) { // 如果要取模运算记得-x+MOD,*inv
    for (int i = 1; i < n; i <<= 1) {
        for (int j = 0; j < n; j += (i << 1)) {
            for (int k = 0; k < i; k++) {
                // or
                p[i + j + k] -= p[j + k];
                // and
                p[j + k] -= p[i + j + k];
                // xor
                tie(p[j + k], p[i + j + k]) = make_pair(p[j + k] + p[i + j + k], p[j + k] - p[i + j + k]);
                p[j + k] = p[j + k] / 2, p[i + j + k] = p[i + j + k] / 2;
            }
        }
    }
}\end{lstlisting}
\subsection{FFT-NTT}
\begin{lstlisting}
const int N = int(1e6 + 48);
const int MOD = 998244353;
const int pr = 3;
const int phi = MOD - 1;
const int sq2 = 116195171; 

int fpow(ll a, ll k) {
    ll c = 1;
    for (; k; k >>= 1) {
        if (k & 1) (c *= a) %= MOD;
        (a *= a) %= MOD;
    }
    return c;
}

int calc(int n) {
    int k = 0;
    while ((1 << k) < n) k++;
    return k;
}

int rk[N], rn = 0;

template<class T>
void change(T p[], int n) {
    int k = calc(n);
    if (rn != n) {
        rn = n;
        for (int i = 0; i < n; i++) rk[i] = (rk[i >> 1] >> 1) | ((i & 1) << (k - 1));
    }
    for (int i = 0; i < n; i++) if (i < rk[i]) swap(p[i], p[rk[i]]);
}

// FFT
const double pi = acos(-1.0);
using Complex = complex<double>;

void FFT(Complex p[], int n, int opt) {
    change(p, n);
    for (int i = 1; i < n; i <<= 1) { //待合并区间的长度的一半
        Complex wn(cos(pi / i), opt * sin(pi / i)); //单位根
        for (int R = i << 1, j = 0; j < n; j += R) { //R是区间的长度,j表示前已经到哪个位置了
            Complex w(1, 0); //幂
            for (int k = 0; k < i; ++k, w = w * wn) { //枚举左半部分
                Complex x = p[j + k], y = w * p[j + i + k]; //蝴蝶效应
                p[j + k] = x + y;
                p[j + i + k] = x - y;
            }
        }
    }
}

void NTT(ll p[], int n, int opt) {
    change(p, n);
    for (int i = 1; i < n; i <<= 1) {
        ll wn = fpow(pr, phi / (i << 1));
        for (int R = i << 1, j = 0; j < n; j += R) {
            ll w = 1;
            for (int k = 0; k < i; ++k, w = w * wn % MOD) {
                ll x = p[j + k], y = w * p[i + j + k] % MOD;
                p[j + k] = (x + y) % MOD;
                p[i + j + k] = (x - y + MOD) % MOD;
            }
        }
    }
    if (opt == -1)reverse(p + 1, p + n);
}\end{lstlisting}
\subsection{圆上整点}
\begin{lstlisting}
// x^2+y^2=r^2 --> y^2=(r-x)*(r+x)
// d = gcd(r-x,r+x), n=(r-x)/d, m=(r+x)/d, y^2=d^2*m*n
// 因为y^2和d^2为完全平方数,所以 n*m 为完全平方数
// 又因为n,m互质,所以 n=u^2, m =v^2
// 所以 r-y=d*u^2, r+y=d*v^2
// 可得 x=d*(v^2-u^2), y=d*u*v

ll gcd(ll a, ll b) { return !b ? a : gcd(b, a % b); }

inline int work(vector<pll> &p, ll d, ll t) { // d * t == 2 * r
    int sum = 0;
    for (ll u = 1; 2 * u * u < t; u++) {
        ll v = ll(sqrt(t - u * u));
        if (v == u || v * v + u * u != t || gcd(u, v) != 1) continue;
        ll x = d * (v * v - u * u) / 2;
        ll y = d * u * v;
        p.push_back({x, y});
        sum++;
    }
    return sum;
}

int calc(vector<pll> &p, ll r) {
    int sum = 0;
    r <<= 1;
    for (ll i = 1; i * i <= r; i++) { // sqrt 枚举因子
        if (r % i) continue;
        sum += work(p, i, r / i);
        if (i * i < r) sum += work(p, r / i, i);
    }
    for (int i = 0, cnt = p.size(); i < cnt; i++) {
        int x = p[i].fi, y = p[i].se;
        p.push_back({x, -y}), p.push_back({-x, y}), p.push_back({-x, -y});
    }
    sum <<= 2;
    r >>= 1;
    sum += 4;
    p.push_back({r, 0}), p.push_back({-r, 0}), p.push_back({0, r}), p.push_back({0, -r});
    sort(p.begin(), p.end());
    return sum;
}\end{lstlisting}
\subsection{线性逆元}
\begin{lstlisting}
// 逆元,阶乘,阶乘逆元
ll inv[N] = {1, 1}, fac[N] = {1, 1}, ifac[N] = {1, 1};
// 线性求逆元
void init() {
	// 如果用的Int记得下面inv[i] = 1ll*...
    for (int i = 2; i < N; i++) {
        inv[i] = (MOD - (MOD / i)) * inv[MOD % i] % MOD;
        fac[i] = fac[i - 1] * i % MOD;
        ifac[i] = ifac[i - 1] * inv[i] % MOD;
    }
}\end{lstlisting}
\clearpage\section{其他}
\subsection{BigInteger}
\begin{lstlisting}
// base and base_digits must be consistent
constexpr int base = 1000000000;
constexpr int base_digits = 9;

struct bigint {
    // value == 0 is represented by empty z
    vector<int> z; // digits

    // sign == 1 <==> value >= 0
    // sign == -1 <==> value < 0
    int sign;

    bigint() : sign(1) {}

    bigint(ll v) { *this = v; }

    bigint &operator=(ll v) {
        sign = v < 0 ? -1 : 1;
        v *= sign;
        z.clear();
        for (; v > 0; v = v / base) z.push_back((int) (v % base));
        return *this;
    }

    bigint(const string &s) { read(s); }

    bigint &operator+=(const bigint &other) {
        if (sign == other.sign) {
            for (int i = 0, carry = 0; i < other.z.size() || carry; ++i) {
                if (i == z.size())
                    z.push_back(0);
                z[i] += carry + (i < other.z.size() ? other.z[i] : 0);
                carry = z[i] >= base;
                if (carry)
                    z[i] -= base;
            }
        } else if (other != 0 /* prevent infinite loop */) {
            *this -= -other;
        }
        return *this;
    }

    friend bigint operator+(bigint a, const bigint &b) { return a += b; }

    bigint &operator-=(const bigint &other) {
        if (sign == other.sign) {
            if (sign == 1 && *this >= other || sign == -1 && *this <= other) {
                for (int i = 0, carry = 0; i < other.z.size() || carry; ++i) {
                    z[i] -= carry + (i < other.z.size() ? other.z[i] : 0);
                    carry = z[i] < 0;
                    if (carry)
                        z[i] += base;
                }
                trim();
            } else {
                *this = other - *this;
                this->sign = -this->sign;
            }
        } else {
            *this += -other;
        }
        return *this;
    }

    friend bigint operator-(bigint a, const bigint &b) {
        return a -= b;
    }

    bigint &operator*=(int v) {
        if (v < 0) sign = -sign, v = -v;
        for (int i = 0, carry = 0; i < z.size() || carry; ++i) {
            if (i == z.size()) z.push_back(0);
            ll cur = (ll) z[i] * v + carry;
            carry = (int) (cur / base);
            z[i] = (int) (cur % base);
        }
        trim();
        return *this;
    }

    bigint operator*(int v) const { return bigint(*this) *= v; }

    friend pair<bigint, bigint> divmod(const bigint &a1, const bigint &b1) {
        int norm = base / (b1.z.back() + 1);
        bigint a = a1.abs() * norm;
        bigint b = b1.abs() * norm;
        bigint q, r;
        q.z.resize(a.z.size());

        for (int i = (int) a.z.size() - 1; i >= 0; i--) {
            r *= base;
            r += a.z[i];
            int s1 = b.z.size() < r.z.size() ? r.z[b.z.size()] : 0;
            int s2 = b.z.size() - 1 < r.z.size() ? r.z[b.z.size() - 1] : 0;
            int d = (int) (((ll) s1 * base + s2) / b.z.back());
            r -= b * d;
            while (r < 0) r += b, --d;
            q.z[i] = d;
        }

        q.sign = a1.sign * b1.sign;
        r.sign = a1.sign;
        q.trim();
        r.trim();
        return {q, r / norm};
    }

    friend bigint sqrt(const bigint &a1) {
        bigint a = a1;
        while (a.z.empty() || a.z.size() % 2 == 1) a.z.push_back(0);

        int n = a.z.size();

        int firstDigit = (int) ::sqrt((double) a.z[n - 1] * base + a.z[n - 2]);
        int norm = base / (firstDigit + 1);
        a *= norm;
        a *= norm;
        while (a.z.empty() || a.z.size() % 2 == 1) a.z.push_back(0);

        bigint r = (ll) a.z[n - 1] * base + a.z[n - 2];
        firstDigit = (int) ::sqrt((double) a.z[n - 1] * base + a.z[n - 2]);
        int q = firstDigit;
        bigint res;

        for (int j = n / 2 - 1; j >= 0; j--) {
            for (;; --q) {
                bigint r1 = (r - (res * 2 * base + q) * q) * base * base +
                            (j > 0 ? (ll) a.z[2 * j - 1] * base + a.z[2 * j - 2] : 0);
                if (r1 >= 0) {
                    r = r1;
                    break;
                }
            }
            (res *= base) += q;

            if (j > 0) {
                int d1 = res.z.size() + 2 < r.z.size() ? r.z[res.z.size() + 2] : 0;
                int d2 = res.z.size() + 1 < r.z.size() ? r.z[res.z.size() + 1] : 0;
                int d3 = res.z.size() < r.z.size() ? r.z[res.z.size()] : 0;
                q = (int) (((ll) d1 * base * base + (ll) d2 * base + d3) / (firstDigit * 2));
            }
        }

        res.trim();
        return res / norm;
    }

    bigint operator/(const bigint &v) const {
        return divmod(*this, v).first;
    }

    bigint operator%(const bigint &v) const {
        return divmod(*this, v).second;
    }

    bigint &operator/=(int v) {
        if (v < 0) sign = -sign, v = -v;
        for (int i = (int) z.size() - 1, rem = 0; i >= 0; --i) {
            ll cur = z[i] + rem * (ll) base;
            z[i] = (int) (cur / v);
            rem = (int) (cur % v);
        }
        trim();
        return *this;
    }

    bigint operator/(int v) const {
        return bigint(*this) /= v;
    }

    int operator%(int v) const {
        if (v < 0) v = -v;
        int m = 0;
        for (int i = (int) z.size() - 1; i >= 0; --i)
            m = (int) ((z[i] + m * (ll) base) % v);
        return m * sign;
    }

    bigint &operator*=(const bigint &v) {
        return *this = *this * v;;
    }

    bigint &operator/=(const bigint &v) {
        return *this = *this / v;
    }

    bool operator<(const bigint &v) const {
        if (sign != v.sign)
            return sign < v.sign;
        if (z.size() != v.z.size())
            return z.size() * sign < v.z.size() * v.sign;
        for (int i = (int) z.size() - 1; i >= 0; i--)
            if (z[i] != v.z[i])
                return z[i] * sign < v.z[i] * sign;
        return false;
    }

    bool operator>(const bigint &v) const { return v < *this; }

    bool operator<=(const bigint &v) const { return !(v < *this); }

    bool operator>=(const bigint &v) const { return !(*this < v); }

    bool operator==(const bigint &v) const { return !(*this < v) && !(v < *this); }

    bool operator!=(const bigint &v) const { return *this < v || v < *this; }

    void trim() {
        while (!z.empty() && z.back() == 0) z.pop_back();
        if (z.empty()) sign = 1;
    }

    bool isZero() const {
        return z.empty();
    }

    friend bigint operator-(bigint v) {
        if (!v.z.empty()) v.sign = -v.sign;
        return v;
    }

    bigint abs() const {
        return sign == 1 ? *this : -*this;
    }

    ll longValue() const {
        ll res = 0;
        for (int i = (int) z.size() - 1; i >= 0; i--)
            res = res * base + z[i];
        return res * sign;
    }

    friend bigint gcd(const bigint &a, const bigint &b) {
        return b.isZero() ? a : gcd(b, a % b);
    }

    friend bigint lcm(const bigint &a, const bigint &b) {
        return a / gcd(a, b) * b;
    }

    void read(const string &s) {
        sign = 1;
        z.clear();
        int pos = 0;
        while (pos < s.size() && (s[pos] == '-' || s[pos] == '+')) {
            if (s[pos] == '-') sign = -sign;
            ++pos;
        }
        for (int i = (int) s.size() - 1; i >= pos; i -= base_digits) {
            int x = 0;
            for (int j = max(pos, i - base_digits + 1); j <= i; j++)
                x = x * 10 + s[j] - '0';
            z.push_back(x);
        }
        trim();
    }

    friend istream &operator>>(istream &stream, bigint &v) {
        string s;
        stream >> s;
        v.read(s);
        return stream;
    }

    friend ostream &operator<<(ostream &stream, const bigint &v) {
        if (v.sign == -1)
            stream << '-';
        stream << (v.z.empty() ? 0 : v.z.back());
        for (int i = (int) v.z.size() - 2; i >= 0; --i)
            stream << setw(base_digits) << setfill('0') << v.z[i];
        return stream;
    }

    static vector<int> convert_base(const vector<int> &a, int old_digits, int new_digits) {
        vector<ll> p(max(old_digits, new_digits) + 1);
        p[0] = 1;
        for (int i = 1; i < p.size(); i++)
            p[i] = p[i - 1] * 10;
        vector<int> res;
        ll cur = 0;
        int cur_digits = 0;
        for (int v : a) {
            cur += v * p[cur_digits];
            cur_digits += old_digits;
            while (cur_digits >= new_digits) {
                res.push_back(int(cur % p[new_digits]));
                cur /= p[new_digits];
                cur_digits -= new_digits;
            }
        }
        res.push_back((int) cur);
        while (!res.empty() && res.back() == 0) res.pop_back();
        return res;
    }

    typedef vector<ll> vll;

    static vll karatsubaMultiply(const vll &a, const vll &b) {
        int n = a.size();
        vll res(n + n);
        if (n <= 32) {
            for (int i = 0; i < n; i++)
                for (int j = 0; j < n; j++)
                    res[i + j] += a[i] * b[j];
            return res;
        }

        int k = n >> 1;
        vll a1(a.begin(), a.begin() + k);
        vll a2(a.begin() + k, a.end());
        vll b1(b.begin(), b.begin() + k);
        vll b2(b.begin() + k, b.end());

        vll a1b1 = karatsubaMultiply(a1, b1);
        vll a2b2 = karatsubaMultiply(a2, b2);

        for (int i = 0; i < k; i++) a2[i] += a1[i];
        for (int i = 0; i < k; i++) b2[i] += b1[i];

        vll r = karatsubaMultiply(a2, b2);
        for (int i = 0; i < a1b1.size(); i++) r[i] -= a1b1[i];
        for (int i = 0; i < a2b2.size(); i++) r[i] -= a2b2[i];

        for (int i = 0; i < r.size(); i++) res[i + k] += r[i];
        for (int i = 0; i < a1b1.size(); i++) res[i] += a1b1[i];
        for (int i = 0; i < a2b2.size(); i++) res[i + n] += a2b2[i];
        return res;
    }

    bigint operator*(const bigint &v) const {
        vector<int> a6 = convert_base(this->z, base_digits, 6);
        vector<int> b6 = convert_base(v.z, base_digits, 6);
        vll a(a6.begin(), a6.end());
        vll b(b6.begin(), b6.end());
        while (a.size() < b.size()) a.push_back(0);
        while (b.size() < a.size()) b.push_back(0);
        while (a.size() & (a.size() - 1)) a.push_back(0), b.push_back(0);
        vll c = karatsubaMultiply(a, b);
        bigint res;
        res.sign = sign * v.sign;
        for (int i = 0, carry = 0; i < c.size(); i++) {
            ll cur = c[i] + carry;
            res.z.push_back((int) (cur % 1000000));
            carry = (int) (cur / 1000000);
        }
        res.z = convert_base(res.z, 6, base_digits);
        res.trim();
        return res;
    }
};
\end{lstlisting}
\subsection{FastIO}
\begin{lstlisting}
/*
 * FastIO
 * 代码模板 !
 * 如有雷同 !
 * 纯属巧合 !
 */
namespace FastIO {
#define BUF_SIZE 10000000
#define OUT_SIZE 10000000
#define ll long long
    //fread->read
    bool IOerror = 0;

    inline char nc() {
        static char buf[BUF_SIZE], *p1 = buf + BUF_SIZE, *pend = buf + BUF_SIZE;
        if (p1 == pend) {
            p1 = buf;
            pend = buf + fread(buf, 1, BUF_SIZE, stdin);
            if (pend == p1) {
                IOerror = 1;
                return -1;
            }
            //{printf("IO error!\n");system("pause");for (;;);exit(0);}
        }
        return *p1++;
    }

    inline bool blank(char ch) { return ch == ' ' || ch == '\n' || ch == '\r' || ch == '\t'; }

    inline void read(int &x) {
        bool sign = 0;
        char ch = nc();
        x = 0;
        for (; blank(ch); ch = nc());
        if (IOerror)return;
        if (ch == '-')sign = 1, ch = nc();
        for (; ch >= '0' && ch <= '9'; ch = nc())x = x * 10 + ch - '0';
        if (sign)x = -x;
    }

    inline void read(ll &x) {
        bool sign = 0;
        char ch = nc();
        x = 0;
        for (; blank(ch); ch = nc());
        if (IOerror)return;
        if (ch == '-')sign = 1, ch = nc();
        for (; ch >= '0' && ch <= '9'; ch = nc())x = x * 10 + ch - '0';
        if (sign)x = -x;
    }

    inline void read(double &x) {
        bool sign = 0;
        char ch = nc();
        x = 0;
        for (; blank(ch); ch = nc());
        if (IOerror)return;
        if (ch == '-')sign = 1, ch = nc();
        for (; ch >= '0' && ch <= '9'; ch = nc())x = x * 10 + ch - '0';
        if (ch == '.') {
            double tmp = 1;
            ch = nc();
            for (; ch >= '0' && ch <= '9'; ch = nc())tmp /= 10.0, x += tmp * (ch - '0');
        }
        if (sign)x = -x;
    }

    inline void read(char *s) {
        char ch = nc();
        for (; blank(ch); ch = nc());
        if (IOerror)return;
        for (; !blank(ch) && !IOerror; ch = nc())*s++ = ch;
        *s = 0;
    }

    inline void read(char &c) {
        for (c = nc(); blank(c); c = nc());
        if (IOerror) {
            c = -1;
            return;
        }
    }

    //fwrite->write
    struct Ostream_fwrite {
        char *buf, *p1, *pend;
        Ostream_fwrite() {
            buf = new char[OUT_SIZE];
            p1 = buf;
            pend = buf + OUT_SIZE;
        }
        void out(char ch) {
            if (p1 == pend) {
                fwrite(buf, 1, OUT_SIZE, stdout);
                p1 = buf;
            }
            *p1++ = ch;
        }
        void print(int x) {
            static char s[15], *s1;
            s1 = s;
            if (!x)*s1++ = '0';
            if (x < 0)out('-'), x = -x;
            while (x)*s1++ = x % 10 + '0', x /= 10;
            while (s1-- != s)out(*s1);
        }
        void println(int x) {
            static char s[15], *s1;
            s1 = s;
            if (!x)*s1++ = '0';
            if (x < 0)out('-'), x = -x;
            while (x)*s1++ = x % 10 + '0', x /= 10;
            while (s1-- != s)out(*s1);
            out('\n');
        }
        void print(ll x) {
            static char s[25], *s1;
            s1 = s;
            if (!x)*s1++ = '0';
            if (x < 0)out('-'), x = -x;
            while (x)*s1++ = x % 10 + '0', x /= 10;
            while (s1-- != s)out(*s1);
        }
        void println(ll x) {
            static char s[25], *s1;
            s1 = s;
            if (!x)*s1++ = '0';
            if (x < 0)out('-'), x = -x;
            while (x)*s1++ = x % 10 + '0', x /= 10;
            while (s1-- != s)out(*s1);
            out('\n');
        }
        void print(double x, int y) {
            static ll mul[] = {1, 10, 100, 1000, 10000, 100000, 1000000, 10000000, 100000000,
                               1000000000, 10000000000LL, 100000000000LL, 1000000000000LL, 10000000000000LL,
                               100000000000000LL, 1000000000000000LL, 10000000000000000LL, 100000000000000000LL};
            if (x < -1e-12)out('-'), x = -x;
            x *= mul[y];
            ll x1 = (ll) floor(x);
            if (x - floor(x) >= 0.5)++x1;
            ll x2 = x1 / mul[y], x3 = x1 - x2 * mul[y];
            print(x2);
            if (y > 0) {
                out('.');
                for (size_t i = 1; i < y && x3 * mul[i] < mul[y]; out('0'), ++i);
                print(x3);
            }
        }
        void println(double x, int y) {
            print(x, y);
            out('\n');
        }
        void print(char *s) { while (*s)out(*s++); }
        void println(char *s) {
            while (*s)out(*s++);
            out('\n');
        }
        void flush() {
            if (p1 != buf) {
                fwrite(buf, 1, p1 - buf, stdout);
                p1 = buf;
            }
        }
        ~Ostream_fwrite() { flush(); }
    } Ostream;
    inline void print(int x) { Ostream.print(x); }
    inline void println(int x) { Ostream.println(x); }
    inline void print(char x) { Ostream.out(x); }
    inline void println(char x) {
        Ostream.out(x);
        Ostream.out('\n');
    }
    inline void print(ll x) { Ostream.print(x); }
    inline void println(ll x) { Ostream.println(x); }
    inline void print(double x, int y) { Ostream.print(x, y); }
    inline void println(double x, int y) { Ostream.println(x, y); }
    inline void print(char *s) { Ostream.print(s); }
    inline void println(char *s) { Ostream.println(s); }
    inline void println() { Ostream.out('\n'); }
    inline void flush() { Ostream.flush(); }
};
using namespace FastIO;\end{lstlisting}
\subsection{InputOutputSpeedUp}
\begin{lstlisting}


template <class T>
inline bool read(T &x) {
    x = 0;
    char c = getchar();
	if(c == EOF) return false;
    bool f = false;
    for (; !isdigit(c); c = getchar()) f ^= (c == '-');
    for (; isdigit(c); c = getchar()) x = x * 10 + (c - '0');
    x = f ? -x : x;
	return true;
}
 
template <class T>
inline void write(T x) {
    if (x < 0) {
        putchar('-');
        x = -x;
    }
    T y = 1;
    int len = 1;
    for (; y <= x / 10; y *= 10) ++len;
    for (; len; --len, x %= y, y /= 10) putchar(x / y + '0');
}\end{lstlisting}
\subsection{gcd}
\begin{lstlisting}
ll gcd(ll x, ll y) { // 循环版
	ll t;
	while (y){
		t = x % y;
		x = y;
		y = t;
	}
	return  x;
}

ll gcd(ll a, ll b) { // 递归版
    return b == 0 ? a : gcd(b, a % b);
}

// 扩展欧几里得
ll exgcd(ll a, ll b, ll &x, ll &y) {
    if (b == 0) {
        x = 1, y = 0;
        return a;
    }
    ll q = exgcd(b, a % b, y, x);
    y -= a / b * x;
    return q;
}\end{lstlisting}
\subsection{myItoa}
\begin{lstlisting}
char * myItoa(int value, char* result, int base = 10);

char * myItoa(int value, char* result, int base) {
	// check that the base if valid  

	if (base < 2 || base > 16) { *result = 0; return result; }
	char* out = result;
	int quotient = abs(value);
	do {
		const int tmp = quotient / base;
		*out = "0123456789abcdef"[quotient - (tmp*base)];
		++out;
		quotient = tmp;
	} while (quotient);
	// Apply negative sign  
	if (value < 0) *out++ = '-';
	std::reverse(result, out);
	*out = 0;
	return result;
}\end{lstlisting}
\subsection{Permutation}
\begin{lstlisting}
// 错排问题
// D(n) = n! [(-1)^2/2! + … + (-1)^(n-1)/(n-1)! + (-1)^n/n!].
long long table[1000] = {0, 0, 1};
void init() {
    for (int i = 3; i <= 20; i++) {
        table[i] = (i - 1) * (table[i - 1] + table[i - 2]);
    }
}\end{lstlisting}
\subsection{prime}
\begin{lstlisting}
// 普通素数筛
const int PMAX = 1000000;

int prime_count = 0;
bool prime_list[PMAX] = { false };//元素值为0代表是素数
int prime_table[PMAX] = { 0 };

void initPrime() {
    for (int i = 2; i < PMAX; i++) {
        if (!prime_list[i])
            prime_table[prime_count++] = i;
        for (int j = 0, e = (PMAX - 1) / i;
             j < prime_count && prime_table[j] <= e; j++) {
            prime_list[i * prime_table[j]] = true;
            if (i % prime_table[j] == 0) break;
        }
    }
}


// 可以得到其中一个质因子的素数筛
const int PMAX = 1000005;

int prime_count = 0;
int prime_list[PMAX] = {0}; //元素值为0代表是素数
int prime_table[PMAX] = {0};

void initPrime() {
    for (int i = 2; i < PMAX; i++) {
        if (!prime_list[i])
            prime_list[i] = prime_table[prime_count++] = i;
        for (int j = 0, e = (PMAX - 1) / i, now;
             j < prime_count && prime_table[j] <= e; j++) {
            prime_list[i * now] = now;
            if (i % now == 0) break;
        }
    }
}\end{lstlisting}
\subsection{Hash}
\begin{lstlisting}
struct Hash {
    int num[N];
    int tot;

    void init() { tot = 0; }

    void insert(int x) { num[tot++] = x; }

    void build() {
        sort(num, num + tot);
        tot = unique(num, num + tot) - num;
    }

    inline int operator[](int x) { return lower_bound(num, num + tot, x) - num; }
} hs;\end{lstlisting}

\end{document}
